%% This is AMSLaTeX source, included by manual_incl.tex.


%%--------------------------------------------------------------------
\section{The Back End}

%%....................................................................
\subsection{Introduction}


The back end is comprised of C++ classes, and global functions which
act on objects in these classes.

We divide the classes into three conceptual types: user, utility, and
private.

\lskip
User classes wrap pointers to object representations
(\S\ref{obj_reps}). They are intended as high-level constructs with
fixed interfaces and intuitive semantics.  As objects in such classes
occupy only as much memory as a pointer, they can be efficiently
copied and assigned.  The typical use of objects in a user class is
through automatic variables and class data members. Only in
extraordinary circumstances would one have a pointer to such an
object; the object itself takes no more space, and construction,
assignment, and destruction are handled automatically.

\lskip
Utility classes are those which do useful, general things, but which
would not reasonably use wrapped object representations.  For example,
a \code{Cell} class might have two data members: an object of a user
class and a pointer to a \code{Cell}; this could be used for
externally linked lists. To be reusable, utility class interfaces and
semantics are fixed.

\lskip
Private classes are those which the implementer uses to hide data
representations or perform other hidden tasks. As they are not
intended to be reusable, their interfaces and functionality are
subject to change without notice.


%% . . . . . . . . . . . . . . . . . . . . . . . . . . . . . . . . . .
\subsubsection{Object Representations}\label{obj_reps}

In the section {\em Writing User Classes} of the {\em Project
Guidelines} chapter we described a simple scheme for wrapping pointers
to object representations using the classes
\code{ObjectOf} and \code{RefCounter}. This scheme is a legacy. The
diagram below describes a more general scheme which better supports
inheritance. The class {\sf A} uses the old paradigm. The hierarchy of
classes {\sf B}, {\sf C}, {\sf D} use the new one.

\vspace{20pt}

\epsfbox{ps/objects.ps}

\vspace{20pt}




%%....................................................................
\subsection{General}

The directory \file{magnus/back\_end/general} contains a number of user
and utility classes of general interest.



%% . . . . . . . . . . . . . . . . . . . . . . . . . . . . . . . . . .
\subsubsection{User class \code{AssociationsOf<Key,Val>}}

\subsubheading{What and Where}

\noindent The class\\
\code{template <class Key, class Val> class AssociationsOf :\\
public ObjectOf< AssociationsRep<Key,Val> >}\\
is defined in \file{magnus/back\_end/general/include/Associations.h}.\\
It implements simple association lists.


\subsubheading{Special Notes}

\noindent Class \code{Key} must have an \code{==} operator.\\
Class \code{Val} must have an assignment operator.\\
Both must have copy constructors and destructors.\\
There must be an\\
\code{ostream\& operator << ( ostream\&, const Type\& )}\\
and a conversion of \code{Key} to \code{Type}.

\noindent Note that if \code{Key} or \code{Val} is not a user class then
\code{AssociationsOf<Key,Val>} will perform inefficient copying operations.


%% . . . . . . . . . . . . . . . . . . . . . . . . . . . . . . . . . .
\subsubsection{User class \code{AssociationsIterator<Key,Val>}}

\subsubheading{What and Where}

\noindent The class\\
\code{template <class Key, class Val> class AssociationsIterator :\\
public ObjectOf< AssociationsIteratorRep<Key,Val> >}\\
is defined in \file{magnus/back\_end/general/include/Associations.h}.\\
It is for iterating over an \code{AssociationsOf<Key,Val>}.


\subsubheading{Special Notes}

\begin{itemize}
\item
A copy constructor, assignment operator, and destructor are
supplied by the compiler.

\item
Copying or assigning an iterator is guaranteed to produce an
iterator which visits the (rest of the) same elements in the
same order as the original.

\item
All iterators work the same way. See \S\ref{setiterator} for examples.
\end{itemize}


%% . . . . . . . . . . . . . . . . . . . . . . . . . . . . . . . . . .
\subsubsection{Utility class \code{BlackBox}}

\subsubheading{What and Where}

\noindent The class \code{class BlackBox}
is defined in \file{magnus/back\_end/general/include/BlackBox.h}.

\noindent A \code{BlackBox} is a wrapper for an external binary
which communicates exclusively through its standard I/O.
The binary must accept some sort of `quit' command from its
standard input.


% \subsubheading{Special Notes}


\subsubheading{Public Members}

\defmember{Constructor}{BlackBox(const char* \var{command})}{
Instantiate a \code{BlackBox} by supplying a sh command which invokes the binary.
Remember to call \code{sanityCheck} afterwards, even with no arg, to see if
everything is ok.
}

\defmethod{Bool sanityCheck(const char* \var{greeting} = NULL)}{
\var{greeting} is a (possibly proper) prefix of the first line you expect the
binary to write to \code{cout}. We might use regular expressions some day.
This returns \code{TRUE} iff it finds \var{greeting} (if not \code{NULL}), and
all other initialization was successful.
}

\defmethod{ostream\& toCommand( )}{
Returns an \code{ostream} bound to the binary's standard input.
}

\defmethod{istream\& fromCommand( )}{
Returns an \code{istream} bound to the binary's standard output.
}


%% . . . . . . . . . . . . . . . . . . . . . . . . . . . . . . . . . .
\subsubsection{Utility class \code{Cell<T>}}

\subsubheading{What and Where}

\noindent The class \code{template<class T> class Cell}\\
is defined in \file{magnus/back\_end/general/include/Cell.h}.

\noindent A \code{Cell<T>} is the basic component of an
externally linked list of \code{T}.


%% . . . . . . . . . . . . . . . . . . . . . . . . . . . . . . . . . .
\subsubsection{User class \code{Chars}}

\subsubheading{What and Where}

\noindent The class \code{class Chars : public ObjectOf<CharsRep>}\\
is defined in \file{magnus/back\_end/general/include/Chars.h}.

\noindent A \code{Chars} is a \code{char*} which knows its length,
wrapped up as a user class.


\subsubheading{Special Notes}

\noindent Revision 1.2 badly needs an overhaul.


%% . . . . . . . . . . . . . . . . . . . . . . . . . . . . . . . . . .
\subsubsection{User class \code{ListOf<T>}}

\subsubheading{What and Where}

\noindent The class\\
\code{template <class T> class ListOf : ObjectOf< ListRep<T> >}\\
is defined in \file{magnus/back\_end/general/include/List.h}.

\noindent A \code{ListOf<T>} is good for ordered sequences of \code{T}
which change size or composition frequently, at the cost of O(n) access.



\subsubheading{Special Notes}

\noindent Class \code{T} must have

\begin{enumerate}
\item
A copy constructor
\item
A destructor
\item
An assignment operator
\item
An \code{==} operator
\end{enumerate}

\noindent There must be an

%@rn\begin{verbatim}
\code{ostream\& operator << ( ostream\&, const Type\& )}
%\end{verbatim}

and a conversion of \code{T} to \code{Type}.

\noindent Some \code{ListOf} methods can safely ignore bad indices,
but they post a warning if \code{SAFETY > 0} on the assumption that
the caller is probably bugged.

\noindent A Copy constructor, assignment operator,
and destructor are supplied by the compiler.


%% . . . . . . . . . . . . . . . . . . . . . . . . . . . . . . . . . .
\subsubsection{User class \code{ListIterator<ListType>}}

\subsubheading{What and Where}

\noindent The class\\
\code{template <class ListType> class ListIterator :\\
public ObjectOf< ListIteratorRep<ListType::ListElementType> >}\\
is defined in \file{magnus/back\_end/general/include/List.h}.

It is for iterating over lists.


\subsubheading{Special Notes}

\noindent A Copy constructor, assignment operator,
and destructor are supplied by the compiler.

\noindent Copying or assigning an iterator is guaranteed to produce an
iterator which visits the (rest of the) same elements in the
same order as the original.

\noindent All iterators work the same way. See \S\ref{setiterator} for examples.


%% . . . . . . . . . . . . . . . . . . . . . . . . . . . . . . . . . .
\subsubsection{User class \code{RandomInt}}

\subsubheading{What and Where}

\noindent The class\\
\code{class RandomInt : ObjectOf<RandomIntRep>}\\
is defined in \file{magnus/back\_end/general/include/RandomInt.h}.


\subsubheading{Special Notes}

Likely to become obsolete.


%% . . . . . . . . . . . . . . . . . . . . . . . . . . . . . . . . . .
\subsubsection{Utility class \code{IStreamPoll}}

\subsubheading{What and Where}

\noindent The class \code{class IStreamPoll}\\
is defined in \file{magnus/back\_end/general/include/IStreamPoll.h}.


\subsubheading{Special Notes}

To appear.


%% . . . . . . . . . . . . . . . . . . . . . . . . . . . . . . . . . .
\subsubsection{User class \code{SetOf<T>}}

\subsubheading{What and Where}

\noindent The class\\
\code{template<class T> class SetOf : public ObjectOf< SetData<T> >}\\
is defined in \file{magnus/back\_end/general/include/Set.h}.

\noindent A \code{SetOf<T>} is good for an unordered collection of \code{T}.
It features fast access and modification.


\subsubheading{Special Notes}

\noindent Class \code{T} must have

\begin{enumerate}
\item
An assignment operator
\item
A copy constructor
\item
An \code{==} operator
\item
A destructor
\item
A method \code{int hash() const} which need not return positive \code{ints}
\end{enumerate}

\noindent A Copy constructor, assignment operator,
and destructor are supplied by the compiler.


%% . . . . . . . . . . . . . . . . . . . . . . . . . . . . . . . . . .
\subsubsection{User class \code{SetIterator<T>}}\label{setiterator}

\subsubheading{What and Where}

\noindent The class\\
\code{template <class T> class SetIterator :\\
public ObjectOf< SetIteratorData<T> >}\\
is defined in \file{magnus/back\_end/general/include/Set.h}.

It is for iterating over a set.


\subsubheading{Special Notes}

\noindent A Copy constructor, assignment operator,
and destructor are supplied by the compiler.

\noindent Examples of use:

\begin{itemize}
\item
To iterate over the elements of a \code{SetOf<T> S}, just do:

\begin{verbatim}
  SetIterator<T> I(S);
  while ( !I.done() ) {
    /* Do something with I.value() */
    I.next();
  }
\end{verbatim}

\item
You may assign one \code{SetIterator} to another, so the following works:

\begin{verbatim}
  SetIterator<T> I(S);
  while( !I.done() ) {
    SetIterator<T> J = I;
    while( J.next() )
      if ( I.value() == J.value() ) error("Set contains duplicates!");
    I.next();
  }
  int count = 0;
  I.reset();
  while( !I.done() ) {
    SetIterator<T> J(I);
    do { if ( I.value() == J.value() ) ++count; } while( J.next() );
    I.next();
  }
  if ( count != S.cardinality() ) error("I give up");
\end{verbatim}

\end{itemize}

Since \code{I} was reset, the two excursions of \code{I} through \code{S}
are guaranteed to produce the same sequence of elements. In any case, \code{J}
is guaranteed to look at the rest of what \code{I} is.
You may alter \code{S} during the iteration, but \code{I} uses the old copy
of \code{S} (see \code{shrinkToIntersectionWith}). You may alter the object
returned by \code{I.value()}, but this will not effect \code{S} or \code{I}.


%% . . . . . . . . . . . . . . . . . . . . . . . . . . . . . . . . . .
\subsubsection{User class \code{VectorOf<T>}}

\subsubheading{What and Where}

\noindent The class\\
\code{template <class T> class VectorOf : public ObjectOf< VectorRep<T> >}\\
is defined in \file{magnus/back\_end/general/include/Vector.h}.

\noindent A \code{VectorOf<T>} is good for ordered sequences of \code{T}
which change size infrequently, and for which O(1) access is desired.



\subsubheading{Special Notes}

\noindent Class \code{T} must have

\begin{enumerate}
\item
A default constructor
\item
A copy constructor
\item
A destructor
\item
An assignment operator
\item
An \code{==} operator
\end{enumerate}

\noindent A Copy constructor, assignment operator,
and destructor are supplied by the compiler.



%%....................................................................
\subsection{Group Elements}



%% . . . . . . . . . . . . . . . . . . . . . . . . . . . . . . . . . .
\subsubsection{Introduction}

Group elements are represented by objects in the \code{Elt} hierarchy.
At present there is a pseudo-abstract user class \code{Elt}, and a
concrete user subclass \code{Word}, for representing abstract words in
formal generators.


%% . . . . . . . . . . . . . . . . . . . . . . . . . . . . . . . . . .
\subsubsection{User class \code{Elt}}

\subsubheading{What and Where}

\noindent The class\\
\code{class Elt : public ObjectOf< EltRep >}\\
is defined in \file{magnus/back\_end/Elt/include/Elt.h}.\\
It is a pseudo-abstract class which can represent any group element.


\subsubheading{Special Notes}

\noindent A copy constructor, destructor, and assignment operator are
supplied by the compiler.

\noindent Since \code{Elt} is pseudo-abstract, there is no default
constructor.


%% . . . . . . . . . . . . . . . . . . . . . . . . . . . . . . . . . .
\subsubsection{Utility class \code{Generator}}

\subsubheading{What and Where}

\noindent The class \code{class Generator}\\
is defined in \file{magnus/back\_end/Elt/include/Generator.h}.\\

Class \code{Word} uses \code{Generator} as an interface class.
A \code{Generator} is an object which has a non-zero integral ordinal value,
and a formal inverse.


% \subsubheading{Special Notes}


%% . . . . . . . . . . . . . . . . . . . . . . . . . . . . . . . . . .
\subsubsection{User class \code{Word}}

\subsubheading{What and Where}

\noindent The class \code{class Word : public Elt}\\
is defined in \file{magnus/back\_end/Elt/include/Word.h}.\\

A \code{Word} may be used outside the context of any particular group,
so the standard operations on words are provided.  However, several
subclasses of \code{Group} use \code{Word} to represent their elements, so
they also have methods for, e.g., multiplying \code{Word}s, which may
also, e.g., freely reduce.


% \subsubheading{Special Notes}


%%....................................................................
\subsection{Groups}


%% . . . . . . . . . . . . . . . . . . . . . . . . . . . . . . . . . .
\subsubsection{User class \code{FGGroup}}

\subsubheading{What and Where}

\noindent The class\\
\code{class FGGroup : public Group}\\
is defined in \file{magnus/back\_end/Group/include/FGGroup.h}.\\
It is a pseudo-abstract class which can represent any
finitely generated group.


\subsubheading{Special Notes}

\noindent Revision 1.2 is still in a state of flux.

\noindent A copy constructor, destructor, and assignment operator are
supplied by the compiler.

\noindent Since \code{FGGroup} is pseudo-abstract, there is
no default constructor.


%% . . . . . . . . . . . . . . . . . . . . . . . . . . . . . . . . . .
\subsubsection{User class \code{FPGroup}}

\subsubheading{What and Where}

\noindent The class\\
\code{class FPGroup : public FGGroup}\\
is defined in \file{magnus/back\_end/Group/include/FPGroup.h}.\\
It is a concrete class which can represent any
finitely presented group.


\subsubheading{Special Notes}

\noindent Revision 1.2 is still in a state of flux.

\noindent A copy constructor, destructor, and assignment operator are
supplied by the compiler.



%% . . . . . . . . . . . . . . . . . . . . . . . . . . . . . . . . . .
\subsubsection{User class \code{FreeGroup}}

\subsubheading{What and Where}

\noindent The class\\
\code{class FreeGroup : public FGGroup}\\
is defined in \file{magnus/back\_end/Group/include/FreeGroup.h}.\\
It is a concrete class which can represent any free group.


\subsubheading{Special Notes}

\noindent Revision 1.2 is still in a state of flux.

\noindent A copy constructor, destructor, and assignment operator are
supplied by the compiler.



%% . . . . . . . . . . . . . . . . . . . . . . . . . . . . . . . . . .
\subsubsection{User class \code{Group}}

\subsubheading{What and Where}

\noindent The class\\
\code{class Group : public ObjectOf< GroupRep >}\\
is defined in \file{magnus/back\_end/Group/include/Group.h}.\\
It is a pseudo-abstract class which can represent any group.


\subsubheading{Special Notes}

\noindent Revision 1.2 is still in a state of flux.

\noindent A copy constructor, destructor, and assignment operator are
supplied by the compiler.

\noindent Since \code{Group} is pseudo-abstract, there is
no default constructor.



%% . . . . . . . . . . . . . . . . . . . . . . . . . . . . . . . . . .
\subsubsection{Utility class \code{PresentationParser}}\label{presentationparser}

\subsubheading{What and Where}

\noindent The class\\
\code{class PresentationParser : public WordParser}\\
is defined in \file{magnus/back\_end/Group/include/PresentationParser.h}.\\
It provides facilities for reading a presentation from a stream.


\subsubheading{Special Notes}

\noindent Here is the presentation grammar:

\begin{verbatim}
Literals are delimited by ''.
The symbols ., *, +, -, |, ?, [, ], (, and ) are part of regular expressions
unless quoted.

<whitespace>     ::= ' ' | '\t' | '\n' | '#'.*'\n'
<index>          ::= 0 | [1-9]+[0-9]*
<integer>        ::= <index> | -<index>
<generator>      ::= [a-zA-Z]+<index>?
<atom>           ::= '1' | <generator> | '('<word>')' |
                     '['<word>','<word>(','<word>)*']'
<term>           ::= <atom>('^'<atom>)* | <term>'^'<integer>
<word>           ::= <term>('*'<term>)* | <term>(<whitespace><term>)*
<relator>        ::= <word>('='<word>)?
<generator list> ::= (<generator>(','<generator>)*)?
<word list>      ::= <word>(','<word list>)?
<relator list>   ::= <word>('='<word>)* |
                     (<word>('='<word>)*(','<word>('='<word>)*)?)?
<FPGroup>        ::= '<'<generator list>('|' | ';' | ':')<relator list>'>' |
                     '<'<generator list>'>'

- whitespace may occur anywhere except within lexemes.

Semantic conventions:

- x^y is y^-1 x y.
- ^ is left-associative, so a^t^2 is (a^t)^2.
- [x,y] is x^-1 y^-1 x y.
- [x1, x2, ..., xn] is left-normed: [[x1,x2], x3, ..., xn].
- Given the print name of a generator, the print name of its inverse
  has each character changed to the opposite case.
  Thus aB2^-1 is the same as Ab2.

\end{verbatim}



%% . . . . . . . . . . . . . . . . . . . . . . . . . . . . . . . . . .
\subsubsection{Utility class \code{TietzeTrekker}}

\subsubheading{What and Where}

\noindent The class \code{class TietzeTrekker}\\
is defined in \file{magnus/back\_end/Group/include/TietzeTrekker.h}.\\

A TietzeTrekker is a wrapper for a BlackBox which invokes
Frank Rimlinger's TietzeTrek with a group presentation.


\subsubheading{Special Notes}

\noindent Revision 1.2 is still in a state of flux.

\noindent Usage:

Instantiate a TietzeTrekker within, e.g., a FPGroup
method by

\begin{verbatim}
   TietzeTrekker( look() ) TT;
\end{verbatim}

Make sure that \code{TT.sanityCheck()} returns \code{TRUE}.
There is now an asynchronous process associcated with \code{TT} which is
continuously generating presentations, along with statements of
certain facts (factoids) which can readily be deduced from certain
presentations.
Get the `next' presentation from \code{getPresentation()}.
To be `notified' of any factoids, first call \code{getFactoid} (see comment
thereon), then call \code{knownTo*}. If you don't like the answer, call
\code{getFactoid} again, etc.


\subsubheading{Public Members}

\defmember{Constructor}{TietzeTrekker (const FPGroupRep\&)}{
Fire up \code{TietzeTrek} with the given presentation.
Passing a \code{FPGroupRep} instead of a \code{FPGroup}
is a (temporary?) hack.
}

\defmethod{Bool sanityCheck( )}{
Says whether this \code{TietzeTrekker}
thinks its initialization was successful.
}

\defmethod{Bool getFactoid(Bool \code{queuePresentations}, long \code{giveUpLimit} = 100)}{
The first argument says whether to save any presentations
which are found before the next factoid. Saved ones are later returned
by \code{getPresentation()}.
The second argument specifies the number of lines to read from the
\code{BlackBox} in search of a factoid before giving up and returning.
The return value says whether a new factoid was discovered.
}

\lskip
The following indicate what facts have been discovered so far by 0
or more calls to \code{getFactoid} (\code{FALSE} means ``don't know (yet)''):
\lskip

\defmethod{Bool knownToBeTrivial( )}{
}

\defmethod{Bool knownToBeCyclic( )}{
}

\defmethod{Bool knownToHaveFreeFactor( )}{
}

\defmethod{Bool knownToBeFinite( )}{
}

\defmethod{Bool knownToBeFree( )}{
}

\defmethod{int getOrder( )}{
Returns the order of the group in `standard' format: -1 == \code{DONTKNOW},
0 == \code{INFINITE}, otherwise actual order. If the answer is -1, you can
call \code{getFactoid} (again) for another throw of the dice.
}

\defmethod{int getRank( )}{
Returns the rank of the group AFTER \code{knownToBeFree()} has returned
\code{TRUE}.
It is a fatal error if this is called when \code{!knownToBeFree()}.
}

\defmethod{FPGroup getPresentation( )}{
Returns another presentation of the finitely presented group
(representation) used to instantiate this \code{TietzeTrekker}, in the form
of a \code{FPGroup}.
}


%% . . . . . . . . . . . . . . . . . . . . . . . . . . . . . . . . . .
\subsubsection{Utility class \code{WordParser}}

\subsubheading{What and Where}

\noindent The class \code{class WordParser}\\
is defined in \file{magnus/back\_end/Group/include/WordParser.h}.\\
It provides facilities for reading words from a stream.


\subsubheading{Special Notes}

\noindent See \S\ref{presentationparser} for the word grammar.
