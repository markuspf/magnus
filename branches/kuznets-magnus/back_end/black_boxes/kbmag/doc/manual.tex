%Latex file manual.tex 17/1/95.
%The style and layout was copied from the GAP manual.
\documentstyle{report}

%%%%%%%%%%%%%%%%%%%%%%%%%%%%%%%%%%%%%%%%%%%%%%%%%%%%%%%%%%%%%%%%%%%%%%%%%%%%%
%%
%%  The following commands intructs {\LaTeX} to stuff more on each  page  and
%%  to move each page towards to outer border.
%%
\topmargin 0 pt
\textheight 47\baselineskip
\advance\textheight by \topskip
\oddsidemargin  0.5 in
\evensidemargin  .25in
\textwidth 5.5in


%%%%%%%%%%%%%%%%%%%%%%%%%%%%%%%%%%%%%%%%%%%%%%%%%%%%%%%%%%%%%%%%%%%%%%%%%%%%%
%%
%%  The following commands instruct  {\LaTeX}  to  separate the paragraphs in
%%  this manual with a small space and to leave them unindented.
%%
\parskip 1.0ex plus 0.5ex minus 0.5ex
\parindent 0pt


%%%%%%%%%%%%%%%%%%%%%%%%%%%%%%%%%%%%%%%%%%%%%%%%%%%%%%%%%%%%%%%%%%%%%%%%%%%%%
%%
%%  'text'
%%
%%  'text' prints the text in  monospaced  typewriter  font.
%%  The text may contain all the usual characters  and |<name>| placeholders.
%%  |\'| can be used to enter a single  quote  character  into  the  text.
%%
\catcode`\'=13 \gdef'#1'{{\tt #1}}
\gdef\'{\char`'}


%%%%%%%%%%%%%%%%%%%%%%%%%%%%%%%%%%%%%%%%%%%%%%%%%%%%%%%%%%%%%%%%%%%%%%%%%%%%%
%%
%%  <text>
%%
%%  <text> prints  the text in  an italics font.
%%  The  text should not contain any special characters.  |\<| can be used to
%%  enter a less than character into the text.
%%
\catcode`\<=13 \gdef<#1>{{\it #1\/}}
\gdef\<{\char`<}


%%%%%%%%%%%%%%%%%%%%%%%%%%%%%%%%%%%%%%%%%%%%%%%%%%%%%%%%%%%%%%%%%%%%%%%%%%%%%
%%
%%  *text*
%%
%%  *text*  prints the text  in boldface font.
%%  The text may contain all  the usual characters.
%%  |\*| can be used to enter a star into the text.
%%
\catcode`\*=13 \gdef*#1*{{\bf #1}}
\gdef\*{\char`*}
\gdef\^{\char`^}


%%%%%%%%%%%%%%%%%%%%%%%%%%%%%%%%%%%%%%%%%%%%%%%%%%%%%%%%%%%%%%%%%%%%%%%%%%%%%
%%
%%  |text|
%%
%%  |text| prints the text between the two  pipe  symbols in typewriter style
%%  obeying the   linebreaks and spaces  in  the   manual.
%%  It should be used to  enter lengthy examples  into the text.
%%  If the hash character '\#' appears in the example the text between it
%%  and  the end of the line  is set in  ordinary mode,
%%  i.e., in  roman   font with  all the  possibilities ordinary available.
%%  |\|'\|' can be used to  enter  a  pipe symbol into the text.
%%
\catcode`\@=11

{\catcode`\ =\active\gdef\xvobeyspaces{\catcode`\ \active\let \xobeysp}}
\def\xobeysp{\leavevmode{} }

\catcode`\|=13
\gdef|{\leavevmode{}\hbox{}\begingroup
\def|{\endgroup}%
\catcode`\\=12\catcode`\{=12\catcode`\}=12\catcode`\$=12\catcode`\&=13
\catcode`\#=13\catcode`\^=12\catcode`\_=12\catcode`\ =12\catcode`\%=12
\catcode`\~=12\catcode`\'=12\catcode`\<=12\catcode`\"=12\catcode`\|=13
\catcode`\*=12\catcode`\:=12
\leftskip\@totalleftmargin\rightskip\z@
\parindent\z@\parfillskip\@flushglue\parskip\z@
\@tempswafalse\def\par{\if@tempswa\hbox{}\fi\@tempswatrue\@@par}%
\tt\obeylines\frenchspacing\xvobeyspaces\samepage}

\catcode`\@=12

{\catcode`\#=13
\gdef#{\begingroup
\catcode`\\=0 \catcode`\{=1 \catcode`\}=2 \catcode`\$=3 \catcode`\&=4
\catcode`\#=6 \catcode`\^=7 \catcode`\_=8 \catcode`\ =10\catcode`\%=14
\catcode`\~=13\catcode`\'=13\catcode`\<=13\catcode`\"=13\catcode`\|=13
\catcode`\*=13\catcode`\:=13
\catcode`\^^M=12 \Comment}}
{\catcode`\^^M=12
\gdef\Comment#1^^M{\rm \# #1 \endgroup \Newline}}
{\obeylines
\gdef\Newline{
}}

{\catcode`\&=13
\gdef&{\#}}

\gdef\|{\char`|}


%%%%%%%%%%%%%%%%%%%%%%%%%%%%%%%%%%%%%%%%%%%%%%%%%%%%%%%%%%%%%%%%%%%%%%%%%%%%%
%%
%%  <item>: <text>
%%
%%  This formats the  paragraph  <text>, i.e.,  everything between  the colon
%%  '\:' and the next  empty  line, indented 1 cm to the right in the manual.
%%  This
%%  convention should be  used to format  a list or an enumeration.    <item>
%%  should be  a single  word  or a short phrase.  It  may contain  all usual
%%  characters and the usual formatting stuff.  <text> is a  normal paragraph
%%  and may contain everything.   \:  can be used  to enter a colon character
%%  into the text.  As  an example consider the  following description.  This
%%  will print quite similar in the printed manual.
%%
%%      A group is represented by a record that must have the components
%%
%%      'generators': \\
%%              a list of group elements that  generate  the  group  that  is
%%              given by the group record.
%%
%%      'identity': \\
%%              the identity element of the group that is given by the  group
%%              record.
%%
\catcode`\:=13
\gdef:{\hangafter=1\hangindent=1cm\hspace{1cm}{}}
\gdef\:{\char`:}

%%%%%%%%%%%%%%%%%%%%%%%%%%%%%%%%%%%%%%%%%%%%%%%%%%%%%%%%%%%%%%%%%%%%%%%%%%%%%
%%
%%  "reference"
%%
%%  "reference" prints the  number of the  chapter or section  in the printed
%%  manual and is  displayed unchanged  in the  on-screen  documentation.  It
%%  should be used when referring to other  chapters or  sections.   The text
%%  should  not contain any special characters.  \"  can be  used  to enter a
%%  double quote into the text.
%%
\catcode`\"=13 \gdef"#1"{\ref{#1}}

\gdef\"{\char`"}

%%%%%%%%%%%%%%%%%%%%%%%%%%%%%%%%%%%%%%%%%%%%%%%%%%%%%%%%%%%%%%%%%%%%%%%%%%%%%
%%
%%  \GAP
%%
%%  \GAP can be used to enter the *sans serif* GAP logo  into  the  text.  If
%%  this is followed by spaces it should be enclosed in curly  braces  as  in
%%  |{\GAP}| is wonderfull.
%%
\newcommand{\GAP}{{\sf GAP}}
\newcommand{\KBMAG}{{\sf KBMAG}}
\newcommand{\rkbp}{{\sf rkbp}}
\newcommand{\Automata}{{\sf Automata}}
\newcommand{\Automate}{{\sf Automate}}
\newcommand{\Grail}{{\sf Grail}}

%%%%%%%%%%%%%%%%%%%%%%%%%%%%%%%%%%%%%%%%%%%%%%%%%%%%%%%%%%%%%%%%%%%%%%%%%%%%%
%%
%%  \Chapter{<name>}
%%  \Section{<name>}
%%
%%  |\Chapter| and |\Section| begin a  new  chapter  or  section.  They  work
%%  basically like the ordinary |\chapter| and |\section| macros except  that
%%  they also create a label for <name>
%%
\catcode`\@=11 \catcode`\%=12 \catcode`\~=14

\newcommand{\Chapter}[1]{{\chapter{#1}~
\label{#1}}}

\newcommand{\Section}[1]{{\pagebreak[3]\section{#1}~
\label{#1}}}

\catcode`\@=12 \catcode`\%=14 \catcode`\~=13

%%%%%%%%%%%%%%%%%%%%%%%%%%%%%%%%%%%%%%%%%%%%%%%%%%%%%%%%%%%%%%%%%%%%%%%%%%%%%
%%
%%  make the title page
%%
\begin{document}

\title{KBMAG \\
Knuth--Bendix in Monoids, and Automatic Groups \\
Version 1.3 (pre-release version)}
\author{
Derek Holt \\
Mathematics Institute, University of Warwick, Coventry CV4 7AL, UK }
\date{20 May 1995}

\maketitle

%%%%%%%%%%%%%%%%%%%%%%%%%%%%%%%%%%%%%%%%%%%%%%%%%%%%%%%%%%%%%%%%%%%%%%%%%%%%%
%%
%%  include the preface
%%
%\include{preface}


%%%%%%%%%%%%%%%%%%%%%%%%%%%%%%%%%%%%%%%%%%%%%%%%%%%%%%%%%%%%%%%%%%%%%%%%%%%%%
%%
%%  make the table of contents
%%
\tableofcontents


%%%%%%%%%%%%%%%%%%%%%%%%%%%%%%%%%%%%%%%%%%%%%%%%%%%%%%%%%%%%%%%%%%%%%%%%%%%%%
%%
%%  and now the chapters
%%

%%File intro.tex
\Chapter{Introduction}
\Section{What is KBMAG?}
{\KBMAG} (pronounced ``Kay-bee-mag\'\')
stands for *Knuth--Bendix on Monoids, and Automatic Groups*.
The purpose of this manual is to provide instructions for its use as a
stand-alone package.  It is also usable from within the {\GAP} system (see
\cite{Sch92}), and instructions for its use in that setting appear as a
chapter of the {\GAP} manual in the directory 'gapdoc'.
Some information on the interface with {\GAP} is also described in Chapter
"The Interface with GAP" of this manual.

The applications of {\KBMAG} can be divided into three inter-relating
categories. These are covered in detail in the Chapters
"The Knuth--Bendix Program for Monoids",
"The Automatic Groups Package" and
"Programs for Manipulating Finite State Automata",
but we will summarize them here.

Firstly, the program 'kbprog' can be used by itself to carry out the
Knuth--Bendix completion procedure on monoids defined by a finite presentation
and, in particular, on finitely presented groups. The user has a choice
between the short-lex ordering and the recursive-path ordering on strings.
Weighted short-lex and wreath-product orderings, are also possible.
The latter are defined on pages 46 -- 50 of \cite{Sims94}.
(It would be easy to make other orderings available if there were ever any
demand.) The implementation is designed more with speed of execution in mind
than with minimizing space requirements; thus, a finite state automaton is
always used to carry out word reduction, which can be space-consuming,
particularly when the number of generators is large. For a more flexible
Knuth--Bendix package, with more general orderings available, and a
choice of word-reduction procedures, the user could try the Rutgers
Knuth--Bendix package {\rkbp} written by Charles Sims. 
After running the program, the current set of reduction rules can be used to
reduce words in the monoid generators. If the rewriting-system produced is
confluent, then words will be correctly reduced to their irreducible normal
form under the given ordering, and so the word problem can be solved efficiently
in the monoid.

Secondly, the package can be used to compute the finite state automata
that constitute the automatic structure of a short-lex automatic group.
This supercedes the existing Warwick {\Automata} package for this purpose;
the current program is generally faster than {\Automata}, and successful with
more examples. For general information about automatic groups, see
\cite{ECHLPT92}, and for more detailed information about the algorithms
used in {\Automata}, see \cite{EHR91} or \cite{Holt94}. There are 
no fundamentally new algorithms employed in {\KBMAG}, but several
improvements have been made to the various components. The most
noticeable change is that a single multiplier automaton is now computed,
with the states labeled to indicate for which generators they are success
states, rather than a separate multiplier for each generator (although the
separate multipliers can still be computed if desired).

Computing an automatic structure is done in several steps, which are
carried out by a number of individual 'C'-programs. A Bourne Shell script,
called 'autgroup' has been provided to run these programs in the correct
sequence with, hopefully, a sensible default choice of options. As an
alternative to the use of this shell script, the individual programs can of
course be run separately. The first step is to run the program
'kbprog', but with the ``word-difference\'\'\ option, which is required
for automatic group calculations. The next program (which can itself be
divided up into different parts if required) computes the word-acceptor
and multiplier automata for the group. The final program (which can again
be split into parts), is the so-called axiom-checking process, which proves
that the automata that have been calculated are correct.
If the process runs to completion, then the automata can be used to reduce
words in the group generators to their irreducible normal forms under the
short-lex ordering, and so the word problem in the group can be solved
efficiently.

The third application of the package is for general manipulation of finite
state automata. Currently, this application is effectively restricted to
deterministic automata, since non-deterministic automata can only be read in and
printed out again. Eventually we plan at least to provide a program that
inputs a nondeterministic machine and outputs a deterministic one accepting
the same language.

There are programs for carrying out logical operations
on automata (in fact, some of the functions that they call are also used
in the automatic group calculations).  There are also programs to count
and to enumerate the language of a finite state automaton. These are
likely to be interesting to apply to the automata associated with
an automatic group, or to the reduction automaton output by 'kbprog' in
its stand-alone mode. There are a number of other, more dedicated,
packages that are available for manipulating finite state automata and
regular expressions. One such is {\Automate} (see \cite{Rie87} or
\cite{ChH91}), and another {\Grail} (see ??) developed at Ontario.

\Section{File formats}
The programs in {\KBMAG} generally do all of their serious input and output
from files, and only print diagnostics to 'stdout'. These files conform to
the format of the  {\GAP} system \cite{Sch92} and so are readable from
within {\GAP}. See the {\GAP} manual chapter in the 'gapdoc' directory, or
Chapter "The Interface with GAP" for details of how to use {\KBMAG} from within 
{\GAP}.

Two principal types of objects are handled, rewriting-systems and
finite state automata. Each file contains a single {\GAP} declaration, which
defines one object of one of these two types. This takes the general form

'<identifier> \:= rec(<list>);'

where <list> is a comma-separated list
of field-definitions that specify the values of the fields of the object
being defined. The two types of object are distinguished by the first
such definition, which must be either

|isRWS := true| \hspace{1cm}  or \hspace{1cm}  |isFSA := true|

for rewriting-systems and finite state automata, respectively.

To use the interface with {\GAP} in its current form, the name of a
rewriting-system (i.e. the value of <identifier>) has to be '\_RWS'.
This is because the relevant {\GAP} functions have '\_RWS' defined as an
external variable, and expect a declaration of this form.

Formal definitions of these formats have not yet been written down.
There is a text document giving an informal, and not completely up-to-data
description of the finite state automaton format in the {\KBMAG}
directory 'doc'. Examples of files containing rewriting-systems can be
found in the directories 'kb\_data' and 'ag\_data', and examples of
finite state automata in 'fsa\_data'. Rather than attempt a description
here, we suggest that the reader takes a look at some of these files.

For the Knuth--Bendix and automatic group applications, the user has to
supply a rewriting-system that defines a monoid or group in a file as input.
The programs produce new files containing rewriting-systems or automata.
In general, the user\'s file will contain a declaration of the above type,
and the computed files will contain a declaration of form

'<identifier>.<suffix> \:= rec(<list>);'

for the same <identifier>, so that if the user reads these files from
within {\GAP}, and the input file is read first, then, from {\GAP}\'s
viewpoint, the computed files will define new components of the original
record. (As mentioned above, for use with {\GAP}, the value of
<identifier> should always be '\_RWS'.)

The best method (and the one always employed by the author)
of creating an input file is to copy an existing one and edit it. Let us
briefly look at an example and discuss it.

|
#Free nilpotent group of rank 2 and class 2
_RWS := rec(
           isRWS := true,
  generatorOrder := [c,C,b,B,a,A],
        inverses := [C,c,B,b,A,a],
        ordering := "recursive",
       equations := [
         [b*a,a*b*c]
       ]
);
|

The first line is a comment, and is ignored by programs. (In general,
comments are preceded by the `\#\'\ symbol and last to the end of the line.)
To comply with the current {\GAP} interface, we name our rewriting-system
'\_RWS'.
As we saw above, the first field definition merely states that this is a
definition of a rewriting system.

The |ordering|\ field specifies which
ordering on strings in the input alphabet is to be used by the Knuth--Bendix
program. Although there is a default ordering, which is '\"shortlex\"',
this field is required by the {\GAP} interface, so it is recommended that
it should always be included.
It is also possible
to define the ordering by means of a command-line option to 'kbprog'
however. The convention for this, and various other fields, is that
command-line options override settings in the file in case of conflict.
Full details of these options are given in Chapters
"The Knuth--Bendix Program for Monoids" and "The Automatic Groups Package".

The remaining three fields provide the definition of the group or monoid.
First comes a list of generators. They must generate the structure as a monoid,
even if it is a group; this means inverses should be included in the generator
list. The field is named |generatorOrder|\ to emphasize the fact that the
order is relevant - it will affect the ordering of strings in the alphabet.
The names of the generators should be alphanumeric strings. In fact,
dots and underscores are also allowed, for compatability with {\GAP}.
Case is significant. It is recommended to use single letters, and use
case-change for inversion, as we have done in this example. Another
recommended policy is to use a common prefix followed by numbers; for
example, a file output by {\GAP} might have its generators named
'G.1, G.2, ..., G.n' for some $n \ge 0$. It is also permissible to name a
generator <name>'\^-1', where <name> is the name of another generator, and
the two are mutually inverse.

The |inverses|\ field supplies the list of two-sided inverses of the
generators, in the same order as the generators. This field must be present,
but, in general, not all generators need have inverses, so the list could
be empty, or contain gaps. For example, if only the first and third
generators have inverses, and these are named 'i1' and 'i2', then the list
would be written as '[i1,,i2]'.  However, if generator 'A' is listed as the
inverse of 'a', then 'a' must also be listed as the inverse of 'A'.
There is currently no mechanism for inputting one-sided inverses (although
that would be useful information for 'kbprog' under some circumstances, and so
it may be introduced in the future).
In the automatic groups applications, the structure must be a group, and all
generators must have inverses specified in the list.
(Currently, there is no way of specifying a default convention, such as
inversion equals case-change. We may introduce such a convention in
future, but this will depend on whether it can be made meaningful to
{\GAP}.)

Finally, there comes the |equations|\ field. This consists of a list of
lists. Each inner list constitutes a defining relation for the
monoid, and should have two entries, which are words in the generators,
and are the left and right hand sides of the relation.
The empty word is denoted (as in {\GAP}) by 'IdWord'.
The word may contain brackets to any level, and positive powers.
So, for example 'a\*(b\*(A\*c)\^4)\^3\*c\^12' is a valid word in the generators
in the example above.
Since the word is in the monoid generators, not all of which will
necessarily have inverses, negative powers are not permitted.
However, if a generator is named 'a\^-1', for example, then the <n>-th
power of it should be written as 'a\^-'<n> rather than as 'a\^-1\^'<n>. 

It is not necessary to include defining relations of type
'[a\*A,IdWord]' in the list of equations, where the generators 'a' and 'A'
have been already been specified as mutually inverse in the ``|inverses|\'\'\ 
field, and this has not been done in the example above. On the other hand
it does no harm if you do include them, and they will be included in
lists of equations output by 'kbprog'.

There are a number of other optional fields, which do not occur in this example,
and provide further detailed instructions for the Knuth--Bendix program.
See the description of this program in Section "kbprog" for details.

\Section{Exit Status of Programs and Meanings of Some Options}

The exit status of nearly all of the programs is 0 if successful and
1 if unsuccessful, and the program aborts with an error message
to 'stderr', without outputting to file.
One or two of the programs can also exit with status 2, which means that
something unusual but non-fatal has occurred. The two most important are
'kbprog' and 'gpaxioms'; see Sections "kbprog", "kbprog -wd", and
"gpaxioms".

Many of the options to the individual programs in {\KBMAG} have the same
meaning wherever they occur. To avoid repeating them over and over again, we
list some of them here.
\begin{description}
\item[|-v |]
The verbose option. A certain amount of extra information on progress of
calculation is printed out to stdout.
(By default, only the main results of calculations are printed out as
comments to stdout.)
\item[|-silent|]
There is no output at all to 'stdout'.
The only output to the terminal is error messages when the program
aborts for some reason.
\item[|-vv |]
The very-verbose option. A huge of diagnostic information is printed out,
much of which may seem incomprehensible.
\item[|-l |]
Large hash-tables. When constructing finite state automata, the states
are identified as sequences of integers, sometimes of varying length.
The sequences are stored in open hash-tables. Space is allocated as
required in blocks. The default size of the block is $2^{18}$ bytes;
with the '-l' option it becomes $2^{21}$ bytes. This makes things
run more efficiently when constructing very large automata.
There is also a '-h' option (huge hash-tables) which is of doubtful
desirability.
\item[|-ip d|]
Store finite state automata in dense format, which means that the
transition table is stored as an $ne \times ns$ array, where $ne$ and
$ns$ are the sizes of the alphabet  and state-set, respectively.
This is always the default, and is the fastest. It can be expensive on
space, however, particularly when the alphabet is large. If you run out
of space, or a program starts to swap heavily, then it may be worth trying
sparse storage.
\item[|-ip s|[<dr>]]
Store finite state automata in sparse format. This means that the
transitions from each state are stored as a sequence of edge-target
pairs. With large automata, with large alphabet (which means size more
than about 5 or 6), it normally requires significantly less space than
dense format. The '[<dr>]' option (dense rows) is a compromise.
Here <dr> should be a positive integer (something like 1000 might be a good
choice). The transitions from the first <dr> states are stored in dense
format, and the remainder in sparse format.
\item[|-op d|]
Automata are written to files in dense format. This is the default
for one-variable automata (such as the word acceptor in an automatic
group).
\item[|-op s|]
Automata are written to files in sparse format. This is the default
for two-variable automata (such as the multiplier in an automatic
group).
\end{description}

%%File kbm.tex
\Chapter{The Knuth--Bendix Program for Monoids}
\Section{kbprog}

'kbprog  [-r] [-ro] [-t <tidyint>] [-me <maxeqns>] [-ms <maxstates>]'\\
'| || || || || || || |[-mrl <maxreducelen>] [-mlr <maxlenleft> <maxlenright>]\
[-mo <maxoverlaplen>]'\\
'| || || || || || || |[-sort <maxoplen>] [-v] [-silent] \
[-rk <minlen> <mineqns>]'\\
'| || || || || || || |[-lex] [-rec] [-rtrec] [-cn <confnum>] <monoidname>'

The program 'kbprog' has zillions of options. Only those that are relevant
to its use as a stand-alone Knuth--Bendix program on monoids are listed here.
Those pertaining to its use as part of the the automatic groups package are
dealt with in Chapter "The Automatic Groups Package".

'kbprog' takes its input from the file <monoidname>, which should contain a
declaration of a record defining a rewriting-system, in the format described
in Chapter "Introduction". Output is to two files, <monoidname>'.kbprog' and
<monoidname>'.reduce'. The first contains an updated declaration of the
original rewriting-system, in which the |equations|\ field contains the
list of all reduction equations found so far. If the process has completed,
and the system is now confluent, then a new field |isConfluent| will
have appeared, and will be set equal to 'true'. In the equations,
the left hand side will always be greater than the right hand side in the
ordering on strings that is being used (see options below).
The second file contains a finite state automaton,
which can be used, together with the contents of the first file,
to reduce words in the monoid generators. This is done using the
program 'wordreduce' (see Section "wordreduce (Knuth--Bendix)").

If the system is confluent, then these reductions will be to the unique
minimal word under the ordering being used that is equal in the monoid to the
input word. We can therefore solve the word problem in the monoid. In this
case, the language of the automaton will be this set of minimal words.
If the monoid is finite, then its order can be determined by using the
program 'fsacount' (see Section "fsacount (Knuth--Bendix)").
In any case, the words accepted
can be enumerated up to a specified length with the program 'fsaenumerate'
(see Section "fsaenumerate (Knuth--Bendix)").

The Knuth--Bendix process will more often than not run forever by default,
and so some conditions have to be specified under which it will stop.
These take the form of limits that are placed on certain variables,
such as the number of reduction equations. 
These limits can be given values by the user, either by use of command-line
options, or with a field setting in the input file
(and the former takes priority in case of conflict).
Wherever possible, if the program halts because one of the limits is exceeded,
it will print a message informing the user what has happened,
and output its current set of equations and the reduction machine.

It is also possible to halt the program interactively at any stage, by
sending it a single interrupt signal (which usually means typing Control-C).
If you do this, then it will halt and output at the next convenient
opportunity, and so you may have to wait a while. If you send two interrupt
signals, then it will abort immediately without outputting.

*Options*\\
For most of the command-line options, the same effect can also be achieved by 
use of a corresponding field setting in the input file. In case of conflict,
the convention is that options set via the command-line override the setting in
the input file. Note, however,  that the two most complicated ordering options,
the weighted short-lex ordering and the wreath-product ordering, can only
be sensibly set within the input file, because they require the generators
to be given weights and levels, respectively, in these two cases.
See the options '-wtlex' and '-wreath' below, for further details.
\begin{description}
\item[|-r |]
This means resume after a previous run in which the output set of equations
was not confluent. Input will be taken from <monoidname>|.kbprog| instead of
from <monoidname>. The output will go to the same place, so the old
<monoidname>|.kbprog| will be overwritten.
It is useful if the program halted on a previous run due to some limit
being exceeded, and you wish to resume with a higher limit.
\item[|-ro |]
This is similar to |-r|, but in addition to taking the 
input from <monoidname>|.kbprog|, the original equations, in the file
<monoidname>, will also be read in and re-inserted at the end of the list.
This is sometimes necessary or advisable, if on the previous run not all
equations have been output, or some have been rejected because they were too
long. In that situation, there is a danger that the monoid defined by
the equations may have changed, and it can always be reset to the original
by re-inserting the original equations.
The output will go to the usual place, so the old
<monoidname>|.kbprog| will be overwritten.
\item[|-t| <tidyint>] | |\newline
After finding <tidyint> new reduction equations, the program interrupts
the main process of looking for overlaps, to tidy up the existing set of
equations. This means eliminating any redundant equations and performing
some reductions on their left and right hand sides to make the set as
compact as possible. (The point is that equations discovered later often
make older equations redundant or too long.) The default value of
<tidyint> is 100, and it can be altered with this option. Different values
work better on different examples. This parameter can also be set by
including a |tidyint| field in the input file.
\item[|-me| <maxeqns>] | |\newline
This puts a limit on the number of reduction equations.
The default is 32767.
If exceeded, the program will stop and output the current equations.
It can also be set as the field 'maxeqns' in the input file.
\item[|-ms| <maxstates>] | |\newline
This is less important, and not usually needed.
It sets a limit on the number of states of the finite state automaton
used for word reduction.
If exceeded, the program will stop and output the current equations.
By default, there is no limit, and the space allocated is increased
dynamically as required. Occasionally, the space required can increase too fast
for the program to cope; in this case, you can try setting a higher limit.
It can also be set as the field 'maxstates' in the input file.
The space needed for the reduction automaton can also be restricted by
using the |-rk| (Rabin-Karp reduction) option - see below.
\item[|-mrl| <maxreducelen>] | |\newline
Again, this is not needed very often. It is the maximum allowed length that
a word can reach during reduction. By default it is 32767.
If exceeded, the program is forced to abort without outputting.
It is only likely to be exceeded if you are using a recursive ordering on words.
It can also be set as the field 'maxreducelen' in the input file.
\item[|-mo| <maxoverlaplen>] | |\newline
If this is used, then only overlaps of total length at most <maxoverlap>
are processed.
Of course this may cause the overlap search to complete on a set
of equations that is not confluent. If this happens, you can always resume
with a higher (or no) limit.
This parameter can also be set as the field 'maxoverlaplen' in the input file.
\item[|-mlr| <maxlenleft> <maxlenright>] | |\newline
If this is used, then only equations in which the left and right hand sides
have lengths at most <maxlenleft> and <maxlenright>, respectively, are
kept. Of course this may cause the overlap search to complete on a set
of equations that is not confluent. If this happens, you can always resume
with higher limits. In some examples, particularly those involving
collapse (i.e. a large intermediate set of equations, which later simplifies
to a small set), it can result in a confluent set being found much more
quickly. It is most often useful when using a recursive ordering on words. 
Another danger with this option is that sometimes discarding equations can
result in information being lost, and the monoid defined by the equations
changes. If this may have happened, a warning message will be printed at
the end. In this case, the recommended policy is to re-run with the
|-ro| option. This option can also be set as a field in the input file.
The syntax for this is

'maxstoredlen \:= [<maxlenleft>,<maxlenright>]'
\item[|-sort| <maxoplen>] | |\newline
This causes equations to be output in order of increasing length of their
left hand sides, rather than the default, which is to output them in the
order in which they were found. <maxoplen> should be a non-negative integer.
If it is positive, then only equations with left hand sides having length
at most <maxoplen> are output. If it is zero, then all equations are output.
Of course, if <maxoplen> is positive, there is a danger that the monoid
defined by the output equations may be different from the original.
In this case, the safest thing is to edit the output file,
re-insert the original relations and re-run (possibly with higher length
limits). In later versions it will be possible to do this automatically.
This option can also be set as fields in the input file.
The syntax for this is

'sorteqns \:= true, maxoplen \:= <maxoplen>'
\item[|-rk| <minlen> <mineqns>] | |\newline
Use the Rabin-Karp algorithm for word-reduction on words having length at least
<minlen>, provided that there are at least <mineqns> equations.
This uses less space than the default reduction automaton, but it is
distinctly slower, so it should only be used when you are seriously short of
memory.
In fact, if the program halts and outputs for any reason, then
the full reduction automaton is output as normal, so it is only really
useful for examples in which collapse occurs - i.e. at some intermediate
stage of the calculation there is a very large set of equations, which later
reduces to a much smaller confluent set. However, this situation is not
uncommon when analysing pathological presentations of finite groups, and
this is one situation where the performance of the Knuth-Bendix algorithm can
be superior to that of Todd-Coxeter coset enumeration.
The best settings for <minlen> and <mineqns> vary from example to
example - generally speaking, the smaller <minlen> is, the slower things
will be, so set it as high as possible subject to not running out of memory.
<mineqns> should be set higher than you expect the final numebr of
equations to be.
This option can also be set as a field in the input file.
The syntax for this is

'RabinKarp \:= [<minlen>,<mineqns>]'
\item[|-v |]
The verbose option. Regular reports on the current number of equations, etc. are
output. This is to be recommended for interactive use.
This parameter can also be set by including a |verbose| field in the input
file, and setting it equal to 'true'.
\item[|-silent|]
There is no output at all to 'stdout'. In particular, the reason for
halting will not be printed.
This parameter can also be set by including a |silent| field in the input
file, and setting it equal to 'true'.
\item[|-lex|]
Use the short-lex ordering on strings. This is the default ordering.
Shorter words come before longer, and for words of equal length,
lexicographical ordering is used, using the given ordering of the generators.
It can also be set as a field in the input file. The syntax for this is

|ordering := "shortlex"|
\item[|-rec, -rtrec|]
Use a recursive ordering on strings. 
There are various ways to define this. Perhaps the quickest is as
follows. Let $u$ and $v$ be strings in the generators.
If one of $u$ and $v$, say $v$,  is empty, then $u \ge v$.
Otherwise, let $u=u^\prime a$ and $v=v^\prime b$,
where $a$ and $b$ are generators.
Then $u > v$ if and only if one of the following holds\:
\begin {description}
\item[(i)] $a = b$ and $u^\prime > v^\prime$;
\item[(ii)] $a > b$ and $u > v^\prime$;
\item[(ii)] $b > a$ and $u^\prime > v$.
\end {description}
This is the ordering used for the |-rec| option. The |-rtrec| option is
similar, but with $u=au^\prime$ and $v=bv^\prime$;
occasionally one or the other runs
significantly quicker, but usually they perform similarly.
It can also be set as a field in the input file. The syntax for this is

|ordering := "recursive"| \hspace{1cm} or  \hspace{1cm}
|ordering := "rt_recursive"|

\item[|-wtlex|]
Use a weighted short-lex ordering.
Although this option does exist as a command-line option, it will usually
be specified within the input file, because each generator needs to be
assigned a weight, which should be a non-negative integer. The \'\'ength\"
of words in the generators is then computed by adding up the weights of the
generators in the words. Otherwise, ordering is as for short-lex.
An example of assignments within the the input file is:

|ordering := "wtlex", weight := [2,1,6,3,0],|

which assigns weights 2,1,6,3 and 0 to the generators. The length of the
list of weights must be equal to the number of generators. The assignment
of the 'weight' field must come after the 'generatorOrder' field.
\item[|-wreath|]
Use a wreath-product ordering.
Although this option does exist as a command-line option, it will usually
be specified within the input file, because each generator needs to be
assigned a level, which should be a non-negative integer.
In this ordering, two strings involving generators of the same level are
ordered using short-lex, but all strings in generators of a higher level are
larger than those involving generators of a lower level. That is not a
complete definition; one can be found  on pages 46 -- 50 of \cite{Sims94}.
Note that the recursive ordering is the special case in which the level
of generator number $i$ is $i$.
An example of assignments within the the input file is:

|ordering := "wreathprod", level := [4,3,2,1],|

which assigns levels 4,3,2 and 1 to the generators. The length of the
list of levels must be equal to the number of generators. The assignment
of the 'level' field must come after the 'generatorOrder' field.
\item[|-cn| <confnum>] | |\newline
If <confnum> overlaps are processed and no new equations are discovered, then
the overlap searching process is interrupted, and a fast check for
confluence performed on the existing set of equations.
The default value is 500. Doing this too often wastes time, but doing it
at the right moment can also save a lot of time. Sometimes a particular
value works very well for a particular example, but it is difficult
to predict this in advance! If <confnum> is set to 0, then the fast
confluence check is performed only when the search for overlaps is
complete.
It can also be set as the field 'confnum' in the input file.
\end{description}

*Exit status*\\
The exit staus is 0 if 'kbprog' completes with a confluent set of equations,
2 if it halts and outputs a non-confluent set because some limit has
been exceeded, or it has received an interrupt signal, and 1 if it exits
without output, with an error message.

\Section{wordreduce (Knuth--Bendix)}
'wordreduce  [-kbprog/-diff1/-diff2/-diffc] [-mrl <maxreducelen>]'\\ 
'| || || || || || || || || || || |<monoidname> [<filename>]'

This program can be used either on the output of 'kbprog' or on the output
of the automata package. The '[-diff<x>]' options refer to the latter use,
which is described in Section
"wordreduce (automatic groups)".
The former use is the default if
the file <monoidname>'.kbprog' is present; to be certain, call the
'-kbprog' flag explicitly.

'wordreduce' reduces words using the output of a run of 'kbprog', which is
read from the files <monoidname>'.kbprog' and <monoidname>'.reduce'.
The reductions will always be
correct in the sense that the output word will represent the same group
element as the input word. If the system of equations in <monoidname>'.kbprog'
is confluent, then the reduction will be to the minimal word that represents
the group element under the ordering on strings of generators that was used
by 'kbprog'. It can therefore be used to solve the word problem in the
monoid. If the system is not confluent, then there will be some pairs of words 
which are equal in the monoid, but which reduce to distinct words, and
so this program cannot be used to solve the word problem.

If the optional argument <filename> is not present, then the program prompts
for the words to be input at the terminal. If <filename> is present, then
<filename> should contain a list of words to be reduced in the form of a
{\GAP} assignment to a list; for example\:

|wordlist := [a^20, c*b*a, (a*b)^5*c];|

The output will be a list of the reduced words in the file <filename>'.reduced'.
The option '-mrl\ <maxreducelen>' is the same as in 'kbprog'.

\Section{fsacount (Knuth--Bendix)}

'fsacount  [-ip d/s] [-silent] [-v] [<filename>]'

This is one of the finite state automata functions. See Chapter 
"Programs for Manipulating Finite State Automata" for the complete list.
The size of the accepted language is counted, and the answer (which may
of course be infinite) is output to 'stdout'. Input is from <filename> if
the optional argument is present, and otherwise from 'stdin', and it
should be a declaration of a finite state automaton record.
If 'kbprog' outputs a confluent set of equations for the monoid in the file
<monoidname>, then running 'fsacount <monoidname>.reduce'
will calculate the size of the monoid.
For description of options, see Section
"Exit Status of Programs and Meanings of Some Options".

\Section{fsaenumerate (Knuth--Bendix)}

'fsaenumerate  [-ip d/s] [-bfs/-dfs] <min> <max> <filename>' 

This is one of the finite state automata functions. See Chapter 
"Programs for Manipulating Finite State Automata" for the complete list.
Input is from <filename>, which should contain a declaration of a finite
state automaton record.
<min> and <max> should be non-negative integers with <min> $\le$ <max>.
The words in the accepted language having lengths at least <min> and at
most <max> are enumerated, and output as a list of words to the file
<filename>'.enumerate'.
If 'kbprog' outputs a confluent set of equations for the monoid in the file
<monoidname>, then running 'fsaenumerate <min> <max> <monoidname>.reduce'
will produce a list of elements in the monoid of which the reduced 
word representatives have lengths between <min> and <max>.
If the option '-dfs' (depth-first search - the default) is called,
then the words
in the list will be in lexicographical order, whereas with '-bfs'
(breadth-first-search), they will be in order of increasing length, and in
lexicographical order for each individual length (i.e. in short-lex order).
Depth-first-search is marginally quicker.
For description of other options, see Section
"Exit Status of Programs and Meanings of Some Options".
\Section{Examples (Knuth--Bendix)}

In this section, we mention some of the examples in the 'kb\_data' directory.
These can usefully be used as test examples, and some of them have been
included to demonstrate particular features.

The 'degen' examples  are all of the trivial group. Note, in particular,
'degen4a', 'degen4b' and 'degen4c'. These are the first three of an
infinite sequences of increasingly complicated presentations of the
trivial group, due to B.H. Neumann. 'kbprog' will run quite quickly on
'degen4b' (although no current Todd-Coxeter program will complete on this
presentation), but it does not appear to complete on 'degen4c'.

The example 'ab2' is the free abelian group on two generators.
It terminates with a confluent set for the given ordering of the
generators, |[a,A,b,B]|, but does not terminate with the ordering |[a,b,A,B]|.

Several of the examples are of finite groups. 
Others are monoid presentations, where generators are not supplied with
inverses. Try 'f25monoid', which is the presentation of the Fibonacci group
$F(2,5)$, but as a monoid. In fact, the structure is almost identical to
the group in this example. The group is cyclic of order 11.
The monoid has order 12, the extra element being the empty word.
The corresponding semigroup (without the empty word) is isomorphic to the
group.

In the examples 'nilp2', 'nonhopf', 'heinnilp' and 'verfiynilp', the
'recursive' ordering is essential. The last two of these are examples of
the use of Knuth--Bendix to prove that a presentation defines a nilpotent
group (first proposed by Sims). In 'verifynilp', things are made much
easier by using the 'maxstoredlen' parameter (or equivalently the '-mrl'
option). (Appropriate settings are already in the input file.)

The example 'f27monoid' is a monoid presentation
corresponding to the Fibonacci group $F(2,7)$, which has order 29.  As is
the case with $F(2,5)$, the monoid is the same structure with the empty
word thrown in, but this example is rather more difficult for 'kbprog'.
The best approach is to use a recursive ordering with a limit on the
lengths of equations stored ('-mrl 15 15' works well for 'f27monoid').
This will halt with the warning message that information may have been lost.
The original equations should then be adjoined to the output equations,
which is achieved by re-running 'kbprog' with the '-ro' option,
(and no limits on lengths).
It will then quickly complete with a confluent set. This is typical of a
number of difficult examples, where good results can be obtained by running
more than once.

The example 'cosets' is an easy example of use of 'kbprog' for enumeration
of the cosets of a nontrivial subgroup.
Note that the group generators have inverses, but the generator
'H' representing the subgroup does not, so 'H' cannot be
cancelled on the left. To make sure 'H' only appears on the left,
we have relations  like |[a*H,H]| which cause it to absorb all generators
on its left. The group is the symmetric group of
degree 4 and order 24, and the subgroup has order 4. The accepted language
of the reduction automaton has size 30; the 24 group elements, and the 6
cosets.

%%File autgp.tex
\Chapter{The Automatic Groups Package}

The main aim of this package is to calculate the finite state automata
associated with a short-lex automatic group. At the end of a
successful calculation, four automata will be stored. These are the first
and second word-difference machines, the word-acceptor, and the multiplier.
The descriptions in this  chapter of the manual will assume some familiarity
with these objects on the part of the reader.
See \cite{EHR91} or \cite{Holt94} for details. One difference in the
current version from the existing {\Automata} package
is that a single multiplier automaton is calculated,
rather than a separate one for each generator.
This single multiplier is known as the general multiplier. Some of its states
are labeled by one of the group generators. These states are the accepting
states for that generator. To obtain the individual multipliers in minimized
form, run the program 'gpmult' (see Section "gpmult"). They are not normally
significantly smaller than the general multiplier, and there is one of them
for each generator, so usually the general multiplier provides a much more
compact way of storing the same information than the individual multipliers.

The resulting automata
can be used for reducing words to their unique minimal representative in
the group under the short-lex ordering (and thereby enabling the
word-problem to be solved in the group), and also for counting and
enumerating the accepted language.
See Sections "wordreduce (automatic groups)", "fsacount (automatic groups)"
and "fsaenumerate (automatic groups)" for details.
(In fact the multiplier automaton is not needed for any of these processes,
but it forms part of the automatic structure of the group from a
theoretical viewpoint, so we regard it as part of the output of the
calculation.)

There are two possibilities for computing the automatic structure.
The simplest is to use the program 'autgroup', which is in fact a
Unix Bourne-shell script, which runs the three C-programs involved,
and attempts to make a sensible choice of options.
The other possibility is to run these three programs 'kbprog -wd',
'gpmakefsa' and 'gpaxioms' individually, and to select the options oneself.
'kbprog -wd' runs the Knuth--Bendix process on the defining relations of the
group, and calculates the resulting word-differences arising from the
equations generated. If the group is short-lex automatic, then the set
of word-differences is finite, and so the calculated set will eventually
be complete; however, 'kbprog -wd' itself will normally run forever, generating
infinitely many equations, and so the difficulty is to devise sensible
halting criteria. Clearly one wants to halt as soon as all of the
word-differneces have been found,if possible. Further details are given in
Section "kbprog -wd". 'gpmakefsa' uses the word-difference output by
'kbprog -wd' to compute the word-acceptor and multiplier automata.
It performs some checks on these which can reveal if the set of
word-differences used was not in fact complete, and find some of the
missing ones. It can then try again with the extended sets of
word-differences. Finally, the program 'gpaxioms' performs the
axiom-checking process, which proves the correctness of the automata
calculated. This process is expensive on resources in terms of both time
and space. Interestingly enough, we have never known it to complete and
return a negative answer; the reason for this is that the tests carried
out in 'gpmakefsa' tend to detect any errors in the automata.
So please inform me immediately if 'gpaxioms' ever reports that the
the structure calculated by 'kbprog\ -wd' and 'gpmakefsa' is incorrect!

For simple examples, these programs work quickly, and do not require
much space. For more difficult examples, they are often capable of
completing successfully, but they can sometimes be expensive in terms of
time and space requirements. If you are running out of space on your computer,
or it is starting to swap heavily, then there are some options available
in some cases (such as '-ip s' and '-f' for 'gpmakefsa' and 'gpaxioms')
which will cause things to use less space, but to take longer.
Another point to be borne in mind is that they produce temporary disk
files which the user does not normally see (because they are
automatically removed after use), but can occasionally be very large.
If you are in danger of exceeding your filestore allocation, or filling up
your disk partition, then you might try running the programs in the '/tmp'
directory. In any case, if a program halts unnaturally (perhaps because
you interrupt it), then you must remove these temporary files yourself.
If the file containing the original group presentation is  named
<groupname>, then all files created by the programs will have names of
form <groupname>'.'<suffix>.
\Section{autgroup}

'autgroup  [-l] [-v] [-vkb] [-silent] [-diff1] [-f] <groupname>'

Compute the finite state automata that constitute the automatic structure
of a short-lex automatic group. Input is taken from the file <groupname>,
which should contain a declaration of a record defining a rewriting-system, in
the format described in Chapter "Introduction". The rewriting-system must
define a group, so all monoid generators must have inverses listed explicitly
(there is no default convention for names of inverses). Output is to the
four files <groupname>'.diff1', <groupname>'.diff2', <groupname>'.wa' and
<groupname>'.gm', which contain, respectively, the first and second
word-difference automata, the word acceptor and the general multiplier.

*Options*\\
For a general description of the options '[-l]', '[-v]' and '[-silent]',
see Section
"Exit Status of Programs and Meanings of Some Options".
For greater flexibility in choice of options, run the programs 'kbprog\ -wd',
'gpmakefsa' and 'gpaxioms' individually, rather than using 'autgroup'. 

\begin{description}
\item[|-l |]
This causes 'gpmakefsa' and 'gpaxioms' to be run with the '-l' option, which
means large hash-tables. However, '-l' also causes 'kbprog\ -wd' to be run
with larger parameters, and you are advised to use it only after you
have tried first without it, since it will cause easy examples to take
much longer to run.
\item[|-vkb|]
This causes 'kbprog\ -wd' to be run with the verbose option, but the other
programs not (and is my personal preferred choice). Looking at the
verbose output of 'kbprog\ -wd' can help to give one an idea of whether the run
is likely to be successful, whether the '-l' option might be
more appropriate, etc. See Section "kbprog -wd" for further details on
'kbprog -wd'.
\item[|-diff1|]
This causes 'gpmakefsa' to be run with the '-diff1' option. See Section
"gpmakefsa" for further details.
\item[|-f |] This causes 'gpmakefsa' and 'gpaxioms' to be run with the
'-f' and '-ip s' options. See Sections "gpmakefsa" and "gpaxioms" for
further details. This is the option to choose if you need to save space,
and don\'t mind waiting a bit longer.
\end{description}
\Section{kbprog -wd}
'kbprog -wd [-t <tidyint>] [-me <maxeqns>] [-ms <maxstates>] \
[-mwd <maxwdiffs>]'\\
'| || || || || || || |[-hf <halting\_factor>] [-mt <min\_time>] \
[-rk <minlen> <mineqns>]'\\
'| || || || || || || |[-v] [-silent] [-cn <confnum>] <groupname>'

Some of these options are of course the same as for when 'kbprog' is being
used as a stand-alone Knuth--Bendix program -- see Section "kbprog".

'kbprog\ -wd' takes its input from the file <groupname>, which should contain a
declaration of a record defining a rewriting-system, in the format described
in Chapter "Introduction". Only the default ordering |"shortlex"| may be used,
since the other orderings make no sense in the automatic groups context.
Remember that the list of generators must include the inverses of all
generators.  Since the rewriting-system must define a group, rather than just
a monoid, the list of inverses, as specified in the 'inverses' field, must be
complete and involutory; i.e. all monoid generators must have inverses, and if
the inverse of <g1> is <g2> then the inverse of <g2> must be <g1>.

The Knuth--Bendix completion process is carried out on the group presentation.
However, unlike in the standalone usage of 'kbprog', the extended set of
equations and the reduction automaton are not output. Instead, the two
word-difference automata arising from the equations are output when the program
halts. These are printed to the two files <groupname>'.diff1' and
<groupname>'.diff2'. The difference between them is that the first automaton
contains only those word-differences and transitions that arise from the
equations themselves. For the second automaton, the set of word-differences
arising from the equations is closed under inversion, and all possible
transitions are included. So the second automaton has both more states and
generally more transitions per state than the first. For their uses, see
Section "gpmakefsa" below.

The main problem with this program is that, unless it completes with a
confluent set of equations, which happens relatively rarely for infinite groups,
the process will continue indefinitely, and so halting criteria have to be
chosen. If the group really is short-lex automatic with this choice of
ordered generating set, then the set of word-differences will be finite.
The general idea is to wait until it appears to have become constant, and
then stop.

There are two ways to do this. The first is to run 'kbprog\ -wd'
interactively with the '-v' option, when regular reports will be printed
on the current number of word-differences. (The word-differences are
calculated after each tidying operation on the equations; see the
'-t <tidyint>' option below.) The program can be interrupted at any time by the
user by sending it a single interrupt signal (which usually means typing
Control-C).  If you do this, then it will halt and output the current 
word-difference machines at the next convenient
opportunity (and so you may have to wait a while). If you send two interrupt
signals, then it will abort immediately without outputting.
The second way (and the one used by the shell-script 'autgroup') is to call
options which cause the program to halt automatically when the number of
word-differences has not increased for some time. This method is described
below under the option descriptions  '-hf <halting\_factor>' and
'-mt <min\_time>'.

This works well in simple examples, but there are various problems
associated with the more difficult examples. Firstly, it can happen that
most word-differences are found relatively quickly, but a few take much longer
to appear, and so the program is halted too early (either interactively or
automatically). This does not always matter, because the next program in
the sequence 'gpmakefsa' has the potential for finding missing word-differences
by performing checks on the automata that it calculates. However, this
will inevitably make 'gpmakefsa' run slower and use more space, and if too
many word-differences are missing, then it might not succeed at all. The thing
to do then, is to give up, and try running 'kbprog\ -wd' for longer, with more
stringent halting criteria (and possibly with a larger value of <tidyint>).
Another problem (the opposite problem) is that in some examples spurious
word-differences keep appearing and disappearing again, and so all required
word-differences may have been found long ago, but the reported number is
showing no sign of becoming constant. The only thing to do
when the number of word-differences keeps increasing and later decreasing
dramatically, is to try stopping it anyway (when the number is at
a low point), and then run 'gpmakefsa'.

Finally, if the number of word-differences is observed to increase quickly and
steadily, and gets up to about 2000, then it is likely that the group is not
short-lex automatic with this choice of ordered generators (and even if it is,
the size of the automata involved is likely to be too large for the
programs). Since short-lex automaticity is dependent on the choice of the
ordered generating set in some examples, it is worth trying different
choices of generators for the same group, and possibly different orderings
of the generators.

*Options*
\begin{description}
\item[|-t| <tidyint>] | |\newline
After finding <tidyint> new reduction equations, the program interrupts
the main process of looking for overlaps, to tidy up the existing set of
equations. This means eliminating any redundant equations and performing
some reductions on their left and right hand sides to make the set as
compact as possible. (The point is that equations discovered later often
make older equations redundant or too long.) The default value of
<tidyint> is 100, and it can be altered with this option. Different values
work better on different examples. This parameter can also be set by
including a |tidyint| field in the input file.
The word-differences arising from the equations are calculated
after each such tidying and the number reported if the '-v' option is called.
The best strategy in general is to try a small value of <tidyint> first and,
if that is not successful, try increasing it. Large values such as 1000 work
best in really difficult examples.
\item[|-me| <maxeqns>] | |\newline
This puts a limit on the number of reduction equations.
The default is 32767.
If exceeded, the program will stop and output the word-difference
automata arising from the current equations.
It can also be set as the field 'maxeqns' in the input file.
\item[|-ms| <maxstates>] | |\newline
This is less important, and not usually needed.
It sets a limit on the number of states of the finite state automaton
used for word reduction.
If exceeded, the program will stop and output the 
word-difference automata arising from current equations.
By default, there is no limit, and the space allocated is increased
dynamically as required. Occasionally, the space required can increase too fast
for the program to cope; in this case, you can try setting a higher limit.
It can also be set as the field 'maxstates' in the input file.
\item[|-mwd| <maxworddiffs>] | |\newline
Again this is not needed very often. It puts a bound on the number of
word-differences allowed. Normally, it is increased dynamically as required,
and so it does not need setting. Occasionally, it increases too fast for
the program to cope, and then it has to abort without output. If this
happens, try a larger setting.
\item[|-hf| <halting\_factor>]
\item[|-mt| <min\_time>]| |\newline
These are the experimental halting-options. If '-hf' is called, then
<halting\_factor> should be a positive integer, and represents a percentage.
After each tidying, it is checked whether both the number of
equations and the number of states have increased by more than
<halting\_factor> percent since the number of word-differences was last
less than what it is now.
If so, then the program halts and outputs the
word-difference automata arising from the current equations. A sensible value
seems to be 100, but occasionally a larger value is necessary. If the
'-mt' option is also called, then halting only occurs if at least <min\_time>
seconds of cpu-time have elapsed altogether.
This is sometimes necessary to prevent very early premature halting.
It is not very satisfactory, because of course the cpu-time
depends heavily on the particular computer being used, but no reasonable
alternative has been found yet.
There is no point in calling '-mt' without '-hf'.
\item[|-rk| <minlen> <mineqns>] | |\newline
Use the Rabin-Karp algorithm for word-reduction on words having length at least
<minlen>, provided that there are at least <mineqns> equations.
This uses less space than the default reduction automaton, but it is
distinctly slower, so it should only be used when you are seriously short of
memory.
The best settings for <minlen> and <mineqns> vary from example to
example - generally speaking, the smaller <minlen> is, the slower things
will be, so set it as high as possible subject to not running out of memory.
This option can also be set as a field in the input file.
The syntax for this is

'RabinKarp \:= [<minlen>,<mineqns>]'
\item[|-v |]
The verbose option. Regular reports on the current number of equations, etc. are
output. This is to be recommended for interactive use.
This parameter can also be set by including a |verbose| field in the input
file, and setting it equal to 'true'.
\item[|-silent|]
There is no output at all to 'stdout'.
This parameter can also be set by including a |silent| field in the input
file, and setting it equal to 'true'.
\item[|-cn| <confnum>] | |\newline
If <confnum> overlaps are processed and no new equations are discovered, then
the overlap searching process is interrupted, and a fast check for
confluence performed on the existing set of equations.
Setting <confnum> equal to 0 turns this feature off completely.
If, as is often the case, you are quite certain that the process is not going
to halt at all, then you should set <confnum> to 0, since the confluence
tests will merely waste time. (In fact, this should arguably be the
default setting for 'kbprog\ -wd'.)
It can also be set as the field 'confnum' in the input file.
\end{description}

*Exit status*\\
The exit staus is 0 if 'kbprog' completes with a confluent set of equations,
or if it halts because the condition in the '-hf <halting\_factor>' option
has been fulfilled,
2 if it halts and outputs because some other limit has
been exceeded, or it has received an interrupt signal, and 1 if it exits
without output, with an error message.
This information is used by the shell-script 'autgroup'.

\Section{gpmakefsa}

'gpmakefsa  [-diff1/-diff2] [-me <maxeqns>] [-mwd <maxwdiffs>] [-f] [-l]'\\
'| || || || || || || || || || |[-ip d/s[dr]] [-opwa d/s] [-opgm d/s]\
[-v] [-silent] <groupname>'

Construct the word-acceptor and general multiplier finite state automata
for the short-lex automatic group, of which the rewriting-system defining
the group is in the file <groupname>.
It assumes that 'kbprog\ -wd' has already been run on the group.
Input is from <groupname>'.diff2' and, if
the '-diff1' option is called, also from <groupname>'.diff1'. The
word-acceptor is output to <groupname>'.wa' and the general multiplier to
<groupname>'.gm'. Certain correctness tests are carried out on the
constructed automata. If they fail, then new word-differences will be
found and <groupname>'.diff2' (and possibly also <groupname>'.diff1')
will be updated, and the multiplier (and possibly the word-acceptor)
will then be re-calculated.

There are programs available which construct the two automata individually,
and also one which performs the correctness test, but it is unlikely that
you will want to use them. They are mentioned briefly in
Section "Other programs".

The algorithm is slightly different according to whether or not
<groupname>'.diff1' or <groupname>'.diff2' is used as input to construct
the word-acceptor. This is controlled by the '-diff1' and '-diff2' options.
The latter is the default; i.e. the default is to use <groupname>'.diff2'.
Theoretically, <groupname>'.diff1' should work, and indeed it is more
efficient when it does work, because the first word-difference automaton is
always smaller than the second. However, in many examples it turns out that
<groupname>'.diff1' as output by 'kbprog\ -wd' is incorrect. In this case, it
is nearly always more efficient overall to use <groupname>'.diff2', which has
a much higher chance of calculating the correct word-acceptor first time. We
have therefore chosen to make this the default. There are one or two
examples, however, in which the use of <groupname>'.diff2' causes severe
space problems in constructing the word-acceptor, whereas <groupname>'.diff1'
does not. If you observe this to be the case, then try again with the
'-diff1' option set.

In the default setting, the word-acceptor and general multiplier are both
constructed using <groupname>'.diff2'. A check is then carried out on the
mulitplier. In fact, it is checked that, for every word $u$ accepted by
the word-acceptor, and every generator $a$, there exists a word $v$
accepted by the word-acceptor such that $(u,v)$ is accepted by the multiplier,
where $ua$ and $v$ are equal in the group.
If this test fails, then the multiplier is incorrect, and a number of explicit
words $u$ and generators $a$ are calculated for which the test fails.
These give rise to equations $(u,v)$ (where $v$ is the word to which
$ua$ reduces), which produce new word-differences, which are in turn used
to re-calculate the second word-difference machine. The file
<groupname>'.diff2' is then updated, and the multiplier recalculated. It
may turn out in the course of this calculation that the word-acceptor also
needs recalculating, which is done if necessary. This process continues
until the multiplier passes the test.

Under the '-diff1' option, the file <groupname>'.diff1' is used to construct
the word-acceptor, but <groupname>'.diff2' is still used for the multiplier.
It may happen, during construction of the multiplier, that some equations
are found that should be accepted by the first word-difference machine, but
are not. These are used to correct the first word-difference machine, and
the file <groupname>'.diff1' is updated. Otherwise, things proceed as for the
'-diff2' option.

*Other options*
\begin{description}
\item[|-me| <maxeqns>] | |\newline
This specifies an upper limit on the number of equations that are processed
when correcting the first or second word-difference machines, as described
above. The default is 512. If it is too small, then the main testing and
correcting process on the multipliers may have to be repeated many more times
than necessary, but if it is too big, then the process of updating the
word-difference machines can be very slow.
\item[|-mwd| <maxwdiffs>] | |\newline
This puts an upper limit on the number of word-differences allowed after the
correction. It is rarely necessary to call this. If it is necessary, then the
program will halt with an informative error message.
\item[|-f |]
This is an option which saves space, at the expense of slightly increased
cpu-time. The multiplier automaton as initially constructed can be very
large. It is then minimized. Under this option, the unminimized multiplier
is not read in all at once, but kept in a file, and read in line-by-line
during minimization.
\end{description}
The remaining options are standard, and described in Section
"Exit Status of Programs and Meanings of Some Options".
Note, however, that the two '-op' options,
'-opwa' and '-opgm', refer to the output format of the word-acceptor and
general multiplier, resepctively. Since the former is one-variable and
the latter two-variable, the default is dense for the former and sparse
for the latter.
\Section{gpaxioms}

'gpaxioms [-ip d/s[dr]] [-op d/s] [-silent] [-v] [-l] [-f] <groupname>'

This program performs the axiom-checking process on the word-acceptor and
general multiplier of the short-lex automatic group,
of which the rewriting-system defining the group is in the file <groupname>.
It assumes that 'kbprog\ -wd' and 'gpmakefsa' have already been run on the
group.
It takes its input from <groupname>'.wa' and <groupname>'.gm'. 
There is no output to files (although plenty of intermediate files are
created and later removed during the course of the calculation).

If successful, there will be no output apart from routine reports, and the exit
status will be 0. If unsuccessful, a message will be printed to 'stdout'
reporting that a relation test failed for one of the group relations, and
the exit status will be 2. I have never known this to happen, at least on
the output of a successful run of 'gpmakefsa', so if it does, please inform
me immediately!

The option '-f' is as in 'gpmakefsa'. That is, unminimized automata are not
read into memory all at once during minimization.
The other options are standard, and described in Section 
"Exit Status of Programs and Meanings of Some Options".
\Section{gpmult}

'gpmult [-ip d/s] [-op d/s] [-silent] [-v] [-l] <groupname>'

This calculates the individual multiplier automata, for the monoid
generators of the short-lex automatic group, of which the rewriting-system
defining the group is in the file <groupname>.
It assumes that 'autgroup' (or the three programs 'kbprog\ -wd', 'gpmakefsa'
and 'gpaxioms') have already been run successfully on the group.
Input is from <groupname>.'gm', and output is to <groupname>'.m'<n>
(one file for each multiplier), for $n = 1, \ldots , ng+1$, where $ng$ is
the number of monoid generators of the group. The final multiplier is
the equality multiplier, which accepts $(u,v)$ if and only if $u = v$ and
$u$ is accepted by the word-acceptor.
All of the options are standard.
\Section{gpminkb}

'gpminkb [-op1 d/s] [-op2 d/s] [-silent] [-v] [-l] <groupname>'

Suppose that the file <groupname> contains a rewriting-system
defining a short-lex automatic group,
and that 'autgroup' (or the three programs 'kbprog\ -wd', 'gpmakefsa'
and 'gpaxioms') have already been run successfully on the group.
This program calculates one or two associated automata, which can be
interesting and useful. Input is from <groupname>'.wa' and from
<groupname>'.diff2'.

Firstly, a finite state automaton which accepts the
minimal reducible words in the monoid generators (i.e. the set of
left-hand-sides  of the (probably infinite) minimal confluent set of
rewrite-rules) for the group is output to <groupname>'.minred'.
Secondly, a two-variable finite state automaton accepting precisely
this minimal confluent set of rewrite-rules
for the group is output to <groupname>'.minkb'.
Finally, the correct first word-difference machine is output to
<groupname>'.diff1c'. It may be interesting to compare this with
<groupname>'.diff1', but remember that the states may be in a different
order. (The states of a finite state automaton can be re-ordered into a
standard order with the program 'fsabfs'. See Section "fsabfs".)
The file <groupname>'.diff1c' can be used efficiently as input for
'wordreduce'; see Section "wordreduce (automatic groups)" below.

The options are all standard, but note that '-op1' refers to the format
of the one-variable automaton in <groupname>'.minred' (and is dense by
default), whereas '-op2' refers to the two-variable automata in
<groupname>'.minkb' and <groupname>'.diff1c', and is sparse by default.

\Section{wordreduce (automatic groups)}
'wordreduce  [-kbprog/-diff1/-diff2/-diff1c] [-mrl <maxreducelen>]'\\
'| || || || || || || || || || || || |<groupname> [<filename>]'

This program can be used either on the output of 'kbprog' or on the output
of the automata package. The '-kbprog' option refers to the former use,
which is described in Section "wordreduce (Knuth--Bendix)".
The '-kbprog' usage is considerably quicker, but it is only guaranteed
accurate when a finite confluent set of equations has been calculated. 
The automatic groups programs are normally only used when such a set
cannot be found, in which case the usage here is the only
accurate one available.
The '-kbprog' usage is the default if the file <groupname>'.kbprog' is present.
To be certain of using the intended algorithm, call one of the flags
'-diff1', '-diff2', '-diff1c' explicitly.
If <groupname>'.kbprog' is not present, then input will be from
<groupname>'.diff2' by default, and otherwise from <groupname>'.diff1'
or <groupname>'.diff1c', according to which option is called.

It is assumed that the file <groupname> contains a rewriting-system
defining a short-lex automatic group,
and that 'autgroup' (or the three programs 'kbprog\ -wd', 'gpmakefsa'
and 'gpaxioms') have already been run successfully on the group.
The '-diff1' option should only be used for input if  'autgroup' or
'gpmakefsa' was run successfully with the '-diff1' option, in which
case it will be the most efficient. (Otherwise you might get wrong answers.)
The  '-diff1c' option can only be used if 'gpminkb' has already been run,
but in that case it will be the most efficient.

'wordreduce' reduces word in the monoid generators of the group to their
minimal irreducible equivalent under the short-lex ordering.
It can therefore be used to solve the word problem in the
monoid.

If the optional argument <filename> is not present, then the program prompts
for the words to be input at the terminal. If <filename> is present, then
<filename> should contain a list of words to be reduced in the form of a
{\GAP} assignment to a list; for example\:

|wordlist := [a^20, c*b*a, (a*b)^5*c];|

The output will be a list of the reduced words in the file <filename>'.reduced'.
The option '-mrl\ <maxreducelen>' puts a maximum length on the words to
be reduced. (In the situation here, they can never get longer during reduction.)
The default is 32767.

\Section{fsacount (automatic groups)}
'fsacount  [-ip d/s] [-silent] [-v] [<filename>]'

This is one of the finite state automata functions. See Chapter
"Programs for Manipulating Finite State Automata" for the complete list.
The size of the accepted language is counted, and the answer (which may
of course be infinite) is output to 'stdout'. Input is from <filename> if
the optional argument is present, and otherwise from 'stdin', and it
should be a declaration of a finite state automaton record.

If 'autgroup' (or the three programs 'kbprog\ -wd', 'gpmakefsa'
and 'gpaxioms') have already been run successfully on the group defined
in the file <groupname>, then running 'fsacount <groupname>.wa'
will calculate the size of the group.
For description of options, see Section
"Exit Status of Programs and Meanings of Some Options".

\Section{fsaenumerate (automatic groups)}

'fsaenumerate  [-ip d/s] [-bfs/-dfs] <min> <max> <filename>'

This is one of the finite state automata functions. See Chapter
"Programs for Manipulating Finite State Automata" for the complete list.
Input is from <filename>, which should contain a declaration of a finite
state automaton record.
<min> and <max> should be non-negative integers with <min> $\le$ <max>.
The words in the accepted language having lengths at least <min> and at
most <max> are enumerated, and output as a list of words to the file
<filename>'.enumerate'.

If 'autgroup' (or the three programs 'kbprog\ -wd', 'gpmakefsa'
and 'gpaxioms') have already been run successfully on the group defined
in the file <groupname>,
then running 'fsaenumerate <min> <max> <groupname>.wa'
will produce a list of elements in the group of which the reduced
word representatives have lengths between <min> and <max>.
If the option '-dfs' (depth-first search - the default) is called,
then the words
in the list will be in lexicographical order, whereas with '-bfs'
(breadth-first-search), they will be in order of increasing length, and in
lexicographical order for each individual length (i.e. in short-lex order).
Depth-first-search is marginally quicker.
For description of other options, see Section
"Exit Status of Programs and Meanings of Some Options".

\Section{Other programs}

The other programs in the package will be simply listed here. For
further details, look at their source files in the 'src' directory.
If you are sure that you do not want them, and wish to save disk-space,
you can safely delete their executables from the 'bin' directory.
\begin{description}
\item['gpwa'] Calculate word-acceptor.
\item['gpwab'] Calculate word-acceptor by another (less efficient) method.
\item['gpgenmult'] Calculate general multiplier.
\item['gpcheckmult'] Carry out the correctness test on the general multiplier.
\item['gpgenmult2'] Calculate the general multiplier for words of length 2.
\item['gpgenmult3'] Calculate the general multiplier for words of length 3.
\item['gpmult2'] Calculate a particular multiplier for a word of length 2.
\item['gpcomp'] Calculate the composite of two multiplier automata.
\end{description}

\Section{Examples (automatic broups)}

There is a collection of examples of short-lex automatic groups in the
directory 'ag\_data' of widely varying difficulty. The automatic structure
has been calculated and verified as correct in all cases, however.
These can usefully be used as test examples for gaining experience in the
use of the programs. We therefore give a brief summary below of how
they behave.

The 'autgroup' program will complete successfully in its default form on the
following examples, in roughly increasing order of difficulty.
The cpu-times quoted are on a SPARCstation 20.
All 'degen<n>' examples, 'c2', 'ab1', 'f2', 'zw2', 'c5', 'ab2', '333',
'334', 'trefoil', '235', 'gp55', '236', '238', '237', 'f2\_unusual',
'shawcroft', 'torus', '237b' (up to this example, they took less than
10 seconds cpu-time), 'listing', 'f26',
'johnson', '777', 'gunn', 'hamish12', 'G1L1', 's9', 'cox535' (less than
a minute up to this example), 'knot23', 'hamish13', 'hamish23',
'orbifold', 'johnsonc', 'f28', the last taking just under 4 minutes. 
Then, 'edjvet\_94\_1' takes about 18 minutes, 'picard' 35 minutes and 'andre'
90 minutes.  Remember, that the '-f' option can be used for the larger
examples if there are problems with memory or swapping, in which case
the times will be a little longer; for example, 'edjvet\_94\_1' took just under
20 minutes with '-f' on.

For the examples, 'cox33334c' and 'cox5335', the '-diff1' option needs
to be used. The former takes about 17 minutes and the latter 215 minutes.
However, 'cox5335' works much better with the '-l' option on as well, when
it only needs about 63 minutes. The examples 'surgery' and 'johnsonb'
only complete with the '-l' option called, and take 65 and 150 minutes,
respectively. 'f29' does not complete at all by just running 'autgroup -l',
since the limit imposed on the maximum number of equations in 'kbprog' is
still too small. This can be completed with a limit of about 500000 on
the 'maxeqns' parameter, but it produces a temporary file of size about 140
Megabytes during the course of the calculation.

%%File fsa.tex
\Chapter{Programs for Manipulating Finite State Automata}
In this chapter, we describe the utility functions that are provided for
manipulating finite state automata. Currently, only 'fsafilter' (which
merely reads in and prints out again) can handle nondeterministic
automata. In the future, we may provide facilities to determinize a
nondeterministic machine. For explanation of all of the standard options,
see Section "Exit Status of Programs and Meanings of Some Options".

\Section{fsafilter}

'fsafilter [-ip d/s[<dr>]] [-op d/s] [-silent] [-v] [<filename>]'

This merely reads in a finite state automaton and prints it out again.
It is used partly for testing, but also if one wants to change the format
of an automaton from dense to sparse or vice-versa.
If the optional <filename> argument is present, then input is from <filename>
and output to <filename>'.fsafilter'. Otherwise, input is from 'stdin' and
output to 'stdout'.

\Section{fsamin}

'fsamin [-ip d/s] [-op d/s] [-silent] [-v] [-l] [<filename>]'

A finite state automaton is read in, minimzed and then printed out again.
If the optional <filename> argument is present, then input is from <filename>
and output to <filename>'.fsamin'. Otherwise, input is from 'stdin' and
output to 'stdout'.

\Section{fsabfs}

'fsabfs [-ip d/s] [-op d/s] [-silent] [-v] [-l] [<filename>]'

A finite state automaton is read in, and then its states are permuted into
bfs-order (bfs = breadth-first-search), and it is printed out again.
This means that the states are numbered $1,2, \ldots, ns$, and if one
scans the transition-table, in order of increasing states, then the states
occur in increasing order. 'fsamin' and 'fsabfs' used together can be used
to check whether two deterministic automata with the same alphabet
have the same language.
First apply 'fsamin' and then 'fsabfs'.
If they have the same language, then the resulting automata should be identical.
If the optional <filename> argument is present, then input is from <filename>
and output to <filename>'.fsabfs'. Otherwise, input is from 'stdin' and
output to 'stdout'.

\Section{fsacount}

'fsacount  [-ip d/s] [-silent] [-v] [<filename>]'

A finite state automaton is read in,
the size of the accepted language is counted, and the answer (which may
of course be infinite) is output to 'stdout'.
Input is from <filename> if
the optional argument is present, and otherwise from 'stdin'.

\Section{fsaeliminate}

'fsaenumerate  [-ip d/s] [-bfs/-dfs] <min> <max> <filename>'

A finite state automaton is read in from the file <filename>.
<min> and <max> should be non-negative integers with <min> $\le$ <max>.
The words in the accepted language having lengths at least <min> and at
most <max> are enumerated, and output as a list of words to the file
<filename>'.enumerate'.

If the option '-dfs' is called (depth-first search -- the default),
then the words
in the list will be in lexicographical order, whereas with '[-bfs]'
(breadth-first-search), they will be in order of increasing length, and in
lexicographical order for each individual length (i.e. in shortlex order).
Depth-first-search is marginally quicker.

\Section{fsaand}

'fsaand [-ip d/s[<dr>]] [-op d/s] [-silent] [-v] [-l]\
 <filename1> <filename2> <outfilename>'

Two finite state automata, which must have the same alphabet,
are read in from the files <filename1> and <filename2>. An automaton that
accepts a word $w$ in the alphabet if and only if both of the input automata
accept $w$ is computed, minimized, and output to the file <outfilename>.

\Section{fsaor}

'fsaor [-ip d/s[<dr>]] [-op d/s] [-silent] [-v] [-l]\
 <filename1> <filename2> <outfilename>'

Two finite state automata, which must have the same alphabet, are read in from
the files <filename1> and <filename2>. An automaton that accepts a word $w$ in
the alphabet if and only if at least one of the input automata
accept $w$ is computed, minimized, and output to the file <outfilename>.

\Section{fsanot}

'fsanot [-ip d/s[<dr>]] [-op d/s] [-silent] [-v] [-l] <filename>'

A finite state automaton is read in from the file <filename>, and an automaton
with the same alphabet, which accepts a word $w$ if and only if the input
automaton does not accept $w$, is calculated and output to <filename>.not.

\Section{fsaexists}

'fsaexists [-ip d/s[<dr>]] [-op d/s] [-silent] [-v] [-l] <filename>'

A two-variable finite state automaton is read in from the file <filename>,
and a one-variable automaton, which accepts a word $v$ if and only if the input
automaton accepts $(v,w)$ for some word $w$,
is calculated and output to <filename>.exists.

%%File gap.tex
\Chapter{The Interface with GAP}

There are two ways in which the interface with {\GAP} can be used.
In the first, files are read into {\GAP} that have been created externally.
A {\GAP} conversion function 'ReadRWS' has been provided for this purpose.
This function assumes that the name of the rewriting system (i.e. the left
hand side of the declaration contained in the file) is |_RWS|, so this
name should always be used for files that are to be accessed by {\GAP}.

The second only works currently for groups (since monoids do not yet
exists as a {\GAP} type). Documentation for this method can also be found as
a chapter in the {\GAP} manual in the 'gapdoc' directory of {\KBMAG}.
A finitely presented group, $G$ say, is first
defined within {\GAP}, and then the {\GAP} function 'FpGroupToRWS' is called
on $G$, and a corresponding rewriting system is returned.
It can be called with an optional boolean second-variable, for example

|R := FpGroupToRWS(G,true);|

in which case inverses of generators are
printed using a case-change convention. Otherwise, inverses are printed in
the usual way by appending the suffix |^-1| to the generator name.

Since the internal storage of rewriting-systems in particular is
different from the structure defined in the external file (for example,
words are stored internally as lists of integers), the user should not
attempt to create rewriting systems directly internally to {\GAP},
but should either read them in from external files, or create them using the
function 'FpGroupToRWS'.

An implementation of finitely presented monoids within {\GAP} is currently
being written, and this too will eventually have an interface with
rewriting-systems.
 
The library files in the directory called 'gap' contain elementary functions
for manipulating finite state automata and rewriting-systems.
Anyone who wishes to use these seriously, should look in the files 'fsa.g' and
'rws.g'. A finite state automaton must either be typed in
interactively, or (better) put into a file first, and read in. Of course,
the automatic group programs calculate finite state automata, and the
functions are probably mainly useful for playing with these. It important to
know that, before using any of the other {\GAP} functions on an automaton <fsa>,
the function 'InitializeFSA(<fsa>)' must be called. Some of the functions,
like those that count or enumerate the language of a finite state automaton,
perform the same operations as the corresponding standalone 'C'-programs,
but they will usually be much slower than the standalones. Nevertheless,
it is often convenient to be able to do such things without having to go
through the process of writing to file, running external program, and then
reading the answer back in.

After installing the package, but before starting to use the {\GAP} functions,
go into the {\GAP} directory and type 'makeinit'. This will create the
required 'init.g' file for the {\GAP} library.
This file should be read in from {\GAP} before using the functions.
You may also need to edit the value of the variable declared as
|_KBExtDir| in the file |rws.g|. This should be set to the absolute path
name of the directory in which the {\KBMAG} executables live (including the
final '/'). 

{\bf UPDATE}\:\ As an alternative to this, it is now possible to install
{\KBMAG} as a share-library. Instuctions for this are in the 'README' file in
the main {\KBMAG} directory.

The use of the library is best illustrated by providing a report of an
actual session, with comments interspersed.
The first example illustrates the use of 'kbprog' from {\GAP}.
First create the following external file called 'S3' (which, we shall assume
is in the directory |../kb_data|).

|
_RWS := rec(
           isRWS := true,
  generatorOrder := [a,b],
        inverses := [a,b],
        ordering := "shortlex",
       equations := [
         [b*a*b,a*b*a]
       ]
);
|

Now start {\GAP} and proceed as follows.

|
gap> Read("init.g");
gap> S3 := ReadRWS("../kb_data/S3");
#Reads in the file, and converts it to an internal rewriting-system.
#However, it is still displayed as in the external file!
rec(
           isRWS := true,
  generatorOrder := [a,b],
        inverses := [a,b],
        ordering := "shortlex",
       equations := [
         [b*a*b,a*b*a]
       ]
)

gap> #Now we run the Knuth-Bendix program using the function 'KB', which
gap> #calls the external program "kbprog".

gap> KB(S3);
#System is confluent.
#Halting with 3 equations.
true
gap> #S3 now contains the confluent set of equations.
gap> S3;
rec (
           isRWS := true,
     isConfluent := true,
        ordering := "shortlex",
  generatorOrder := [a,b],
        inverses := [a,b],
       equations := [
         [a^2,IdWord],
         [b^2,IdWord],
         [b*a*b,a*b*a]
       ]
)

gap> #Now we can do some word-reductions.
gap> ReduceWordRWS(S3,(a*b)^3);
IdWord
gap> ReduceWordRWS(S3,a*b*a);
a*b*a
gap> ReduceWordRWS(S3,b*a*b*a);
a*b
gap> SizeRWS(S3);
6
gap> EnumerateRWS(S3,0,5);
[ IdWord, a, a*b, a*b*a, b, b*a ]
|

The next example illustrates the use of the automatic group package from {\GAP}.
This example is not quite so trivial as the preceding one. The group is the
fundamental group of the complement of the Borromean rings, which is a
three-dimensional hyperbolic manifold. The word-acceptor was used to enumerate
the words of length up to 4. This was used to speed up the calculations
required for drawing views of a tesselation of hyperbolic space by regular
dodecahedra, which can be seen in the video ``Not Knot\'\'.

First make the following file, called |BR| (in the directory |../ag_data|).
|
_RWS := rec(
 isRWS := true,
 ordering := "shortlex",
 generatorOrder := [a,A,b,B,c,C,d,D,e,E,f,F],
 inverses := [A,a,B,b,C,c,D,d,E,e,F,f],
 equations := [
  [a*a*a*a,IdWord], [b*b*b*b,IdWord], [c*c*c*c,IdWord],
  [d*d*d*d,IdWord], [e*e*e*e,IdWord], [f*f*f*f,IdWord],
  [a*b*A*e,IdWord], [b*c*B*f,IdWord], [c*d*C*a,IdWord],
  [d*e*D*b,IdWord], [e*f*E*c,IdWord], [f*a*F*d,IdWord]
 ]
);
|

Now start {\GAP} and proceed as follows.

|
gap> Read("init.g");
gap> BR := ReadRWS("../ag_data/BR");
rec(
           isRWS := true,
  generatorOrder := [a,A,b,B,c,C,d,D,e,E,f,F],
        inverses := [A,a,B,b,C,c,D,d,E,e,F,f],
        ordering := "shortlex",
       equations := [
         [a^4,IdWord],
         [b^4,IdWord],
         [c^4,IdWord],
         [d^4,IdWord],
         [e^4,IdWord],
         [f^4,IdWord],
         [a*b*A*e,IdWord],
         [b*c*B*f,IdWord],
         [c*d*C*a,IdWord],
         [d*e*D*b,IdWord],
         [e*f*E*c,IdWord],
         [f*a*F*d,IdWord]
       ]
)

gap> #Now we run the automatic group program using the function 'AutGroup',
gap> #which calls the external program "autgroup".

gap> AutGroup(BR);

.....   (output omitted)   .....

gap> SizeRWS(BR);
"infinity"

gap> #First do a few reductions.
gap> ReduceWordRWS(BR,a*B*c*D*e*F);
a*B*a*c^2*e
gap> ReduceWordRWS(BR,(a*c*e)^5);
a*c*a*B*c*a*B*c*a*B*c*a*B*c*e

gap> #Now we enumerate words in the group up to length 4.
gap> #``SortEnumerate\'\'\ puts them in order of increasing length.
gap> SortEnumerateRWS(BR,0,4);
[ IdWord, a, A, b, B, c, C, d, D, e, E, f, F, a^2, a*b, a*B, a*c, a*C, a*d, 
  a*D, a*e, a*E, a*f, a*F, A*b, A*B, A*c, A*C, A*d, A*D, A*e, A*E, A*f, A*F, 
  b*a, b^2, b*c, b*C, b*d, b*D, b*e, b*E, b*f, b*F, B*a, B*c, B*C, B*d, B*D, 
  B*e, B*E, B*f, B*F, c*a, c*A, c*b, c^2, c*e, c*E, c*f, c*F, C*a, C*A, C*b, 
  C*d, C*D, C*e, C*E, C*f, C*F, d*a, d*A, d*b, d*B, d*c, d^2, d*f, d*F, D*a, 
  D*A, D*b, D*B, D*c, D*e, D*E, D*f, D*F, e*A, e*b, e*B, e*c, e*C, e*d, e^2, 
  E*A, E*b, E*B, E*c, E*C, E*d, E*f, E*F, f*B, f*c, f*C, f*d, f*D, f*e, f^2, 
  F*a, F*A, F*B, F*c, F*C, F*e, a^2*b, a^2*B, a^2*c, a^2*C, a^2*d, a^2*D, 

  .....  (much output omitted)   .....

  F*e*b*f, F*e*b*F, F*e*B*a, F*e*B*c, F*e*B*C, F*e*B*d, F*e*B*D, F*e*B*e, 
  F*e*B*E, F*e*B*f, F*e*B*F, F*e*c*a, F*e*c*A, F*e*c*b, F*e*c^2, F*e*c*E, 
  F*e*c*f, F*e*c*F, F*e*C*A, F*e*C*b, F*e*C*d, F*e*C*D, F*e*C*E, F*e*C*f, 
  F*e*C*F, F*e*d*a, F*e*d*A, F*e*d*b, F*e*d*B, F*e*d*c, F*e*d*F ]

gap> #Finally, let\'s see how many words there are of different lengths.
gap> for ct in [0..6] do
> Print(SizeEnumerateRWS(BR,ct,ct),"\n");
> od;
1
12
102
812
6402
50412
396902
gap> quit;
|



%%%%%%%%%%%%%%%%%%%%%%%%%%%%%%%%%%%%%%%%%%%%%%%%%%%%%%%%%%%%%%%%%%%%%%%%%%%%%
%%
%%  and the bibliography
%%
\begin{sloppypar}
\bibliographystyle{alpha}
\newcommand{\ignore}[1]{}
\catcode`\'=12 \catcode`\<=12 \catcode`\*=12
\catcode`\|=12 \catcode`\:=12 \catcode`\"=12
\bibliography{manual}
\end{sloppypar}

\end{document}
