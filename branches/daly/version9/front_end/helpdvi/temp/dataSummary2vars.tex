%  Typeset this with amstex and amsppt version 2.0.
\input amstex
\documentstyle{amsppt}
\NoBlackBoxes


\tolerance 10000
\magnification\magstep1
\loadeusm  %for portability, we do this automatically
\font\Gfont=eusm10 

%John's super-duper definitions

\catcode`\@=11  %John's brilliant way to update for the 1991 classification
\def\subjclass{\nofrills@{{\rm1991 {\it Mathematics Subject
   Classification\/}.\enspace}}\subjclass@
 \DNii@##1\endsubjclass{\def\thesubjclass@{\def\usualspace
  {{\rm\enspace}}\eightpoint\subjclass@\ignorespaces##1\unskip.}}%
 \FN@\next@}
\catcode`\@=\active

       


%% The following code eliminates the "Typeset by Amstex" logo from page 1
\catcode`\@=11
\output={\output@}
\def\output@{\shipout\vbox{%
 \iffirstpage@ \global\firstpage@false
  \pagebody
\makefootline%
 \else \ifrunheads@ \makeheadline \pagebody
       \else \pagebody \makefootline \fi
 \fi}%
 \advancepageno \ifnum\outputpenalty>-\@MM\else\dosupereject\fi}
\catcode`\@=\active
%%  End of Stallings's stuff

\def\picture#1 by #2 (#3){\vbox to #2
      {\vfill\special{picture #3}
        \hrule width #1 height 0pt depth 0pt
        }}
\def\midpicture#1 by #2 (#3)#4{\goodbreak\midinsert
\hrule width #1 height 0pt depth 0pt
\hbox to\hsize{\hfill%
\picture#1 by #2 (#3)\hfill}%
\centerline{\rm #4}\endinsert}

\def\figA{
\hfil\picture 5.19 truein by 3.13truein (Hyptest1 scaled 850)\hfil}
\def\figB{
\hfil\picture 4.69truein by 2.18truein (Legal scaled 850)\hfil}
\def\figC{
\hfil\picture 3.56truein by 1.76truein (Reject1 scaled 850)\hfil} 
\def\figD{
\hfil\picture 3.21truein by 1.79truein (Reject2 scaled 850)\hfil}
\def\figE{
\hfil\picture 5.64truein by 3.64truein (ParamBox scaled 850)\hfil}
\def\figF{
\hfil\picture 3.81truein by 1.82truein (Reject3 scaled 850)\hfil}
\def\figG{
\hfil\picture 4.43truein by 2.06truein (Reject4 scaled 850)\hfil}
\def\figH{
\hfil\picture 3.49truein by 1.83truein (Reject5 scaled 850)\hfil}
\def\figI{
\hfil\picture 3.64truein by 1.78truein (Reject6 scaled 850)\hfil}    



\pagewidth{ 4.5in}
\pageheight{7in}
\hcorrection {.3in}
\define\im{\bold  i}
\define\to{\rightarrow}
\define\N{\Bbb N}
\define\Z{\Bbb Z}
\define\C{\Bbb C}
\define\R{\Bbb R}
\define\V{\Cal V}
\define\U{\cal U}



\topmatter


\endtopmatter


\document



$\vphantom{xxx}$

\centerline{\bf THE DATA SUMMARY TOOL}

\bigskip

The {\bf Data Summary Tool} will construct a frequency distribution for each of the two given data
sets. Recall that a frequency distribution is a tabular summary of the data that arranges the data
into classes and then gives a count or frequency in each class.  The classes will either be
constructed automatically or by user prompts.
 \bigskip

\centerline{ For further information click here}


Note: The link here should be to the long description of data summary in one-variable data sets.

    