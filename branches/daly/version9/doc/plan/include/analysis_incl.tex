%% This is AMSLaTeX source, included by plan.tex.


%%%%%%%%%%%%%%%%%%%%%%%%%%%%%%%%%%%%%%%%%%%%%%%%%%%%%%%%%%%%%%%%%%%%%%
\chapter{Analysis}


%%--------------------------------------------------------------------
\section{The Front End}

%%....................................................................
\subsection{Introduction}

The purpose of the front end is to

\begin{itemize}

\item
allow for the end user to define persistent algebraic objects, and
delete existing ones;

\item
present the currently existing objects in a comprehendible way;

\item
display all information known about a given object;

\item
facilitate the initiation of queries about collections of existing
objects;

\item
provide a means of controlling the strategy and execution of those
queries which are not known to terminate in small time and space.

\end{itemize}


Queries will generate information in the form of new algebraic
objects, along with free-format text, such as
\begin{verbatim}
Let w = a b^-1 a b^-1
\end{verbatim}
\begin{verbatim}
The word w = v^2, where v = a b^-1
\end{verbatim}
etc. This information is divided into three categories: primary, which
is what was explicitly asked for; secondary, which was not asked for
directly, but is related and likely to be of interest; tertiary, which
is of little interest to most people, but of great interest to some.


The three persistent components of the front end are

\begin{enumerate}

\item
{\em A greeting window.}\\
The function of this window is to anchor global menus, and to display
a greeting which gives the end user a clue as to how to proceed.

\item
{\em A session log.}\\
This is a global transcript of all transactions between the end user
and the system.

\item
{\em A graphical workspace.}\\
This contains icons representing all algebraic objects which
currently exist in the session, and all artificial objects which,
e.g., control an expensive query.

\end{enumerate}


%%....................................................................
\subsection{The Greeting Window}

This anchors the standard {\tt File} and {\tt Edit} menus, and a {\tt
Help} menu.


%%....................................................................
\subsection{The Session Log}

This is a window of editable, scrollable text, comprising a global
transcript of the current session. The text is selectable, so that
algebraic objects defined therein can be copied and pasted into
an object definition dialog.

The types of information which might be written to the log are:

\begin{itemize}

\item
An acknowledgement of the creation of an end user object.

\item
The result of a query, along with any secondary information.

\end{itemize}

A log entry might include the date and time at which it was posted.


%%....................................................................
\subsection{The Workspace}\label{The_Workspace}

The workspace is a scrollable window which contains only icons
representing algebraic and artificial objects. Other graphical
artifacts may be added later, such as arrows indicating relationships
between the objects. The underlying plane over which the window
scrolls expands automatically to make room for new icons.

The workspace supports precisely four end user events:

\begin{enumerate}

\item
{\em Select zero or more objects.}\\
This is supported by the standard mouse protocol.

\item
{\em Drag an object to a new position.}\\
This is supported by the standard mouse protocol.

\item
{\em Display information about the object represented by a given
icon.}\\
This is supported by the standard mouse protocol for de-iconifying an
icon.  It may bring up a top-level window which contains the
information.

\item
{\em Initiate an action on the currently selected objects.}\\
The action may be a group-theoretic query, or a synthetic action such
as creating an object, or deleting the selected ones.  The workspace
has menus, the contents of which change as the selection changes, to
reflect which actions are appropriate on the selection. When the
ordering on the selection is ambiguous, all possibilites are listed.

It is evident from the name of an action whether its purpose is
primarily to return an object. If so, the object goes into the
workspace. If not, any object constructed as a byproduct of the action
is secondary.

\end{enumerate}

Attached to the workspace, possibly at the bottom, is a minilog
containing the last session log entry posted. Its purpose is
reminiscent of the emacs minibuffer, or an X console.


%% . . . . . . . . . . . . . . . . . . . . . . . . . . . . . . . . . .
\subsubsection{Artificial Objects}\label{Artificial_Objects}

We currently envisage two classes of artificial objects in the
workspace: those which allow for the initialization and control of
expensive computations, such as the word problem, and report their
current status; and those which enumerate things.


\subsubsubsection{Problem Objects}


A problem object contains standard components:

\begin{itemize}

\item
A description of the group-theoretic parameters of the problem.

\item
A list of component computations, along with a means of specifying
{\em Abstract Resource Credits} (ARCs).  An ARC is a unit of
computation which takes, very roughly, on the order of seconds, rather
than milliseconds or hours. Implementors will strive to break lengthy
computations into natural steps, so that each step can reasonably be
expected to execute within this range of time, but there will be wide
variations.

It is not anticipated for the first release, but parameters for the
component computations may also be specified.

\item
A panel of control buttons, with these behaviors and semantics:

\begin{enumerate}

\item
{\em Start}: Active initially, permanently inactive after starting.
The problem does not {\em run} (consume ARCs) until it is started.

\item
{\em Suspend}: Active only when the problem is running. ARCs may be
changed only before starting, and when the problem is suspended.

\item
{\em Resume}: Active only when the problem is suspended.

\item
{\em Terminate}: Active after the problem starts. Halts all activities
of the problem, and frees any resources used by it directly.

\item
{\em Cancel}: Always active. Reverts any editing, and closes.

\item
{\em Close}: Always active. Enter any editing, and iconify.

\end{enumerate}


\item
A log for all primary, secondary, and tertiary data generated by the
computation.

\end{itemize}



\subsubsubsection{Enumerators}

The examples we have in mind come in two flavors: those which make new
workspace objects one at a time, such as a random small-cancellation
group generator, and those which enumerate large amounts of data which
the end user must be able to browse in some sensible way, such as a
normal closure enumerator.

Generally speaking, there will be many parameters for such
objects. The parameters will be used both for initialization, and for
`changing direction' on-the-fly.

As these are closely related to ideal objects (below), their
properties are still nebulous.


%% . . . . . . . . . . . . . . . . . . . . . . . . . . . . . . . . . .
\subsubsection{Algebraic Objects}

The kinds of algebraic objects which can exist in the workspace
are: groups, subgroups, morphisms, elements, and sets of elements or
morphisms, as well as {\em ideal} objects, discussed in the next
section.

The view window for such objects contains certain standard
components: the name of the object, its textual definition
(scrollable), a description of its heritage, and a log which contains
all primary and secondary information discovered about the object.

Subgroups, morphisms, elements, and sets of elements or morphisms are
clearly understood by the system to `belong' to higher level
structures (a word belongs to a group), and possibly to depend on
other objects (a morphism belongs to a hom set, and depends on one or
two groups). The user may perceive this at present only from the
heritage information for each object.

Some end user conventions on words:

\begin{itemize}

\item
Words use 1 based indexing.

\item
Words are not freely reduced on input, but stored as given.

\item
Functions which return words which are understood to be the best
element representative that can be (efficiently) found are free to
alter (e.g., freely reduce) the word. Some functions are understood to
manipulate words syntactically, so they do not reduce the result.

\end{itemize}


%% . . . . . . . . . . . . . . . . . . . . . . . . . . . . . . . . . .
\subsubsection{Ideal Objects}

By {\em ideal sets} we mean sets which may be infinite, but are at
least recursively enumerable. We try to get a handle on them by
approximating them with finite sets. It is not clear yet whether an
ideal set is comprised of, e.g., all approximations computed so far,
or just the current `best', etc.

We plan to support at least ideal subsets of groups, subgroups, and
sets such as $Hom(G,H)$. They may or may not be endowed with
group-theoretic structure. We don't know yet, but ideal sets may look
to the end user more like ongoing computations rather than static
objects.

We have not as yet identified any properties of ideal sets which would
appear to the end user to be common properties of all ideal sets.


%%--------------------------------------------------------------------
\section{Functionality}

The following sections list the desired end user functionality for
various classes of groups. The functionality is in the form of
functions or queries which the end user can invoke.  The inputs are
understood to be workspace objects, except integers, for which the end
user must be prompted.  All functions which return objects are
understood to return workspace objects.

As these are end user queries, there is no need for a unique return
type. For example, the end user can ask whether an element of a group
$G$ is in a term of the lower central series for $G$. If the answer is
yes, the user gets secondary information: in this case, the term of
the series to which the element belongs.

There is an important paradigm for representing computations which are
either lengthy or possibly non-terminating. We associate with the
computation a persistent workspace object, through which the end user
can: allocate ARCs to, guide the strategy of, monitor the progress of,
and receive results from, the computation. The computation returns a
result in the usual sense only when it can answer the question or
produce the desired object; otherwise, it consumes all of its
allocated ARCs, and becomes idle until given more ARCs or explicitly
terminated (see \S\ref{Artificial_Objects}).



%% This is AMSLaTeX source, included by analysis_incl.tex.


%%....................................................................
\subsection{Free Groups}

Fix the following workspace objects: let $F$ be a free group of
finite rank, $H$ and $K$ be finitely generated subgroups of $F$, and
$w$, $u$, and $v$ be words in the generators of $F$.


%% . . . . . . . . . . . . . . . . . . . . . . . . . . . . . . . . . .
\subsubsection{Elements}

\begin{enumerate}

\wish{Inverse}
{$F$, $w$}
{The syntactic inverse of $w$}

\wish{Free Reduction}
{$F$, $w$}
{The freely reduced form of $w$}

\wish{Shorten}
{$F$, $w$}
{The word which is $w$ with the first generator-inverse pair removed}

\wish{Product}
{$F$, $u$, $v$}
{A canonic representative of $uv$}

\wishtwo{Word Problem}
{$F$, $w$}
{Yes or No}
{The freely reduced form of $w$}

\wish{The $i$-th initial segment of a word}
{$F$, $w$, a positive integer $i$}
{The $i$-th initial segment of $w$}

\wish{The $i$-th terminal segment of a word}
{$F$, $w$, a positive integer $i$}
{The $i$-th terminal segment of $w$}

\wish{A subword of a word}
{$F$, $w$, a pair of integers $1\leq i\leq j \leq |w|$}
{The subword of $w$ beginning with the
$i$-th letter of $w$ and ending with the $j$-th letter of $w$
}

\wish{The $n$-th element of $F$}
{$F$, a positive integer $n$}
{The $n$-th element of $F$ (using the lexicographic ordering)}

\wish{The $n$-th next element in the lex order}
{$F$, $w$, a positive integer $n$}
{The $n$-th next element of $w$}

\wish{Initialize an object which enumerates the elements of $F$}
{$F$}
{An object which can enumerate the elements of $F$}

\wish{Equality Problem of elements given as words}
{$F$, $u$, $v$}
{Yes or No}

\wishtwo{Conjugacy Problem}
{$F$, $u$, $v$}
{Yes or No}
{A word $w$ such that $u^w=_F v$}

\wishtwo{Power Problem}
{$F$, $w$}
{Yes or No}
{The maximal root $u$ and the integer $n$ such that $u^n =_F w$}

\wishtwo{Maximal root of an element}
{$F$, $w$}
{The maximal root $u$ of $w$}
{The integer $n$ such that $u^n =_F w$}

\wishtwo{Element in commutator subgroup}
{$F$, $w$}
{Yes or No}
{If yes, $w$ written as a product of commutators}

\wishtwo{Element is a commutator}
{$F$, $w$}
{Yes or No}
{If yes, $w$ written as a commutator}

\wish{$\dag$ Initialize a random stream of sets of $n$ elements of $F$}
{$F$, a positive integer $n$}
{A random stream of sets of $n$ elements of $F$}

\wish{$\dag$ Initialize a random stream of elements of $F$}
{$F$, positive integers $m$, $n$}
{A random element of $F$ of length at least $m$, at most $n$}

\wish{$\dag$ Hausdorff derivative}
{$F$, $w$, a generator $x$ of $F$}
{The Hausdorff derivative, with respect to $x$, of $w$}

\end{enumerate}


%% . . . . . . . . . . . . . . . . . . . . . . . . . . . . . . . . . .
\subsubsection{Elements in Subgroups}

\begin{enumerate}

\wishtwo{Subgroup Membership Problem}
{$F$, $w$, $H$}
{Yes or No}
{If yes, $w$ written as a structured word in the generators given by
the user for $H$, a Nielsen basis for $F$, and $w$ written as a
structured word in that Nielsen basis
}

\wishtwo{Power of an element in a subgroup}
{$F$, $w$, $H$}
{Yes or No}
{As for membership problem, along with minimal power}

\wishtwo{Conjugate of an element in a subgroup}
{$F$, $w$, $H$}
{Yes or No}
{As for membership problem, along with conjugator}

\end{enumerate}


%% . . . . . . . . . . . . . . . . . . . . . . . . . . . . . . . . . .
\subsubsection{Subgroups}

\begin{enumerate}

\wishtwo{Subgroup Containment Problem: $H\subseteq K$}
{$F$, $H$, $K$}
{Yes or No}
{If yes, the generators of $H$ expressed in the generators of $K$}

\wishtwo{Subgroup Equality Problem}
{$F$, $H$, $K$}
{Yes or No}
{If yes, the generators of $H$ expressed in the generators of $K$, and vice
versa
}

\wishtwo{Subgroup Conjugacy Problem}
{$F$, $H$, $K$}
{Yes or No}
{The conjugator}

\wish{Index of a subgroup Problem}
{$F$, $H$}
{The index of $H$ in $F$}

\wish{Compute a virtual free complement for a subgroup (M.~Hall)}
{$F$, $H$}
{A subgroup $K$ of finite index in $F$ with $H$ as a free factor, $F:K$}

\wish{Subgroup-Enumerator}
{$F$, $H$}
{Initializes a subgroup element enumerator in the workspace}

\wish{Normal-closure-enumerator}
{$F$, $H$}
{Initializes an enumerator of $gp_F(H)$ in the workspace}

\wish{Whitehead production of a basis}
{$F$, $H$}
{A basis for $H$ obtained by Whitehead transformations}

\wish{Nielsen production of a basis}
{$F$, $H$}
{A Nielsen basis for $H$}

\wish{Schreier representative of an element}
{$F$, $H$, $w$}
{The Schreier representative of the coset of $H$ containing $w$}

\wish{Is a subgroup normal}
{$F$, $H$}
{Yes or No}

\wish{The normaliser of a subgroup}
{$F$, $H$}
{The finitely generated subgroup which is the normaliser of $H$}

\wish{Enumeration of basic commutators}
{$F$}
{A sequence enumerator, which enumerates the
elements in a basic sequence in the given generators of $F$.
}

\wish{Join of two Subgroups}
{$F$, $H$, $K$}
{The subgroup generated by $H$ and $K$}

\wish{Intersection of two Subgroups}
{$F$, $H$, $K$}
{The subgroup $H\cap K$}

\wish{Infinitely generated intersection}
{$F$, $H$, $K$}
{Ideal subgroup: $H\cap gp_F(K)$}

\wish{Infinitely generated intersection}
{$F$, $H$, $K$}
{Ideal subgroup: $gp_F(H)\cap gp_F(K)$}

\wishtwo{Extension-to-a-basis}
{$F$, a finite set $Y$ of words in the generators of $F$}
{Yes or No, as the set $Y$ can be extended to a basis of $F$}
{If yes, the rest of the basis}

\wish{Free-generators-question}
{$F$, a finite set $Y$ of words in the generators of $F$}
{Yes or No, as the set $Y$ is a basis of $F$}

\wishtwo{Is a subgroup a free factor}
{$F$, $H$}
{Yes or No}
{If yes, a Nielsen basis $B$ for $H$, and a Nielsen basis for
$F$ which contains $B$.
}

\wish{$\dag$ Ball of radius $n$}
{$F$, a non-negative integer $n$}
{The ball in the Cayley graph of $F$, relative to the user-given generators, of
radius $n$, i.e., the set of all elements of $F$ of length at most $n$.
}

\wish{$\dag$ Subgroup generation (random)}
{$F$, $H$, a non-negative integer $n$}
{Random generation of $n$ elements of $H$}

\wish{$\dag$ Generation of a subset}
{$F$, a non-negative integer $n$}
{Random generation of a subset of $F$ of cardinality $n$}

\wish{$\dag$ Generation of subsets}
{$F$, two non-negative integers $n$ and $m$}
{Random generation of $n$ subsets of $F$ of cardinality $m$}

\wishtwo{$\dag$ Subgroup Automorphic Image Problem}
{$F$, $H$, $K$}
{Yes or No}
{An automorphism $\alpha$ such that $\alpha(H)=_F K$}

\wish{$\dag$ Commutators coming from the derived series}
{$F$, a pair of non-negative integers $n$ and $m$.}
{The first $n$ elements from a basis for the $m$-th term of
the derived series of $F$.
}

\end{enumerate}


%% . . . . . . . . . . . . . . . . . . . . . . . . . . . . . . . . . .
\subsubsection{Morphisms}

\begin{enumerate}

\wishtwo{Test for an endomorphism being an automorphism}
{$F$, $\alpha\in Hom(F,F)$}
{Yes or No}
{$\alpha^{-1}$}

\wish{Aut enumerator}
{$F$}
{Ideal group: $Aut(F)$}

\wish{Product of automorphisms}
{$F$, $\alpha, \beta\in Aut(F)$}
{$\alpha\beta$}

\wishtwo{Inner automorphism}
{$F$, $\alpha\in Aut(F)$}
{Yes or No}
{The $w$ such that $\alpha(x)=_F x^w$}

\wishtwo{Order of an automorphism}
{$F$, $\alpha\in Aut(F)$}
{Yes or No}
{The order of $\alpha$}

\wish{Inverse of automorphism}
{$F$, $\alpha\in Aut(F)$}
{$\alpha^{-1}$}

\wish{$\dag$ Fixed point set of an automorphism}
{$F$, $\alpha\in Aut(F)$}
{The generators of the subgroup of elements left
identically fixed by $\alpha$
}

\wish{$\dag$ Fixing subgroup of an element}
{$F$, $w$}
{The generators of the subgroup of $Aut(F)$ which leaves $w$ fixed}

\end{enumerate}

%% This is AMSLaTeX source, included by analysis_incl.tex.


%%....................................................................
\subsection{Finitely Presented Groups}

Fix the following workspace objects: let $G$, $G_1$, $G_2$ be a
finitely presented groups, $H$ and $K$ be finitely generated subgroups
of $G$, and $w$, $u$, and $v$ be words in the generators of $G$.


%% . . . . . . . . . . . . . . . . . . . . . . . . . . . . . . . . . .
\subsubsection{Elements}

\begin{enumerate}

\wishtwo{Word Problem}
{$G$, $w$}
{Yes or No}
{The best form of $w$}

\wish{Equality Problem of elements given as words}
{$G$, $u$, $v$}
{Yes or No}

\wishtwo{Conjugacy Problem}
{$G$, $u$, $v$}
{Yes or No}
{A word $w$ such that $u^w=_G v$}

\wishtwo{Power Problem}
{$G$, $w$}
{Yes or No}
{A word $u$ and an integer $n$ such that $u^n =_G w$}

\wishtwo{Maximal root of an element}
{$G$, $w$}
{The maximal root $u$ of $w$}
{The integer $n$ such that $u^n =_G w$}

\wishtwo{Element in commutator subgroup}
{$G$, $w$}
{Yes or No}
{If yes, $w$ written as a product of commutators}

\wishtwo{Element is a commutator}
{$G$, $w$}
{Yes or No}
{If yes, $w$ written as a commutator}

\wishtwo{Is an element in a term of the lower central series}
{$G$, $w$}
{Yes or No}
{If Yes, the term of the lower central series of $G$ to which $w$ belongs}

\end{enumerate}


%% . . . . . . . . . . . . . . . . . . . . . . . . . . . . . . . . . .
\subsubsection{Elements in Subgroups}

\begin{enumerate}

\wishtwo{Subgroup Membership Problem}
{$G$, $w$, $H$}
{Yes or No}
{If yes, $w$ written as a structured word in the generators given by
the user for $H$
}

\wishtwo{Power of an element in a subgroup}
{$G$, $w$, $H$}
{Yes or No}
{As for membership problem, along with (not necessarily minimal) power}

\end{enumerate}


%% . . . . . . . . . . . . . . . . . . . . . . . . . . . . . . . . . .
\subsubsection{Subgroups}


\begin{enumerate}

\wishtwo{Subgroup Containment Problem: $H\subseteq K$}
{$G$, $H$, $K$}
{Yes or No}
{If yes, the generators of $H$ expressed in the generators of $K$}

\wishtwo{Subgroup Equality Problem}
{$G$, $H$, $K$}
{Yes or No}
{If yes, the generators of $H$ expressed in the generators of $K$, and vice
versa
}

\wishtwo{Subgroup Conjugacy Problem}
{$G$, $H$, $K$}
{Yes or No}
{The conjugator}

\wish{Subgroup is central}
{$G$, $H$}
{Yes or No}

\wish{Is a subgroup normal}
{$G$, $H$}
{Yes or No}

\wish{Normaliser of a subgroup}
{$G$, $H$}
{?}

\wish{Centralizer of a subgroup enumerator}
{$G$, $H$}
{?}

\wish{Index of a subgroup Problem}
{$G$, $H$}
{The index of $H$ in $G$}

\wish{Normal-closure-enumerator}
{$G$, $H$}
{Initializes an enumerator of $gp_G(H)$ in the workspace}

\wish{Random normal closure}
{$G$, $H$}
{?}

\wish{Intersection enumerator}
{$G$, $H$, $K$}
{?}

\wish{Infinitely generated intersection}
{$G$, $H$, $K$}
{Ideal subgroup: $H\cap gp_G(K)$}

\wish{Infinitely generated intersection}
{$G$, $H$, $K$}
{Ideal subgroup: $gp_G(H)\cap gp_G(K)$}

\wish{Generation}
{$G$, $H$}
{Yes or No as $H$ generates $G$}

\wish{Enumerate relations of a subgroup}
{$G$, $H$}
{?}


\end{enumerate}


%% . . . . . . . . . . . . . . . . . . . . . . . . . . . . . . . . . .
\subsubsection{Morphisms}

\begin{enumerate}

\wish{Hom enumerator}
{$G_1$, $G_2$}
{An enumerator of $Hom(G_1,G_2)$}

\wishtwo{Test for an endomorphism being an automorphism}
{$G$, $\alpha\in Hom(G,G)$}
{Yes or No}
{$\alpha^{-1}$}

\wish{Product of automorphisms}
{$G$, $\alpha, \beta\in Aut(G)$}
{$\alpha\beta$}

\wishtwo{Inner automorphism problem}
{$G$, $\alpha\in Aut(F)$}
{Yes or No}
{The $w$ such that $\alpha(x)=_G x^w$}

\wishtwo{Order of an automorphism problem}
{$G$, $\alpha\in Aut(F)$}
{Yes or No}
{The order of $\alpha$}

\wish{Inverse of automorphism}
{$G$, $\alpha\in Aut(G)$}
{$\alpha^{-1}$}

\wish{$\dag$ Fixed point set of an automorphism}
{$G$, $\alpha\in Aut(G)$}
{The generators of the subgroup of elements left
identically fixed by $\alpha$
}

\wish{$\dag$ Fixing subgroup of an element}
{$G$, $w$}
{The generators of the subgroup of $Aut(G)$ which leaves $w$ fixed}

\end{enumerate}


%% . . . . . . . . . . . . . . . . . . . . . . . . . . . . . . . . . .
\subsubsection{The Group}

\begin{enumerate}

\wish{The center problem}
{$G$}
{Yes or No as $G$ has non-trivial center}

\wish{The isomorphism problem}
{$G_1$, $G_2$}
{An isomorphism between $G_1$ and $G_2$ if there is one}

When the following answer Yes, they store a faithful representation of
the corresponding type with $G$. The representation is available to
the end user.

\wish{Is a group trivial}
{$G$}
{Yes or No}

\wish{Is a group abelian}
{$G$}
{Yes or No}

\wish{Is a group perfect}
{$G$}
{Yes or No}

\wish{Is a group infinite}
{$G$}
{Yes or No}

\wish{Is a group free}
{$G$}
{Yes or No}

\wish{Is a group a non-trivial free product}
{$G$}
{Yes or No}

\wish{Is a group a non-trivial amalgamated product}
{$G$}
{Yes or No}

\wish{Is a group a non-trivial direct product}
{$G$}
{Yes or No}

\wish{Is a group simple}
{$G$}
{Yes or No}

\wish{Is a group nilpotent}
{$G$}
{Yes or No}

\wish{Is a group polycyclic}
{$G$}
{Yes or No}

\wish{Is a group solvable}
{$G$}
{Yes or No}

\wish{Is a group a one-relator group}
{$G$}
{Yes or No}

\wish{Is a group hyperbolic}
{$G$}
{Yes or No}

\wish{Is a group automatic}
{$G$}
{Yes or No}

\wish{Is a group a small cancellation group}
{$G$}
{Yes or No}

\end{enumerate}




%%--------------------------------------------------------------------
\section{Reanalysis, July 1995}

%%....................................................................
\subsection{The Front End}\update{08/08/95}

Some possible additional features of the front have been suggested:

\begin{itemize}

\item
The more sophisticated end users will need a way to navigate among the
extant computations, and add new ones below the top level. Figures~\ref{WPview}
and \ref{NQview} suggest some possibilities.

\begin{figure}[hbtp]
\epsfbox{ps/WPview.ps}
\caption{A toplevel problem manager view.}\label{WPview}
\end{figure}

\begin{figure}[hbtp]
\epsfbox{ps/NQview.ps}
\caption{A submanager view: nilpotent quotients.}\label{NQview}
\end{figure}

\item
An object view might display two logs, to keep tertiary output
separate from primary and secondary.

\item
It can be difficult to tell what functionality is available without
entering every possible kind of object, and looking at the menus. We
need an overall synopsis.

\item
A definition dialog might allow toplevel name substitution, e.g.,
words which are already in the workspace could be entered by name as
relators for a group.

\item
The end user must be able to force a computation which is not actually
needed (because a solution already exists by other means). Such
`redundant' computations could be systematically used to cross-check
results.

\item
We need an (optional) syntax for naming subgroup generators, e.g.,
{\tt \{~a~=~[x,y],~b~=~z~y~x~\}}

\end{itemize}


%%....................................................................
\subsection{Revised Basic Functionality}\update{08/08/95}

This section describes the current capabilities of \magnus\ (checked
box), capabilities which are near completion (unchecked box), and
those desired in the near future (heart).

%% This is AMSLaTeX source, included by analysis_incl.tex.

%% . . . . . . . . . . . . . . . . . . . . . . . . . . . . . . . . . .
\subsubsection{Maps}

These do not depend on the domain or range. A homomorphism {\em is a} map, and
an automorphism {\em is a} homomorphism.

\begin{enumerate}

\capa{The image of a word under a map}{\done}

\capa{The image of a finitely generated subgroup under a map}{\nearly}

\capa{Composition of maps}{\done}

\capa{Power of an endomorphic map}{\done}

\end{enumerate}


%% . . . . . . . . . . . . . . . . . . . . . . . . . . . . . . . . . .
\subsubsection{Free Groups}

%% .  .  .  .  .  .  .  .  .  .  .  .  .  .  .  .  .  .  .  .  .  .  .
\subsubsubsection{Elements}

\begin{enumerate}

\capa{Formal inverse}{\done}

\capa{Free reduction}{\done}

\capa{Shorten}{\nearly}

\capa{Formal product}{\done}

\capa{Word problem}{\done}

\capa{The $i$-th initial segment of a word}{\done}

\capa{The $i$-th terminal segment of a word}{\done}

\capa{A subword of a word}{\nearly}

\capa{The $n$-th element of $F$}{\nearly}

\capa{The $n$-th next element in the lex order}{\nearly}

\capa{Enumerate elements of $F$}{\nearly}

\capa{Equality of words}{\done}

\capa{Conjugacy problem}{\done}

\capa{Power problem}{\nearly}

\capa{Maximal root of an element}{\nearly}

\capa{Element in commutator subgroup}{\nearly}

%@rn Given $x_1^{\epsilon_1}\dots x_n^{\epsilon_n}$, cumulate list of words
%@rn of length $n$, where the $i$th word is found as follows: let $s_j$ be
%@rn the abelian representative of $x_1^{\epsilon_1}\dots
%@rn x_{j}^{\epsilon_{j}}$. If $\epsilon_i=1$ the word is
%@rn $s_{i-1}x_i(\overline{s_{i-1}x_i})^{-1}$. If $\epsilon_i=-1$, the word
%@rn is $s_{i-1}x_i^{-1}(\overline{s_{i-1}x_i^{-1}})^{-1}$.

\capa{Element is a commutator}{\done}

\capa{Random stream of elements of $F$}{\nearly}

\capa{Hausdorff derivative}{\planned}

\end{enumerate}


%% .  .  .  .  .  .  .  .  .  .  .  .  .  .  .  .  .  .  .  .  .  .  .
\subsubsubsection{Elements in Subgroups}

\begin{enumerate}

\capa{Subgroup membership problem}{\done}

\capa{Power of an element in a subgroup}{\planned}

\capa{Conjugate of an element in a subgroup}{\nearly}

\end{enumerate}


%% .  .  .  .  .  .  .  .  .  .  .  .  .  .  .  .  .  .  .  .  .  .  .
\subsubsubsection{Subgroups}

\begin{enumerate}

\capa{Subgroup containment}{\done}

\capa{Subgroup equality problem}{\done}

\capa{Subgroup conjugacy problem}{\nearly}

\capa{Index of a subgroup problem}{\done}

\capa{Virtual free complement for a subgroup}{\done}

\capa{Normal closure enumerator}{\nearly}

We want this for any subgroup.

\capa{Whitehead production of a basis}{\nearly}

\capa{Nielsen production of a basis}{\done}

\capa{Schreier representative of an element}{\done}

\capa{Is a subgroup normal}{\done}

\capa{The normaliser of a subgroup}{\done}

\capa{Join of two subgroups}{\done}

\capa{Intersection of two subgroups}{\done}

\capa{Extension to a basis}{\nearly}

%@rn Iff Whitehead basis is a subset of the parent group generators (modulo
%@rn inverses).

\capa{Free generators question}{\nearly}

%@rn Iff free basis has same size.

\capa{Enumeration of basic commutators}{\planned}

\capa{Infinitely generated intersection}{\planned}

\capa{Is a subgroup a free factor}{\planned}

%@rn Iff Whitehead gives subset of parent group generators?

\capa{Ball of radius $n$}{\planned}

\capa{Random subgroup generation}{\planned}

\capa{Is a subgroup an automorphic image of another}{\planned}

\capa{Commutators coming from the derived series}{\planned}

\end{enumerate}


%% .  .  .  .  .  .  .  .  .  .  .  .  .  .  .  .  .  .  .  .  .  .  .
\subsubsubsection{Morphisms}

\begin{enumerate}

\capa{Test for an endomorphism being an automorphism}{\done}

\capa{Automorphism enumerator}{\done}

\capa{Product of automorphisms}{\done}

\capa{Inner automorphism}{\done}

\capa{Order of an automorphism}{\nearly}

\capa{Inverse of automorphism}{\done}

\capa{Fixed point set of an automorphism}{\planned}

\capa{Fixing subgroup of an element}{\planned}

\end{enumerate}

%% .  .  .  .  .  .  .  .  .  .  .  .  .  .  .  .  .  .  .  .  .  .  .
\subsubsubsection{The Group}

\begin{enumerate}

\capa{Extend by an automorphism}{\done}

\end{enumerate}




%% . . . . . . . . . . . . . . . . . . . . . . . . . . . . . . . . . .
\subsubsection{Finitely Presented Groups}

%% .  .  .  .  .  .  .  .  .  .  .  .  .  .  .  .  .  .  .  .  .  .  .
\subsubsubsection{Elements}

\begin{enumerate}

\capa{Word problem}{\done}

\capa{Equality problem}{\done}

\capa{Conjugacy problem}{\nearly}

\capa{Is an element in a term of the lower central series}{\nearly}

\capa{Element in commutator subgroup}{\nearly}

\capa{Element is a commutator}{\planned}

\capa{Power problem}{\planned}

\capa{Maximal root of an element}{\planned}

\end{enumerate}


%% .  .  .  .  .  .  .  .  .  .  .  .  .  .  .  .  .  .  .  .  .  .  .
\subsubsubsection{Elements in Subgroups}

\begin{enumerate}

\capa{Subgroup membership problem}{\planned}

\capa{Power of an element in a subgroup}{\planned}

\end{enumerate}


%% .  .  .  .  .  .  .  .  .  .  .  .  .  .  .  .  .  .  .  .  .  .  .
\subsubsubsection{Subgroups}


\begin{enumerate}

\capa{Subgroup containment problem}{\planned}

\capa{Subgroup equality problem}{\planned}

\capa{Subgroup conjugacy problem}{\planned}

\capa{Is a subgroup central}{\planned}

\capa{Is a subgroup normal}{\planned}

\capa{Normaliser of a subgroup}{\planned}

\capa{Enumerate centralizer of a subgroup}{\planned}

\capa{Index of a subgroup}{\planned}

\capa{Enumerate normal closure}{\nearly}

\capa{Enumerate normal closure randomly}{\nearly}

\capa{Intersection enumerator}{\planned}

\capa{Infinitely generated intersection}{\planned}

\capa{Is the subgroup the whole group}{\planned}

\capa{Enumerate relations of a subgroup}{\done}

\capa{Find low index subgroups}{\planned}

\capa{Present a finite index subgroup}{\planned}

\end{enumerate}


%% .  .  .  .  .  .  .  .  .  .  .  .  .  .  .  .  .  .  .  .  .  .  .
\subsubsubsection{Morphisms}

\begin{enumerate}

\capa{Does a map extend to a hom}{\nearly}

\capa{Hom enumerator}{\nearly}

\capa{Is an endomorphism an automorphism}{\planned}

\capa{Is an automorphism inner}{\planned}

\capa{Order of an automorphism}{\planned}

\capa{Inverse of an automorphism}{\planned}

\capa{Fixed point set of an automorphism}{\planned}

\capa{Fixing subgroup of an element}{\planned}

\end{enumerate}


%% .  .  .  .  .  .  .  .  .  .  .  .  .  .  .  .  .  .  .  .  .  .  .
\subsubsubsection{The Group}

\begin{enumerate}

\capa{The center problem}{\planned}

\capa{The isomorphism problem}{\planned}

\capa{The order problem}{\done}

\capa{Is the group abelian}{\nearly}

\capa{Is the group free}{\planned}

\capa{Is the group a non-trivial free product}{\planned}

\capa{Is the group a non-trivial amalgamated product}{\planned}

\capa{Is the group a non-trivial direct product}{\planned}

\capa{Is the group simple}{\planned}

\capa{Is the group nilpotent}{\nearly}

\capa{Is the group polycyclic}{\planned}

\capa{Is the group solvable}{\planned}

\capa{Is the group a one-relator group}{\planned}

\capa{Is the group hyperbolic}{\planned}

\capa{Is the group automatic}{\done}

\capa{Is the group a small cancellation group}{\planned}

\capa{Is the given presentation small cancellation}{\done}

\capa{Does the group have a rewriting system}{\done}

\capa{Compute the Homology of the group}{\nearly}

\capa{Compute random representations in random $C'({1\over 6})$ groups}{\nearly}

\capa{Compute random linear representations}{\planned}

\capa{Compute representations in the Higman-Neumann group}{\planned}

\capa{Compute representations in some finitely presented infinite simple groups}
{\planned}

\capa{Compute generic representations in $GL(n,R)$}{\planned}

\capa{Enumerate presentations by Tietze transformations}{\planned}

\end{enumerate}




%% . . . . . . . . . . . . . . . . . . . . . . . . . . . . . . . . . .
\subsubsection{Abelian Groups}

As for finitely presented groups, but in addition, or done better:

%% .  .  .  .  .  .  .  .  .  .  .  .  .  .  .  .  .  .  .  .  .  .  .
\subsubsubsection{Elements}

\begin{enumerate}

\capa{Word problem}{\done}

\capa{Equality problem}{\done}

\capa{Power problem}{\planned}

\capa{Maximal root of an element}{\planned}

\end{enumerate}


%% .  .  .  .  .  .  .  .  .  .  .  .  .  .  .  .  .  .  .  .  .  .  .
\subsubsubsection{Elements in Subgroups}

\begin{enumerate}

\capa{Subgroup membership problem}{\planned}

\capa{Power of an element in a subgroup}{\planned}

\end{enumerate}


%% .  .  .  .  .  .  .  .  .  .  .  .  .  .  .  .  .  .  .  .  .  .  .
\subsubsubsection{Subgroups}


\begin{enumerate}

\capa{Subgroup containment problem}{\planned}

\capa{Subgroup equality problem}{\planned}

\capa{Index of a subgroup Problem}{\planned}

\capa{Intersection}{\planned}

\capa{Is the subgroup the whole group}{\planned}

\capa{Enumerate relations of a subgroup}{\planned}

\end{enumerate}


%% .  .  .  .  .  .  .  .  .  .  .  .  .  .  .  .  .  .  .  .  .  .  .
\subsubsubsection{The Group}

\begin{enumerate}

\capa{The isomorphism problem}{\nearly}

\capa{Order problem}{\done}

\capa{Is the group free}{\done}

\capa{Is the group simple}{\nearly}

\capa{The rewriting system}{\planned}

\end{enumerate}



%% . . . . . . . . . . . . . . . . . . . . . . . . . . . . . . . . . .
\subsubsection{Small Cancellation Groups}

As for finitely presented groups, but in addition, or done better:

%% .  .  .  .  .  .  .  .  .  .  .  .  .  .  .  .  .  .  .  .  .  .  .
\subsubsubsection{Elements}

\begin{enumerate}

\capa{Word problem}{\done}

\capa{Equality problem}{\done}

\capa{Conjugacy problem}{\nearly}

\end{enumerate}


%% .  .  .  .  .  .  .  .  .  .  .  .  .  .  .  .  .  .  .  .  .  .  .
\subsubsubsection{The Group}

\begin{enumerate}

\capa{Order problem}{\done}

\end{enumerate}



%% . . . . . . . . . . . . . . . . . . . . . . . . . . . . . . . . . .
\subsubsection{Nilpotent Groups}

As for finitely presented groups, but in addition, or done better:

%% .  .  .  .  .  .  .  .  .  .  .  .  .  .  .  .  .  .  .  .  .  .  .
\subsubsubsection{Elements}

\begin{enumerate}

\capa{Word problem}{\done}

\capa{Equality problem}{\done}

\capa{Conjugacy problem}{\nearly}

\end{enumerate}

%% .  .  .  .  .  .  .  .  .  .  .  .  .  .  .  .  .  .  .  .  .  .  .
\subsubsubsection{The Group}

\begin{enumerate}

\capa{Mal'cev basis}{\nearly}

\capa{The rewriting system}{\planned}

\end{enumerate}



%% . . . . . . . . . . . . . . . . . . . . . . . . . . . . . . . . . .
\subsubsection{One Relator Groups}

As for finitely presented groups, but in addition, or done better:

%% .  .  .  .  .  .  .  .  .  .  .  .  .  .  .  .  .  .  .  .  .  .  .
\subsubsubsection{Elements}

\begin{enumerate}

\capa{Word problem}{\done}

\end{enumerate}

