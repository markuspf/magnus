%% This is AMSLaTeX source, included by plan.tex.


%%%%%%%%%%%%%%%%%%%%%%%%%%%%%%%%%%%%%%%%%%%%%%%%%%%%%%%%%%%%%%%%%%%%%%
\chapter{Implementation}

%%--------------------------------------------------------------------
\section{Areas Needing Work}\update{08/09/95}

%%....................................................................
\subsection{General}

The type system ({\tt Type.h}) should be reconsidered. We have been
sliding into the use of dynamic typing. This can be very powerful, but
it throws away a principal advantage of a language like {\tt C++},
namely, that when the compiler can determine the type of a variable,
it can also check its usage.  With dynamic typing, one says to the
compiler: ``I know what I'm doing. I don't need your help.'' \magnus\
is already 145,000 lines. I think we need the compiler's help.


%%....................................................................
\subsection{Classes in {\tt general/}}

%% . . . . . . . . . . . . . . . . . . . . . . . . . . . . . . . . . .
\subsubsection{Class {\tt AssociationsOf}}

To make lookup faster than $O(n)$, this could use either or both of
hashing and binary search. But the linear method must be retained too,
since not every {\tt Key} class can supply the hash or order function
needed. Some points to consider:

\begin{itemize}

\item
How will this effect the current usage of AssociationsOf?

\item
Should we use stl, or C library functions, or write the low-level code
ourselves?

\item
Should the new functionality be reflected in subclasses?

\end{itemize}


%% . . . . . . . . . . . . . . . . . . . . . . . . . . . . . . . . . .
\subsubsection{Class {\tt BlackBox}}

The main requirement is that execution and polling of input be
asynchronous.

The external binary will be invoked by either a {\tt system} call or
{\tt fork/exec}. The latter is preferable, so a {\tt kill} signal can
be sent to subprocesses which may not ask for input for some time (if
ever).

Input from the spawned process can be made non-blocking (this requires
constructing the {\tt ifstream} from a file descriptor, so the {\tt
BlackBox} can get its hands on low-level stuff first).  There must be
a special convention (such as returning EOF) to indicate that a read
request failed because there is nothing to read yet.

Thus it is not desirable to return the actual {\tt ifstream} to
callers, since a reference to it may be erroneously stored. The input
might be buffered in an {\tt istrstream}.

Even with a process id, unix appears to provide no way to tell whether
the spawned process started successfully. Perhaps the first request
for input could be flagged, and timed out. Any subsequent crash might
be detected by the pipe closing. Hangs are undetectable by their very
nature.

A {\tt BlackBox} could use {\tt getpid} to get better control over its
temporary file names.

A {\tt BlackBox} might look in several directories, such as {\tt
/tmp}, {\tt /usr/tmp}, and {\tt \$HOME/tmp}, and use {\tt statfs} to see if
there is a reasonable amount of space (10MB?).  It could warn the end
user if not enough is available.

We will have to redirect both {\tt stdout} and {\tt stderr} to the
{\tt BlackBox}'s input channel.


%% . . . . . . . . . . . . . . . . . . . . . . . . . . . . . . . . . .
\subsubsection{Class {\tt ListOf<T>}}

This might override {\tt new} and {\tt delete} for more efficient
allocation of its {\tt Cells}.

%% . . . . . . . . . . . . . . . . . . . . . . . . . . . . . . . . . .
\subsubsection{Class {\tt VectorOf<T>}}

A vector is the most obvious data structure for many things. The
current interface and semantics of {\tt VectorOf} place us squarely
between Scylla and Charybdis: one the one hand, everyone has used {\tt
VectorOf} for a wide variety of purposes, so if we change it at all we
will have to deal with a large cascade of changes throughout \magnus.
On the other hand, {\tt VectorOf} is clearly inadequate for many
purposes.

One of the most common uses is as an automatic variable in a function.
For this we might want a variant which does not protect the data with
reference counting, but still does bounds checking and cleans up after
itself. This would be more dangerous to use, but if its use is kept
local by convention, the danger is acceptable.

Before starting on any such variant, we will need a reasonably
complete list of the desired functionality (i.e., should the new
vectors be resizable? Do we want to catenate them? Etc.)

We will have to think about some problems with {\tt VectorOf} we have
observed:

\begin{itemize}

\item
It requires class {\tt T} to have a default constructor. Possibly
derive a subclass {\tt VectorOfPtr<T*>} to make the danger of
uninitialized data explicit?

\item
Entries returned by {\tt operator [ ]} cannot invoke member functions,
since the conversion from {\tt VectorItemRef} is not allowed. The
current workaround is the two new members:

  {\tt T val( int i ) const \{ return look()->val(i); \}}

  {\tt T\& ref( int i ) \{ return change()->ref(i); \}}

which will probably have to be permanent. If we had a vector class
which did not protect its data, this problem would not arise. In
particular, {\tt operator [ ]} could always return a reference to a
{\tt T}, thus not requiring {\tt T} to have efficient copying.


\end{itemize}

%%....................................................................
\subsection{Higher level structures}


%% . . . . . . . . . . . . . . . . . . . . . . . . . . . . . . . . . .
\subsubsection{Class {\tt Map}}

This needs a complete overhaul, mostly to hew out nonsense.

%% . . . . . . . . . . . . . . . . . . . . . . . . . . . . . . . . . .
\subsubsection{Class {\tt Word}}

\begin{enumerate}

\ckitem{\notick}{@rn}{}{
Merge functionality of {\tt isProperPower} and {\tt maximalRoot} in class
{\tt Word}
}

\ckitem{\notick}{@rn}{}{
Sort out {\tt Word::replaceGenerators}
}

\ckitem{\notick}{@rn}{}{
Improve {\tt int Word::agreementLength(const Word\&) const}
}

\end{enumerate}


%% . . . . . . . . . . . . . . . . . . . . . . . . . . . . . . . . . .
\subsubsection{The {\tt Group} hierarchy}

\begin{enumerate}

\ckitem{\notick}{@rn}{}{
Remove almost all `intelligence' from this hierarchy, making it
contain only such things as presentations and generator names.
}

\ckitem{\notick}{@rn}{}{
Excise {\tt Chars FGGroup::nameOfGenerator(int i) const}.
}

\ckitem{\notick}{@rn}{}{
Replace {\tt FPGroupRep::shortenByRelators(const Word\&)}.

Perhaps the Dehn's algorithm code (which uses {\tt SymmetricRelators})
belongs here, since we want to use it even if we do not know that it
solves the word problem.  }

\end{enumerate}

%% . . . . . . . . . . . . . . . . . . . . . . . . . . . . . . . . . .
\subsubsection{Class {\tt Subgroup}}

This should be almost nothing more than a container class for the
generators and their names. Thus it can still derive from {\tt
FGGroup}.



%%--------------------------------------------------------------------
\section{Details Checklist}\update{08/09/95}


\begin{enumerate}

\ckitem{\notick}{@rn}{}{
The manual is now almost useless. Redo it.
}

\ckitem{\notick}{}{}{
Derive {\tt class Stack<T>} and {\tt class Queue<T>}
from {\tt class ListOf<T>}.
}

\ckitem{\notick}{@rn}{}{
Make something like a global file unlinker with signal handlers.
}

\ckitem{\notick}{@sl}{}{An intrusive list iterator.}

\ckitem{\notick}{@rn}{}{
Make {\tt Trichotomy} a class, with conversion from, {\em but not to},
{\tt bool}.

This will be a problem if you have done things like:

{\tt Trichotomy done; ...}

{\tt if ( done ) \{...\}}

(there should not be a conversion to {\tt bool}, since {\tt done ==
dontknow} triggers a run-time error).
}

\ckitem{\notick}{@rn}{}{
See about yanking {\tt Bool, YES, NO, DONTKNOW}. This will be a
problem if anyone has illicitly used these as integers, but that is
all the more reason to banish them.
}

\ckitem{\notick}{@rn}{}{
See to {\tt istream::eof()} problem in {\tt WordParser}.
}

\ckitem{\notick}{}{}{
Do more conversions between container classes.
}

\ckitem{\notick}{@rn}{}{
Put free-reducing version of
{\tt EltRep* WordRep::operator* (const EltRep\&) const} in {\tt FreeGroup}?
}

\ckitem{\notick}{@rn}{}{
Consider use of {\tt long} in {\tt RefCounter} and {\tt PureRep}.
}

\ckitem{\tick}{@rn}{6/95}{
Overhaul class {\tt Chars} as a function of time and necessity.
}

\ckitem{\tick}{@rn}{6/95}{
Excise GNU random stuff
}

\ckitem{\tick}{@sr}{12/94}{
Remove all dependence on the {\tt Group} hierarchy in
{\tt FSA} and {\tt RKBP}.
}

\end{enumerate}
