%% This is AMSLaTeX source, included by plan.tex.


%%%%%%%%%%%%%%%%%%%%%%%%%%%%%%%%%%%%%%%%%%%%%%%%%%%%%%%%%%%%%%%%%%%%%%
\chapter{Design}


%%--------------------------------------------------------------------
\section{The System}\update{08/15/95}

There are four abstraction layers in the design of \magnus:

\begin{enumerate}

\item
The system as a whole.

\item
The primary modules: the front end and session manager.

\item
The session manager subsystems, such as computation managers.

\item
The algorithms and procedures which \magnus\ supports.

\end{enumerate}

The highest layer is shown in Figure~\ref{arch}.

\begin{figure}[htbp]
\epsfbox{ps/arch.ps}
\caption{The System}\label{arch}
\end{figure}

The front end (\FE) and session manager (\SM) run as separate
processes, and communicate with each other via named pipes with
non-blocking I/O. The black boxes are separate processes, usually
`foreign' programs contributed to \magnus. They are wrapped by native
\magnus\ code which monitors and controls them, according to their
API's, via named pipes with non-blocking I/O connected to their
standard I/O.  Some of these programs expect to do additional I/O via
ordinary files.

The other layers are treated in the following sections.


%%--------------------------------------------------------------------
\section{The Primary Modules}\update{08/15/95}

We strive to keep the communication protocol adhered to by the \SM\
and \FE\ as simple as possible, for the obvious reasons which include
robustness and maintainability.

Our primary leverage on this goal is for the \SM\ to `bootstrap' the
\FE\ with templates for the messages expected by the \SM. For each of
a fairly small number of end user events, the \SM\ supplies the \FE\
with a template to be filled in with the details of the event (such
as, which objects are selected). Thus, the \FE\ knows as little as
possible about what it says to the \SM. Conversely, the \SM\ does not
give detailed instructions about the `look and feel' of the \FE, but
communicates abstractly about what is to be displayed.

In the following sections we will need these `forward declarations':

An {\em algebraic object} (\S\ref{algebraic_objects}) is an entity
which represents a word, matrix, group, etc, and the information known
about it. It exists in the \SM, and is represented graphically in some
form by the \FE.

A {\em computation manager} (\S\ref{computation_managers}) is an
entity which represents and manages the state of a lengthy
computation. It may manage other computation managers; in some cases,
that is its only purpose. Only those computation managers
corresponding to a direct end user query (the {\em problem objects} of
\S\ref{Artificial_Objects}) are represented graphically by the \FE.


%%....................................................................
\subsection{The Front End}\update{08/15/95}

The responsibilities of the \FE\ are:

\begin{itemize}

\item
To start the \SM\ upon startup, attaching pipes to its {\tt stdin} and
{\tt stdout}, and to kill it on exit.

\item
To manage the graphic location and manipulation of each object in the
workspace.

\item
To keep track of which objects are selected.

\item
To decide which menu items should be posted for a given selection.

This requires more information from the \SM\ than the simple types of
the selected objects in some cases. For example, composition of maps,
and testing a hom for being an automorphism require comparing domains
and ranges.  Subgroup operations like containment, equality,
conjugacy, and computing joins and intersections require comparing
supergroups.

Therefore, the \FE\ maintains a `blind' database, in the form of an
associative array, on behalf of the \SM. The database can be updated
at any time, and contains such information as `object 12 is an
explicit subgroup of object 7', and `object 7 is known to be abelian'.
The \SM\ can attach {\em conditions} to the posting of a menu item,
expressed in terms of the database keys and a few operators like {\tt
==} and {\tt \&\&}.  The \FE\ resolves these conditions by extracting
the corresponding data, without knowing what it means.

\item
To forward the end user's menu choices to the \SM.

\item
To manage view windows for all algebraic objects, and for those
computation managers which export views.

At present, the \FE\ provides a small number of predefined, generic
views. The \SM\ can specify, at the time an object is created, which
of these generic views to use for the object, and what the contents of
certain variable fields should be.

As the objects which these views represent become richer in structure
relevant to the end user, it may be necessary to extend the \FE\
$\leftrightarrow$ \SM\ protocol to allow the \SM\ to build views out
of a small number of graphical components predefined by the \FE.

\end{itemize}


%%....................................................................
\subsection{The Session Manager}\update{09/11/95}

The primary responsibilities of the \SM\ in aggregate are to:

\begin{enumerate}

\item
Manage the physical storage of algebraic objects and computation
managers.

Each algebraic object and computation manager class derives from an
abstract base, {\tt SMObject}. Each {\tt SMObject} has an {\em object
id} (OID) which is unique during a session, but is not necessarily
persistent across sessions. The OIDs are the definitive means by which
the \FE\ and \SM\ talk about the same objects.

{\tt TheObjects} permits {\tt AlgObjSmith}, {\tt ToolSmith}, and {\tt
SMObjects} to enroll new {\tt SMObjects}. It requires a list of those
{\tt SMObjects} to which the enrolled one can store a reference; this
is so that the deletion of objects does not leave dangling references.
See Figure~\ref{sm1}.

\begin{figure}[htbp]
\epsfbox{ps/sm1.ps}
\caption{Object enrollment and storage}\label{sm1}
\end{figure}

\item
Supply information to the \FE, either spontaneously or upon request.
There is a separate class for each kind of message to the \FE, in
order to hide the message protocol and ease coding. Some are sketched
in Figure~\ref{sm3}.

\begin{figure}[htbp]
\epsfbox{ps/sm3.ps}
\caption{Outgoing messages}\label{sm3}
\end{figure}

Messages to the \FE\ include:

\begin{itemize}

\item
A computation manager has significantly changed state.

{\tt OutMessage} may queue such messages in static storage until, say,
some time has elapsed, if the end user wants to reduce message traffic
for efficiency (a session running overnight does not need to clog
pipes with up-to-the-second info when the end user is asleep).

\item
A computation manager has discovered interesting information.

{\tt LogMessage} could derive from {\tt ostrstream} to allow for
maximally flexible construction of log messages.

\item
Export a view description for an object.

All concrete {\tt SMObjects} should be potentially viewable by the end
user, but in practice some views may be empty.

When a workspace object is created, its view description is included
in the message to the \FE\ which says that the object now exists. But,
for example, not all {\tt ComputationManagers} are workspace objects.
However, the end user will eventually have the ability to navigate
through supervising {\tt ComputationManagers} and request views on
subordinate ones. Thus, {\tt SMObject} has a pure virtual method for
supplying a view description.

\end{itemize}

\item
Initialize the \FE.

On startup, {\tt SessionManager} calls methods of {\tt AlgObjSmith}
and {\tt ToolSmith} to output menu definitions to the \FE. These
definitions contain the templates for the message the \FE\ should send
back when the end user chooses the corresponding menu item.

{\tt SessionManager} outputs global message templates itself.

There is no need for message classes here, since it is done exactly
once.

\item
Execute an incoming message.

{\tt SessionManager} periodically checks whether there is a message
from the \FE\ to read. It decides from each message header who the
recipient is, and passes the rest of the message to the appropriate
{\tt MessageHandler} (See Figure~\ref{sm2}). A pure virtual method of
{\tt MessageHandler} is overridded to do the reading.

\begin{figure}[htbp]
\epsfbox{ps/sm2.ps}
\caption{Receiving messages}\label{sm2}
\end{figure}

The message might say to:

\begin{itemize}

\item
Create an algebraic object.

{\tt AlgObjSmith} parses the ASCII definition of the object (supplied
in the message), and may have to pass a parse error message back to
the \FE. In this case, the \FE\ is responsible for resending the
message if the end user desires; {\tt AlgObjSmith} need not keep a
record. Otherwise, {\tt AlgObjSmith} uses {\tt CreateMsg} to tell the \FE\
that the object was successfully created.

\item
Do a computation.

{\tt ToolSmith} enrolls a {\tt ComputationManager} with {\tt
TheObjects} to do the computation.

\item
Update the state of a computation manager.

The end user has made generic state changes to the computation
manager: changed ARC values, or started/suspended/resumed it; or, the
end user has made state changes which are specific to the type of the
manager. A virtual member of {\tt ComputationManager} handles the
generic case, and derivates can read the update message as needed.

\item
Delete some objects.

When {\tt TheObjects} deletes a {\tt SMObject}, all of its {\em
dependencies}, i.e. {\tt SMObjects} which store a reference to it,
must be deleted also. For this purpose, {\tt TheObjects} stores a
partial order $\succ$, such that $A\succ B$ iff $B$ stores a reference
to $A$.

For greater safety, the enroll method of {\tt TheObjects} might
require a constructor wrapper and a list of the OIDs of the arguments,
rather than trust that the caller will be honest about which {\tt
SMObjects} were passed to the constructor of the {\tt SMObject} to be
enrolled. Certainly {\tt TheObjects} does not grant access to {\tt
SMObjects} via OIDs, to preclude rogue reference storing.

\item
Quit.

The destructors of derivatives of {\tt ComputationManager} should
insure that any pipes, temporary files, and subprocesses are cleaned
up. Thus {\tt TheObjects} should call these destructors, even though
{\tt exit} will free the memory anyway.

\end{itemize}

\item
Pass control to computation managers in a way which guarantees fair
multitasking.

{\tt TheObjects} knows which of the {\tt SMObjects} it stores are {\tt
ComputationManagers}. During `idle' periods, i.e., when there are no
incoming messages to process, it passes control to a {\tt
ComputationManager}. Each {\tt ComputationManager} is responsible for
returning control after a short time, but in some cases this
requirement may be too impractical to meet.

\end{enumerate}




%%--------------------------------------------------------------------
\section{The \SM\ Subsystems}

%%....................................................................
\subsection{Computation Managers}\label{computation_managers}\update{08/20/95}

Almost every computation attempted by \magnus\ is so complex, and
possibly non-terminating, that it is encapsulated by a class object.
These objects are called {\em computation managers} (\CMs). They
preserve the state of a computation, so that it can be cleanly
interrupted and continued. The most important property of a
\CM\ is that it tries to use CPU time within limits set
by its supervising manager, via ARCs.  Later, we will also limit the
memory used by a \CM\ in a similar way.

\CMs\ can interact with each other in three ways:

\begin{enumerate}

\item
One \CM\ may {\em allocate} some of its resources to another.

When end users initiate a computation in \magnus, they allocate
resources to it (currently just ARCs, but later a memory bound), so
that it does not get out of control. The corresponding \CM\ may, for
convenience or out of necessity, want the services of one or more
other \CMs. The end user may indicate which of these to use (see
\S\ref{Artificial_Objects}). Also, a \CM\ may make autonomous decisions
to create subordinate \CMs; this is the {\em supervision} interaction,
below. In both cases, resources must be passed from one \CM\ to
another. The details of resource allocation are discussed in
\S\ref{resource_manager}.

\item
One \CM\ may {\em supervise} another.

For example, the end user may wish to compute some nilpotent quotients
of a group (perhaps to prove that a word is not trivial), but has no
particular preference as to {\em which} quotients. A \CM\ could exist
which a) checks which quotients have already been computed, b) decides
which one to do next, c) autonomously creates a subordinate \CM\ to
compute the desired quotient. Of course, there would be an {\em
allocation} interaction between these \CMs.

\item
One \CM\ may {\em query} another for partial results.

This is a rare interaction. In some cases, e.g., the \CM\ which seeks
a confluent rewrite system, even partial results (a rewrite system
which is not known to be confluent) can reasonably used by another
\CM. Therefore, such \CMs\ permit others to access the partial
results in a useful way.

\end{enumerate}

These three interactions are illustrated in Figures~\ref{gcm}
and \ref{wordprob1}.


%% . . . . . . . . . . . . . . . . . . . . . . . . . . . . . . . . . .
\subsubsection{The Resource Manager}\label{resource_manager}

Each \CM\ has a {\em resource manager}, which is responsible for

\begin{itemize}

\item
Recording the resources allocated to the \CM, along with where they
came from.

\item
Recording which \CMs\ have been given resources by this one.

\item
When the \CM\ is suspended or terminated (\S\ref{Artificial_Objects}),
informing the resource manager of each \CM\ which has been given
resources that those resources may not be used.

\item
When the \CM\ is resumed after suspension, informing the resource
manager of each \CM\ which has been given resources that those
resources may now be used.

\item
Announcing to the \FE\ that the state of the \CM\ has changed as each
unit of each resource is used.

\end{itemize}


%%....................................................................
\subsection{The Group Computation Manager}\update{08/20/95}

The mother of all \CMs\ is the {\em group computation manager}
(\GCM). Many computations produce information which is useful for many
different things; e.g., a rewrite system which is not even known to be
confluent can sometimes be used to prove that a given word is trivial,
and also that the group is finite. Such computations must be shared.
The \GCM\ does no actual computation itself; it merely supervises all
shared computations of information about a group.

\begin{figure}[hbtp]
\epsfbox{ps/gcm.ps}
\caption{A \GCM}\label{gcm}
\end{figure}

An example is sketched in Figure~\ref{gcm}. The arrows indicate the
supervision interactions.


%%....................................................................
\subsection{The Group Information Center}\label{gic}\update{08/20/95}

The {\em group information center} (\GIC) contains all information
known about a finitely presented group. This can be as simple as a
{\tt Trichotomy} which indicates whether the group is abelian, or as
complex as a rewrite system or automatic structure.

Fundamentally, we do not distinguish the various algebraic classes of
finitely presented groups. An abelian or nilpotent group is
represented by \magnus\ in exactly the same way as an arbitrary
finitely presented group; they have an associated \GCM, and a \GIC\
which (dynamically) records abelianness or the nilpotentcy class.

The reason is twofold: an arbitrary finitely presented group might be
discovered to be abelian during a session, and we do not want to
change its internal representation when this happens. Also, a group
may have been entered by the end user as nilpotent of class 12, but
there is no practical way to use our nilpotent algorithms on it; in
effect, for our purposes, the group is {\em not} nilpotent, and we
have to use other procedures on it.

The design automatically allows for any procedure for finitely
presented groups to be applied, and automatically handles `promotion'
to a better algebraic class in a uniform way.

For any given piece of information which might be known about a
finitely presented group, the \GIC\ has four interacting kinds of
members:

\begin{enumerate}

\item
A data member which can store the information, when and if it is
found.

\item
A method by which the information can be supplied (usually by a \CM).
This method also reports the discovery of the information to the \FE.

\item
A method which says whether the information exists.

\item
An accessor for the information, which must trigger an error if it is
called when the information does not exist.

\end{enumerate}



%%....................................................................
\subsection{Algebraic Objects}\label{algebraic_objects}\update{08/20/95}

An {\em algebraic object} is really just a container for the actual
data which represents the object, the data which represents what is
known about the object, and housekeeping information required by the
\SM.

Thus an {\tt SMWord} has a {\tt Word} as a data member, perhaps a {\tt
Trichotomy} indicating whether it is known to be trivial (etc), and an
OID.

An {\tt SMFPGroup} has a {\tt FPGroup}, a \GIC, an OID, and a
reference to the \GCM\ which was created by the {\tt SMFPGroup}
constructor.

Figure~\ref{subsystems} shows some of the subsystems we have
discussed.

\begin{figure}[hbtp]
\epsfbox{ps/subsys.ps}
\caption{Some \SM\ subsystems}\label{subsystems}
\end{figure}


%%....................................................................
\subsection{A Scenario}\update{08/20/95}

Figure~\ref{wordprob1} shows a {\tt WordProblem} interacting with a
group's \GCM\ and \GIC. The abelian quotient has already been computed,
and the result passed to the \GIC. The {\tt WordProblem} has mapped
its word into this quotient, but the image was trivial, and the group
itself is not abelian. The class 2 nilpotent quotient has also been
computed, but was of no use for the same reason. Thus the {\tt
WordProblem} continues to allocate resources to nilpotent quotients
(two simultaneously, in this case), and to finding a rewrite system.
Two incarnations of Knuth-Bendix, with different orderings, are
running.  Since an answer may be found by even a non-confluent rewrite
system, the {\tt WordProblem} directly queries the Knuth-Bendix
manager for these.

\begin{figure}[hbtp]
\epsfbox{ps/wordprob1.ps}
\caption{A {\tt WordProblem} in progress}\label{wordprob1}
\end{figure}

In Figure~\ref{wordprob2}, one of the Knuth-Bendices has completed,
passed its rewrite system to the \GIC, and terminated itself. The {\tt
WordProblem} discovers this upon its next activation, and gets its
answer.

\begin{figure}[hbtp]
\epsfbox{ps/wordprob2.ps}
\caption{A {\tt WordProblem} finishing}\label{wordprob2}
\end{figure}


%%--------------------------------------------------------------------
\section{Algorithms and Procedures}\update{08/22/95}

Coders in this layer need only provide for the computation to cleanly
save its state (preferably after a few seconds), so that it can be
continued later. An appropriate \CM\ can then wrap it for cooperation
with the \SM.


%%--------------------------------------------------------------------
\section{Other Functionality}\update{08/22/95}

%%....................................................................
\subsection{Save and Restore}

It may involve more programming than it is worth to save and restore
ongoing computations. But at the very least, we must save and restore
algebraic objects, and the contents of each \GIC. This means that all
classes of objects which a \GIC\ uses to store information must have
robust ASCII I/O. A side-benefit of this is that it gives us a
ready-made way to present `technical' views of internal data to the
end user, provided that the ASCII representation is human-readable.
