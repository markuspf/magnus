\documentclass[11pt]{article}
\title{\bf{Research project 9 \\ Alternating directions method}}
\author{\bf Dmitry Bormotov}
\date{}
\begin{document}
\maketitle

The classical parabolic differential equation for a function
$u(x,y,t)$ of the two space variables $x,y$ and the time variable $t$
is

\begin{equation}
u_{t} = u_{xx} + u_{yy}.  \label{10.95}
\end{equation}
A solution is required in a region $G \subset R^{2}$ of the $(x,y)$
plane with boundary $\Gamma$ for times $t>0$. The solution of
(\ref{10.95}) is subject to an {\em initial condition} 
\begin{equation}
u(x,y,0) = f(x,y) \ in \ G  \label{10.96}
\end{equation}
and to similar {\em boundary conditions} on the boundary $\Gamma$ as for
elliptic boundary value problems. The function $u(x,y,t)$ depends on
the time, thus the functions occurring in the Dirichlet, Neumann and
Cauchy conditions may also depend on t. The set of values $(x,y,t)$
for which the solution is sought consists of the half cylinder $R^{3}$
above the region $G$. 

To discretize the initial boundary value problem we use a regular
three-dimensional net which is generated from a regular net in the
region $G$ with the mesh size $h$ by translating it by multiples of k
in the time direction. Then we look for approximations $u_{\mu,v,j}$
of the solution $u(x,y,t)$ at the grid points $P(x_{\mu},y_{v},t_{j})$.

The approximation of the differential equation (\ref{10.95}) is done
in two stages.  The expression $u_{xx}+u_{yy}$ with partial
derivatives with respect to the space variables is approximated for a
fixed time layer $j$ at each grid point by an appropriate difference
expression. For the simplest case of a regular interior point
$P(x_{\mu},y_{v},t_{j})$ we use 
\[ (u_{xx}+u_{yy})_{P} \approx \frac{1}{h^2}
(u_{\mu,v+1,j}+u_{\mu-1,v,j}+u_{\mu,v-1,j}+u_{\mu+1,v,j}-4u_{\mu,v,j}).
\]

The partial derivative with respect to the time t may, for instance,
be approximated by means of the forward difference quotient in P
\[ u_{t} \approx \frac{1}{k}(u_{\mu,v,j+1}-u_{\mu,v,j}).\]
This defines the {\em explicit Richardson method} with the
computational scheme for a regular interior grid point
\begin{equation}
u_{\mu,v,j+1} = u_{\mu,v,j} +
r(u_{\mu,v+1,j}+u_{\mu-1,v,j}+u_{\mu,v-1,j}+u_{\mu+1,v,j}-4u_{\mu,v,j}).
 \label{10.97}
\end{equation}
Again we use the quantity $r = k/h^2$. In order to simplify the index
notation we number the grid points with unknown value $u$ in each time
layer from 1 to n. Then we combine the approximate values of the $j$th
time layer, with $t_{j} = jk$, in the vector
\begin{equation}
u_{j} = (u_{1,j}, u_{2,j}, \ldots , u_{n,j})^T \in R^n  \label{10.98}
\end{equation}
where the first index $i$ of $u_{i,j}$ refers to the number of the
grid point. The computational scheme (\ref{10.97}) can be written as
\begin{equation}
u_{j+1} = (I - rA)u_j + b_{j} \ \ (j = 0,1,2,\ldots). \label{10.99}
\end{equation}
The matrix $A \in R^{n \times n}$ is the coefficient matrix of the
system of difference equations which discretizes the Poisson equation
in the region $G$, and $b_j$ contains the constants that originate
form the boundary conditions.

The explicit method satisfies the condition of absolute stability if
and only if the absolute values of the eigenvalues of the matrix
$(I-rA)$ are less than one. The resulting condition for $r$ can in
general only be given for symmetric and positive definite matrices
$A$. In this case the eigenvalues $\lambda_v$ of $(I-rA)$ are related
to the eigenvalues $\mu_v$ of $A$ by
\[ \lambda_v = 1 - r \mu_v \ \ (v = 1,2,\ldots,n) \ \ \mu_v > 0. \] 
As $1-r\mu_v>-1$ must be satisfied for all $v$ we deduce that
\begin{equation}
r < 2/\max_v(\mu_v).  \label{10.100}
\end{equation}

For this reason the {\em implicit Crank-Nicholson method} must be
applied. In (\ref{10.97}) the expression in parentheses simply
replaced by the arithmetic mean of the expressions of the $j$th and
$(j+1)$th time layer. Instead of (\ref{10.99}) we obtain a computational
scheme  of the form
\begin{equation}
(2I + rA )u_{j+1} = (2I - rA)u_j + b_j \ \ (j = 0,1,2,\ldots). \label{10.102}
\end{equation}
This method is absolutely stable for symmetric and positive definite
matrices $A$ and for nonsymmetric matrices $A$ whose eigenvalues $\mu_v$
all have a positive real part, because the eigenvalues $\lambda_v$ of
$(2I + rA)^{-1} (2I-rA)$ are then in absolute value less than one for
all $r>0$.

The computation of the values $u_{j+1}$ at the grid points of the
$(j+1)$th time layer requires the solution of a system of linear
equations, (\ref{10.102}), with the matrix $(2I+rA)$ that is constant
for all time steps and usually diagonally dominant. Thus the $LR$
decomposition has to be performed just once, and then the processes of
the forward and backward substitution have to be applied to the known
right-hand side of (\ref{10.102}). In the case of a small mesh
size,$h$, the order of the matrix $(2I+rA)$ as well as the bandwidth
is large, so that the storage requirement and the computational effort
per time step is considerable.

In order to substantially reduce the mentioned requirements, Peaceman
and Rachford(1955) proposed a discretization so that in each step a
sequence of tridiagonal systems of equations has to be solved. The
idea consists of combining two different difference approximations per
step. For this reason the time step k is halved, and the auxiliary
values $u_{\mu,v,j+\frac{1}{2}} = u_{\mu,v}^\ast$ for the time $t_j+k/2 =
t_{j+\frac{1}{2}}$ are defined as the solution of the difference equations
\begin{eqnarray}
\frac{2}{k}(u_{\mu,v}^\ast - u_{\mu,v,j}) =
\frac{1}{h^2}(u_{\mu+1,v}^\ast - 2u_{\mu,v}^\ast + u_{\mu-1,v}^\ast)
\nonumber \\ \\ + \frac{1}{h^2}(u_{\mu,v+1,j} - 2u_{\mu,v,j} + u_{\mu,v-1,j}).
\nonumber \label{10.103}
\end{eqnarray}
To approximate $u_{xx}$ the second difference quotient is used with
the auxiliary values of the time layer $t_{j+\frac{1}{2}}$. The second
partial derivative $u_{yy}$, however, is approximated by means of,
known, approximations of the time layer $t_j$. The derivative $u_t$ is
approximated by the usual forward difference quotient with the half
step size $k/2$, of course. In the next step we take the auxiliary
values $u_{\mu,v}^\ast$ for fixed $v$ together, that is those values
that correspond to grid points along a net line parallel to the $x$
axis. For each such group of values the set of equations
(\ref{10.103}) is a tridiagonal system with the typical
representative
\begin{eqnarray}
-ru_{\mu-1,v}^\ast+(2+2r)u_{\mu,v}^\ast-ru_{\mu+1,v}^\ast \nonumber
 \\ = ru_{\mu,v-1,j}+(2-2r)u_{\mu,v,j}+ru_{\mu,v+1,j} \ \ r = k/h^2.
\label{10.104} 
\end{eqnarray}
The determination of all the auxiliary values $u_{\mu,v}^\ast$
requires the solution of a set of tridiagonal systems of equations,
each corresponding to a line of the net that is parallel to the $x$
axis. With these auxiliary values, the approximations $u_{\mu,v,j+1}$
of the time layer $t_{j+1}$ are determined from the difference
equations
\begin{eqnarray}
\frac{2}{k}(u_{\mu,v,j+1}-u_{\mu,v}^\ast) =
\frac{1}{h^2}(u_{\mu+1,v}^\ast - 2 u_{\mu,v}^\ast + u_{\mu-1,v}^\ast)
\nonumber \\ +\frac{1}{h^2}(u_{\mu,v+1,j+1} - 2u_{\mu,v,j+1} +
u_{\mu,v-1,j+1}).
\label{10.105} 
\end{eqnarray}
Now $u_{xx}$ is approximated by means of the known auxiliary values
and $u_{yy}$ by means of the desired approximations of the $(j+1)$th
time layer. For this part it is important that we unite the unknown
values $u_{\mu,v,j+1}$ for fixed $\mu$ in (\ref{10.105}), that is
those values that correspond to grid points on a net line parallel to
the $y$ axis. Each of these groups of unknowns, (\ref{10.105}), is the
solution of another tridiagonal system of equations whose typical
representative is
\begin{eqnarray}
-ru_{\mu,v-1,j+1}+(2+2r)u_{\mu,v,j+1}-ru_{\mu,v+1,j+1} \nonumber
\\ = ru_{\mu-1,v}^\ast+(2-2r)u_{\mu,v}^\ast+ru_{\mu+1,v}^\ast \ \ r = k/h^2.
\label{10.106} 
\end{eqnarray}
Once again we have to solve a sequence of tridiagonal systems of
linear equations for the unknowns $u_{\mu.v.j+1}$, each corresponding
to grid points on net lines parallel the $y$ axis. Due to change of
direction in which the grid points are grouped, the Peaceman and
Rachford procedure is usually called the {\bf alternating directions
method}.

\vspace{5mm}
\begin{center} \Large\textbf{Sources} \end{center}
\vspace{5mm}
1. H.R.Schwartz, Numerical Analysis, 1989 

\end{document}


