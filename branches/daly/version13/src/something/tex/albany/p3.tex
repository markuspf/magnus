\documentclass[12pt]{slides} 
\title{Genetic algorithms for solving
equations in free groups and semigroups.  Genus problem.}

\author{Dmitry Bormotov} \date{}
\begin{document}
%\maketitle
\pagestyle{empty}

\vspace{5mm}
\begin{center} \Large\textbf{Experiment 1} \end{center}
%\vspace{3mm}

Let us consider a simple experiment where equations are of the form
\[
a^N x \: b^N = 1
\]
where $x$ is a variable, $a$ and $b$ are generators of a free group,
$N$ is a positive integer. 

Though solutions are obvious in this case,
the general algorithm does not know that and therefore, the experiment
is sound. 

Given a simple equation like this we can control the length
of the minimal solution for each $N$ and estimate the growth of time
needed for different lengths. 


\newpage

\noindent
\begin{tabular}{|c|c|c|c|} \hline
Len & Av \# of gen & Av time per gen & Av time \\ 
\hline
10 & 13 & 0.00810277 & 0.105336 \\ \hline
20 & 38.7 & 0.00878626 & 0.340028 \\ \hline
30 & 175.3 & 0.0107125 & 1.8779 \\ \hline
40 & 194.7 & 0.0133946 & 2.60792 \\ \hline
50 & 215.3 & 0.0152716 & 3.28797 \\ \hline
60 & 412.7 & 0.0168407 & 6.95016 \\ \hline
100 & 899.5 & 0.026206 & 23.5723 \\ \hline
200 & 2013 & 0.0473142 & 95.2434 \\ \hline
\end{tabular}
%\vspace{3mm}

As can be seen from the description of the algorithm and as confirmed
by the table above, the average time per generation is growing
linearly with the length of the equation. The number of generations
can also be bound by a linear function, and therefore the average
time can be bounded by a quadratic polynomial. Though a more thorough
analysis is needed, the experiment has already shown dramatic
improvement over exponential (at least) algorithm discussed above.


\end{document}







