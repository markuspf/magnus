\documentclass[12pt]{slides} 
\title{Genetic algorithms for solving
equations in free groups and semigroups.  Genus problem.}

\author{Dmitry Bormotov} \date{}
\begin{document}
%\maketitle
\pagestyle{empty}

%\vspace{5mm}
\begin{center} \Large\textbf{The genus problem} \end{center}
\vspace{3mm}
Let $f$ be an element from the derived subgroup of a free
group $F$. Genus of $f$ is the minimal number of
commutators, say $n$, such that $f$ can be expressed as a
product of $n$ commutators. Let us consider the quadratic equation
\[
x_1^2 x_2^2 x_3^2 x_4^2 = 1 
\]
where $x_1,x_2,x_3,x_4$ are variables. {\em Genus } of a solution 
\[
x_1= u_1, x_2 = u_2, x_3 = u_3, x_4 = u_4 \ \  ( u_i \in F )
\]
is the genus of
the product $u_1 u_2 u_3 u_4$.
Genus of the equation is the supremum of the genuses of all its
solutions.  What is the genus of the equation?

\end{document}







