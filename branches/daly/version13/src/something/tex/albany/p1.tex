\documentclass[12pt]{slides} 
\title{Genetic algorithms for solving
equations in free groups and semigroups.  Genus problem.}

\author{Dmitry Bormotov} \date{}
\begin{document}
%\maketitle
%\pagestyle{empty}

%\vspace{5mm}
\begin{center} \Large\textbf{Equations in a free group} \end{center}
\vspace{12mm}

Given a free group $G$ we define an equation in variables
$x_1,x_2,...,x_n$ over $G$ as formal equation of the form
\[
g_1 x_{i_1}^{\epsilon_1} g_2 x_{i_2}^{\epsilon_2} ... x_{i_n}^{\epsilon_n} g_{n+1} = 1,
\]
where $g_i \in G, \epsilon_i = \pm 1, x_{i_j} \in {x_1,...,x_n}$. 

\vspace{5mm}
{\bf The problem} is to find a solution, i.e. a substitution $x_i
\rightarrow u_i, u_i \in G$ such that the equation becomes identity in
$G$.

\newpage

%\vspace{5mm}
\begin{center} \Large\textbf{Known results} \end{center}

\vspace{5mm}
{\bf Equations with one variable.} Lyndon (1960) described all
solutions. Appel (1968) obtained effective upper bounds. They
are at least exponential.

%\vspace{1mm}
{\bf Quadratic equations.} Comerford and Edmunds (1981,1989) found an
algorithm for solving quadratic equations and described all
solutions. Their algorithm is at least exponential.

%\vspace{1mm}
{\bf General case.} Makanin (1982) showed that there exists a
recursive function $f(u)$ such that an arbitrary equation $S = 1$
(over a free group $F$) has a solution in $F$ if and only if it has a
solution of length $ \le f(|S|)$, where $|S|$ is the length of
equation $S$. It is at least exponential.

%\vspace{1mm}
{\bf Quadratic case is the principal one.} Kharlampovich and Myasnikov 
(1998) showed that the problem of solving an arbitrary equation over a free
group can be reduced to solving a finitely many quadratic equations.

\end{document}
