\documentclass[12pt]{slides} 
\title{Genetic algorithms for solving
equations in free groups and semigroups.  Genus problem.}

\author{Dmitry Bormotov} \date{}
\begin{document}
%\maketitle
\pagestyle{empty}

%\vspace{5mm}
\begin{center} \Large\textbf{Population} \end{center}
\vspace{5mm}

Since we are looking for a solution for a given equation, it is
natural to work with a population of candidate solutions,
i.e. population consisting of a finite number of substitutions for the
variables of the equation. The algorithm uses population of 50 tuples
of words. Each tuple has the same number of elements as the number of
variables and thereby represent one possible substitution for the
variables.

\vspace{5mm}
\begin{center} \Large\textbf{Fitness function} \end{center}
\vspace{5mm}

The fitness function has to measure how close a potential solution is
to a real solution, i.e. if the substitution is performed, how close
the image of a given equation is to a trivial word. In this case one
of the simple choices is to use the length of the substituted image as
a fitness function with zero as as optimum value. 


\newpage

\begin{center} \Large\textbf{Selection} \end{center}
\vspace{5mm}

Survival of the fittest is implemented by {\em fitness proportional
selection} or {\em roulette wheel selection}.  The probability
$Pr(s_i)$ of selecting solution $s_i$ for reproduction is given by
\[
	Pr(s_i) = \frac{Fitness(s_i)}{\sum_{j=1}^p Fitness(s_j)},
\]
where $p$ is the population size. This process of selecting population
members for reproduction is called {\em selection}. Then we perform
reproduction by selecting $p$ pairs $(s_1,s_2)$ according to the
distribution of $Pr(s_i)$ and applying the {\em Crossover} and
{\em Mutation} operators. 

\newpage

\begin{center} \Large\textbf{Crossover} \end{center}
\vspace{5mm}

{\bf Crossover} between two tuples of words is done by coordinates,
i.e. the first variable of the first tuple crosses over with the first
variable of the second tuple, the second variable of the first tuple
crosses over with the second variable of the second tuple and so
on. Therefore we need to define a crossover on words. In this case the
classic one-point crossover can be used. In other words, we take a
random initial segment of the first word and concatenate it with a
random terminal segment of the second word.

\newpage

\begin{center} \Large\textbf{Mutation} \end{center}
\vspace{5mm}

{\bf Mutation} on tuples can again be defined through a mutation on
their components - words. A mutation is random change in a word, and
requires us to replace or one or more generators in a word. The
following three mutations were used in the algorithm:

%\vspace{3mm}
Insert a new letter in a randomly chosen position - 10\% chance

%\vspace{3mm}
Delete one randomly chosen letter(generator) from the word -  10\% chance

%\vspace{3mm}
Replace one randomly chosen letter by a different one -  80\% chance
\vspace{3mm}

\end{document}







