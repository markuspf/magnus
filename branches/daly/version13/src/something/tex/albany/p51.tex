\documentclass{slides} 
\title{Genetic algorithms for solving
equations in free groups and semigroups.  Genus problem.}

\author{Dmitry Bormotov} \date{}
\begin{document}
%\maketitle
\pagestyle{empty}

\begin{center} \Large\textbf{New fitness function} \end{center}
%\vspace{3mm}

Here we compare two words coordinate-wise. The fitness function is the
Hamming distance between the words. The shortest word is appended by
\$'s, so that the words have the same length. The fitness value is
initialized to zero and gets increased by one for every mismatch
between the letters. Then we try to minimize it.

However, under such fitness function the similar words $a b c d$ and
$b c d$ will have fitness 4, so it is better to compute values for all
cyclical permutations of one of the words and take the minimum. When
working with the forms from the theorem we must check all cyclical
permutations of a given constant anyway, so here we kill two birds
with one stone.

\end{document}







