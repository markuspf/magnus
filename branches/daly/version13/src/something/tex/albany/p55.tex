\documentclass{slides} 
\title{Genetic algorithms for solving
equations in free groups and semigroups.  Genus problem.}

\author{Dmitry Bormotov} \date{}
\begin{document}
%\maketitle
\pagestyle{empty}

\begin{center} \Large\textbf{Experiment 3} \end{center}
%\vspace{3mm}

In the example produced by D. Spellman 

$x_1 = a,$ \\
$x_2 = B a b,$ \\
$x_3 = a a B a a b b A A B A A b A A,$ \\
$x_4 = a a B a a b a a B a a B A A b A A B A A b A A$

Then $x_1^2 x_2^2 x_3^2 x_4^2 = 1$ \\ 
and $C = x_1 x_2 x_3 x_4$ is not a commutator.

Performing the substitution will give us 
\begin{eqnarray*}
C = a b^{-1} a b a^2 b^{-1} a^2 b^2 a^{-2} b^{-1} a^{-2} b a^{-2} \\ a^2 b^{-1}
 a^2 b a^2 b^{-1} a^2 b^{-1} a^{-2} b a^{-2} b^{-1} a^{-2} b a^{-2}
\end{eqnarray*}
As computed by the genetic algorithm it can be expressed as a product
of two commutators:
\begin{eqnarray*}
& [a^3 b^{-1} a^2 b a^2 b^{-1} a b a^{-1}, \\
& a^3 b^{-1} a^2 b a^2 b^{-1} a^{-1} b a^{-2} b^{-1} a^{-2} b a^{-3}] 
\end{eqnarray*}
and
\[
[a^{-1}, a b a^{-2} b^{-1} a^{-2} b a^{-2}]
\]

\end{document}







