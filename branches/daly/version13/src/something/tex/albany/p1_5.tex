\documentclass[12pt]{slides} 
\title{Genetic algorithms for solving
equations in free groups and semigroups.  Genus problem.}

\author{Dmitry Bormotov} \date{}
\begin{document}
%\maketitle
\pagestyle{empty}

%\vspace{5mm}
\begin{center} \Large\textbf{Structure of GA} \end{center}
%\vspace{2mm}

{\bf 1}. Start with a randomly generated population of $n$ chromosomes
  (candidate solutions) 

{\bf 2}. Calculate the fitness $f(x)$ of each chromosome $x$ in the population. 

{\bf 3}. Repeat the following steps until $n$ offspring have been created:

{\bf a)} select a pair of parent chromosomes from the current population,
the probability of selection being an increasing function of fitness;\\
{\bf b)} with probability $p_c$ cross over the pair at a randomly chosen
point to form two offspring;\\
{\bf c)} mutate the two offspring with probability $p_m$ and place the
resulting cromosomes in the new population.

{\bf 4}. Replace the current population with the new population.

{\bf 5}. Go to step 2.

\end{document}







