\documentclass{slides} 
\title{Genetic algorithms for solving
equations in free groups and semigroups.  Genus problem.}

\author{Dmitry Bormotov} \date{}
\begin{document}
%\maketitle
\pagestyle{empty}

\begin{center} \Large\textbf{Experiment 2} \end{center}
%\vspace{3mm}

This example comes from J. Comerford and Y. Lee. Let

$x_1 = a B A,$ \\
$x_2 = a b A A B A b a a B A,$ \\
$x_3 = a b A B A,$ \\
$x_4 = a a a b A$

Then $x_1^2 x_2^2 x_3^2 x_4^2 = 1$ \\ 
and $C = x_1 x_2 x_3 x_4$ is not a commutator.

Performing the substitution will give us 
\begin{eqnarray*}
C = a B A a b A A B A b a a B A a b A B A a a a b A
\end{eqnarray*}
As computed by the genetic algorithm it can be expressed as a product
of two commutators:
\begin{eqnarray*}
& [ A B a b a B a b a, A B a b A B A b a ] \\
& [ A B A b a, A B ] ] \\
\end{eqnarray*}

\end{document}







