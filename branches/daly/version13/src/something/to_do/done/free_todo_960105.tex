From gilbert Fri Jan  5 13:04:15 1996
Date: Fri, 5 Jan 96 13:04:15 EST
From: gilbert ( Gilbert Baumslag)
To: bormotov@jolly2
Content-Length: 2799

\def\m{\medskip\noindent}

\m
In checking in a free group, the help file should read:

Here are examples of how to define a free group:

<x,y>, <a,b,c;>, <x,y,z|> 

The tools menu that comes up when a free group has been
checked in should read:

Randomly enumerate automorphisms of F1

Randomly enumerate automorphisms of F1 of finite order
_________________________________________

Etc.

_________________________________________

*********************************************************
The dialogue box that comes up for randomly  enumerating
automorphisms of a free group should be headed:

Choose an integer

and the name Whitehead should be replaced by Nielsen

The dialogue box that comes up for randomly  enumerating
automorphisms of finite order of a free group should be 
headed:

Choose an integer

and the name Whitehead should be replaced by Nielsen

********************************************************
In checking in a word in a free group, the dialogue box
should read:

Define a word in the generators of F1

In the help file, conventions on word notation should be
the same as that for finitely presented groups:

Conventions on word notation:

Generator names can be any letter, such as a, b, ...,
x,y, ..  or any letter followed by a sequence of digits
such as a1, b23 and so on. In putting generators together
to form words, there must be white space between them unless
other punctuation is used. For example we allow  a (b c) in 
place of a b c. We sometimes use capital letters to denote 
inverses of generators. So the inverses of a,b,c11, .. are 
denoted by A, B, C11,... and also by a^-1, b^-1, c11^-1 and 
so on. The ^ is used to denote exponentiation. The exponents
are allowed to be group elements as well as integers, with 
a^b defined to be the conjugate B a b of a by b. Thus

a^b = B a b  = b^-1 a b .

Our convention is that ^ is left-associative. So

a^b^2 = (a^b)^2.

We use square brackets to denote commutators:

[a,b]= a^-1 b^-1 a b ( = A B a b ).

We then define, for n > 2, 

[a1,a2,...,an]= [[a1,a2],a3,...,an].

Thus multiple commutators are "left-normed".

**********************************************************
The tools menu when one checks in a word in a free
group should read as follows:

Compute the formal inverse of w
Freely reduce w
Cyclically reduce w
___________________________________

Is w trivial in F1?
Is w1 a proper power in F1?
Find the maximal root of w1 in F1
___________________________________

Compute the centraliser of w1 in F1
____________________________________

Delete selected objects and dependencies
____________________________________

Synopsis
____________________________________

*********************************************************
January 5
*********************************************************

