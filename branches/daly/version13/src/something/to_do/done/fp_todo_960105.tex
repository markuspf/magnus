From gilbert Fri Jan  5 15:45:18 1996
Date: Fri, 5 Jan 96 15:45:17 EST
From: gilbert ( Gilbert Baumslag)
To: bormotov@jolly2
Content-Length: 4577

Finitely presented groups menus, latest version
(throw out the old one which is dated January 5, 1996)

The  define menu for a finitely presented group
should read:

Define a finitely presented group

*****************************************************

The define menu for an abelian group should read:

Define an abelian group

*****************************************************

The define menu for a free group should read:

Define a free group

*****************************************************

The define menu for a nilpotent group should read:

Define a nilpotent group

*****************************************************

The define menu for a one-relator group should read:

Define a one-relator group

*****************************************************

The define menu for a small cancellation group should 
read:

Define a small cancellation group

******************************************************

In checking in a finitely presented group, the help
file should read:

Here are examples of how to enter a presentation:

<x,y; x^2=y^3>

<a,b,c;[a,b], b^2^c=b^-3 a b = 1>.

******************************************************

The tools menu should read:

Find the cyclic decomposition of G abelianised.
Is the presentation of G1 metric small cancellation?
__________________________________________________________

Is G trivial?
Is G of finite order?
Is G abelian?
Is G nilpotent?
Is G polycyclic?
Is G free?
Is G hyperbolic?
___________________________________________________________

Find a ShortLex automatic structure for G.
Find a rewriting system for G.
___________________________________________________________

Compute an integral homology group of G.
Compute the center of G.
___________________________________________________________

Delete selected objects and dependencies.
___________________________________________________________

Synopsis.
___________________________________________________________

***********************************************************

When a finitely presented group has been checked in, the
define menu for checking in a word should read:

Define a word in the generators of G

The associated help file should read:

Conventions on word notation:

Generator names can be any letter, such as a, b, ...,
x,y, ..  or any letter followed by a sequence of digits
such as a1, b23 and so on. In putting generators together
to form words, there must be white space between them unless
other punctuation is used. For example we allow  a (b c) in 
place of a b c. We sometimes use capital letters to denote 
inverses of generators. So the inverses of a,b,c11, .. are 
denoted by A, B, C11,... and also by a^-1, b^-1, c11^-1 and 
so on. The ^ is used to denote exponentiation. The exponents
are allowed to be group elements as well as integers, with 
a^b defined to be the conjugate B a b of a by b. Thus

a^b = B a b  = b^-1 a b .

Our convention is that ^ is left-associative. So

a^b^2 = (a^b)^2.

We use square brackets to denote commutators:

[a,b]= a^-1 b^-1 a b ( = A B a b ).

We then define, for n > 2, 

[a1,a2,...,an]= [[a1,a2],a3,...,an].

Thus multiple commutators are "left-normed".

**********************************************************

If one selects a word in a finitely presented group, the items
should be re-arranged:

Compute the formal inverse of w
Freely reduce w
Cyclically reduce w
___________________________________

Is w trivial in G?
Is w of finite order in G?
Is w of infinite order in G?
Is w a proper power?
Is w central in G?
___________________________________

Compute the centraliser of w
____________________________________

Delete selected objects and dependencies
____________________________________

Synopsis
____________________________________



***********************************************************

When one checks in a finitely presented group with a single
defining relator the message that comes up should read:

G is a one-relator group and so it has a solvable word problem.

***********************************************************

When one checks in a finitely presented group with a single
defining relator which is a proper power, the message that 
comes up should read:

G is a one-relator group with torsion and so it has a solvable
word and conjugacy problem and is also hyperbolic.

***********************************************************

If one checks in a word in a one-relator group, the word
problem window is incorrect.

Fri Jan 5, 1996, 3.50 p.m.
**********************************************************








