From bormotov@groups.sci.ccny.cuny.edu Sun Jan  7 15:44:36 1996
Return-Path: <bormotov@groups.sci.ccny.cuny.edu>
Received: from groups.sci.ccny.cuny.edu by jolly2.sci.ccny.cuny.edu (4.1/SMI-4.1-CCNY-Group Theory Cooperative)
	id AA07846; Sun, 7 Jan 96 15:44:34 EST
Received: by groups.sci.ccny.cuny.edu (4.1/SMI-4.1-CCNY-Group Theory Cooperative)
	id AA15177; Sun, 7 Jan 96 15:44:33 EST
Date: Sun, 7 Jan 96 15:44:33 EST
From: bormotov@groups.sci.ccny.cuny.edu (Dmitry Bormotov)
Message-Id: <9601072044.AA15177@groups.sci.ccny.cuny.edu>
To: bormotov@jolly2.sci.ccny.cuny.edu
Subject: gb960107.fr
Status: R

>From gilbert Sat Jan  6 18:47:27 1996
Return-Path: <gilbert>
Received: by groups.sci.ccny.cuny.edu (4.1/SMI-4.1-CCNY-Group Theory Cooperative)
	id AA14940; Sat, 6 Jan 96 18:47:26 EST
Date: Sat, 6 Jan 96 18:47:26 EST
From: gilbert (Gilbert Baumslag)
Message-Id: <9601062347.AA14940@groups.sci.ccny.cuny.edu>
To: bormotov
Status: R

Hello Dima,

Here are some nilpotent groups text changes.

Gilbert
______________________________________________________


Nilpotent groups menus
_______________________________________________________


In checking in a nilpotent group, the menu should read:

Define a nilpotent group

The help file should be replaced by:

We define a nilpotent group by first specifying its class
and then appending the finitely many additional relations 
needed for its definition. For example:

3 <a,b,c; [a,b,c]=1, [b,c]^2 = [a,b]>

defines the factor group of the free nilpotent group oi
class 3 on a,b,c obtained by adding the additional relations
[a,b,c]=1, [b,c]^2 = [a,b]. In general, given the class c,
we define a nilpotent group as a quotient of the appropriate
free nilpotent group N of class c by the relation subgroup, 
i.e., the normal closure in N of the given relators.

The log file in the view of G1, after checking in a nilpotent 
group, should read as follows:

We can compute a Mal'cev basis for the relation subgroup
of G1. This makes it possible to solve the word problem,
the order problem and to compute the center of G1 and the 
centralizers of its elements.

In computing the center of a nilpotent group, the log message
should read:

The center of G1 is generated by: .......

*****************************************************************
After checking in a nilpotent group, the tools menu should 
read:

Is G1 trivial?
Is G1 finite?
Is G1 abelian?
______________________________________________________________

Compute a Malcev basis for the the relation subgroup of G1
Compute the center of G1
Compute the abelianization of G1
_______________________________________________________________

Delete selected objects and dependencies
_______________________________________________________________

*****************************************************************
If one checks in a word in a nilpotent group, then the
tools menu should read:



Is w1 trivial in G1?
Is w1 of finite order in G1?
Is w1 of infinite order in G1?
Is w1 a proper power in G1?
___________________________________________________________

After checking in a word in a nilpotent group and using
the tool "is w1 trivial in G1?",  the nilpotent word problem
contained the item:

Compute the nilpotent structure of G1

When I did the same thing with a different word, this item
was absent. Why? In any case, it should be replaced by:

Compute a Mal'cev basis for the relation subgroup of G1

The view of G1 should report as follows:

___________________________________________________________

Compute the centraliser of w1 in G1
___________________________________________________________

Delete selected objects and dependencies
___________________________________________________________

Synopsis
___________________________________________________________

After checking in a word in a nilpotent group and using

Compute the centraliser of w1 in G1

the problem object that comes up should be headed:

Compute the centralizer of w1 in G1

The log files should read:

The centralizer of w1 is generated by: ...    

**************************************************************

When one checks in a subgroup - this holds for all subgroups 
for all of the groups checked in - the 
heading should be:

Define a subgroup of G1

The help file, except for abelian groups - should be changed 
to read as follows:

Subgroups are defined by listing their generators, which are
then enclosed by braces { ... }. These generators are represented 
by words in the generators of the containing group. Here is an
example:

{ a b^2 c, [a,b], c^a, A^3 C }

************************************************************

When one checks in a subgroup of a nilpotent the tools menu
should read as follows:

Is H1 trivial?
Is H1 finite?
Is H1 abelian?
Is H1 central?

************************************************************

After checking in two words in a nilpotent group, the tools
menu should read:

Compute the formal product w1 * w2
Compute the formal product w2 * w1
___________________________________________________________

Are w1 and w2 conjugate in G1?
Is w1 a power of w2?
Is w2 a power of w1?
___________________________________________________________

Delete selected objects and dependencies
___________________________________________________________

Synopsis
___________________________________________________________

*************************************************************

After checking in a subgroup and a word in a nilpotent group,
the tools menu should read:

Is w1 in H1?
Is a power of w1 in H1?

Etc.

**************************************************************

After checking in two subgroups of a nilpotent group, the
tools menu should read:

Is H1 contained in H2?
Is H2 contained in H1?
Is H1=H2?
_____________________________________________________________

Etc.

************************************************************


