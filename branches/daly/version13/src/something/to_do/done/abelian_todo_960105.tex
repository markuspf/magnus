From gilbert Thu Jan  4 19:27:15 1996
Return-Path: <gilbert>
Received: from lara.sci.ccny.cuny.edu ([134.74.120.10]) by jolly2.sci.ccny.cuny.edu (4.1/SMI-4.1-CCNY-Group Theory Cooperative)
	id AA01576; Thu, 4 Jan 96 19:27:15 EST
Date: Thu, 4 Jan 96 19:27:15 EST
From: gilbert ( Gilbert Baumslag)
Message-Id: <9601050027.AA01576@jolly2.sci.ccny.cuny.edu>
To: bormotov@jolly2
Status: R

%\magnification=\magstep1
\font\text=cmss10
\hsize 4.75in
\font\sm=cmssbx10
\font\med=cmssbx10 scaled \magstep1
\font\headofpage=cmss8
\font\hof=cmbx7
\def\subject{Menus}
\def\proven{\vrule height 8pt width8pt depth0pt}
\def\litem{\par\noindent
               \hangindent=\parindent\ltextindent}
\def\ltextindent#1{\hbox to \hangindent{#1\hss}\ignorespaces}
\def\sp{\vskip 20 pt}
\def\ni{\noindent}
\def\ol#1{\overline{#1}}
\def\wdt#1{\widetilde{#1}}
\def\av{\big\arrowvert}
\def\sc{\scriptstyle}
\def\sss{\scriptscriptstyle}
\def\ph{\varphi}
\def\eps{\varepsilon}
\def\rh{\varrho}
\def\r#1{{\rm #1}}
\def\la{\langle}
\def\ra{\rangle}
\def\lra{\longrightarrow}
\def\lms{\longmapsto}
\def\abs{\vskip 0 pt}
\def\b{\bigskip}
\def\m{\medskip}
\headline=%{\ifnum\pageno>\titlep
                {\vbox
 {\line
                     {\strut\headofpage\subject, Abelian groups  
\hss\folio}
                 \hrule}}
           %\fi}
\footline={\hss}
%\baselineskip=16pt
\overfullrule=0pt
\phantom{}
%\hrule
\def\titlep{1}
\def\chap{Chapter I}
\pageno=1
%\vskip 2 cm
\centerline{\med   MENUS}
%\b\b
%\hrule
\b\b

\text
\ni {\med 0. Startup}
\b\m

\ni {\tt Magnus session started at Sun Dec3 09:01:33 EST 1995 by gilbert 
on kalman at}   should be replaced by a complete sentence: 
\m\ni
Magnus session started
on Sunday, December 3, 09:01:33 EST 1995, by gilbert at kalman.
\m\ni {\bf General comment.} All sentences should begin with capital 
letters, be as complete as possible and end with a period,
unless this turns out to be inappropriate for one reason or another,
i.e., exceptions to the rule are permissible. I am only going to 
detail some of them here.

\b\b\hrule

\b\b
\ni {\med 1. Abelian groups}

\b\m\ni
{\tt 1.1. Check-in an abelian group.}

\m\ni Checked in an abelian group A1 and asked
for abelian invariants of A1. 

\m\ni
The session log reports: {\tt Let $A1=<a,b,c;a^2 b^2 c^2>$}

\ni
{\tt User defined abelian group}
\m\ni This should be replaced by: 

\m\ni User defined abelian group:
$$ A1=<a,b,c;a^2 b^2 c^2>.$$

\m\ni In my version, Magnus 1.0.0 beta, release date 9 November, 1995, 
the following message came up in the Session Log:

\m\ni{\tt Solved the word problem for A1: the presentation is
one-relator} This sentence makes no sense here, since A1 is 
not a one-relator group. This has probably been corrected.

\m\ni The tools menu, when one has checked in an abelian group,
should read as follows:

Is A1 trivial?
Find the order of A1
Find the torsion-free rank of A1
Find the order of the torsion subgroup of A1
_______________________________________________________

Find the cyclic decomposition of A1
Find the cyclic decomposition of the
p-primary components of A1
________________________________________________________

Compute the integral homology of A1
________________________________________________________

Delete selected objects and dependencies
_________________________________________________________

Synopsis
_________________________________________________________


\m\ni The heading of the associated problem window should now read:
\m\ni{ Compute the cyclic decomposition} - note that the C in
compute is now capitalised.
\m\ni
The reports in the log window of this "Compute the 
cyclic decomposition of A1" window, and  in the Session Log, if
applicable, should be corrected as follows: 

\ni Problem 1 started.  - capitalise P and add a period.

\ni   Problem 1 finished.  - capitalise P and add a period.

\m\ni In the workspace the skull and cross-bones should
be replaced, if possible, by a picture of Magnus; otherwise
by a check-mark, or simply omitted altogether if nothing
suitable can be found. The skull and cross-bones is inappropriate.

\m\ni In the session log, the report should read:

\ni  Found the cyclic decomposition of A1: $Z^2$.  
(Note the capital F and the period.)

...

\ni Problem finished.

\m\ni
In typing in an abelian group presentation:
$<a,b,c; a^2 b^2 c$\^{}$>$,
the carrot on the last c in the relator should have a 2 
accompanying it, and so {\tt Problem with check-in} 
signalled an error. It should read:

\m\ni Expected ...  here. (E is capitalised, ends with period.)

\m\ni The accompanying help window  explains possible 
ways of entering a presentation, one of which is 

\m\ni $<a,b;[a,b]>$ - since the group was presented as an abelian group
and since the instructions for presenting an abelian group expressly
note  that it is unnecessary to put in commutativity relators,
either this way of entering a presentation should be  be 
omitted or else this help file, and perhaps the others as well, should
be specially tailored to the task at hand (I favour the latter
solution, but it may be too difficult).
\b\m\ni
{\tt 1.2. Check-in a word (in an abelian group).}

\m\ni Checked in a word in an abelian group A1. 

\m\ni The help menu is entirely inappropriate for checking
in a word in an abelian group. I suggest suppressing it for
the moment.

\m\ni The tools menu, once we have checked in a word in
an abelian group, should read as follows:

Reduce w1 to its abelian form
Find the inverse of w1, in abelian form
________________________________________________

Is w1 trivial?
Find the order of w1
Is an element of infinite order a proper power?
Find the p-height of an element of p-power order
Is the subgroup generated by w1 pure?
_________________________________________________

Find the canonical decomposition of w1
_________________________________________________

Delete selected objects and dependencies
_________________________________________________

Synopsis
_________________________________________________

In the tools menu for a word in an abelian group, the first
item should read:

\m\ni Find the canonical decompostion of w1

\m\ni In the "view" of a given word w1, the message should read
something like this, if this does not provoke additional work:

\ni We use additive notation,  denoting the generators of
the infinite cyclic summands in the canonical decomposition
of an abelian group, by $f1, f2, \dots ,$, and those 
of the finite cyclic summands, by $t1, t2, \dots , $.
The canonical decomposition of  w1 is $w1 = 3 f1 + f2$ .
These canonical generators are related to the given generators
as follows:

f1= ....
f2= ...
etc. 

Notice that we have switched back to multiplicative
notation again.


\m\ni The session log should report: The canonical decomposition of
w1 is $3 f1 + f2$ .

\m\ni In answer to the question as to whether
a word in an abelian group is trivial, the 
Object View Log (and the Session Log) should report:

\m\ni The canonical decomposition of w1 shows that it is non-trivial.

\m\ni In answer to the question as to whether two abelian groups
A1 and A2 are isomorphic. the session should report: A1 and A2 
are isomorphic because they have the same invariants.

{\tt Check-in two words in an abelian group}
Compute the product w1 w2, in abelian form
______________________________________________________________

Is w1 = w2?
Is w1 a proper power of w2?
Is w2 a proper power of w1?
______________________________________________________________

Delete selected objects and dependencies
______________________________________________________________

Synopsis
_______________________________________________________________

{\tt 1.3. Check-in a subgroup (of an abelian group).}

\m\ni  In checking in a subgroup of an abelian group, the help
file in the dialog box should read as follows:
\m\ni
Enter the generators of the subgroup as words in the generators of
the supergroup, separating them by commas and enclosing the list
by {}s. For example, in the case of generators of a subgroup of an
abelian group with generators a,b,c,
$$\{ a^2 b^3, C^7, b^4 c^7\, B a^2}.$$
(For more information about our notation for words, see the help file
{\it words}.

\m\ni The log messages after check-in for a subgroup should be:
User defined subgroup 
$$H1 = gp(a^5, a b c, c^4)$$
of A1.
\m\ni

{\tt The menu items for a subgroup of an abelian group should
be expanded as follows:}

Is H1 trivial?
Find the order of H1
Find the torsion-free rank of H1
Find the order of the torsion subgroup of H1
_______________________________________________________

Find the cyclic decomposition of H1

________________________________________________________

Is H1=A1?
Find the index of H1 in A1
Find a virtual complement of H1 in A1
Is H1 pure in A1?
________________________________________________________

Delete selected objects and dependencies
_________________________________________________________

Synopsis
_________________________________________________________

{\tt If an element and a subgroup of an abelian group have
been checked in, the tools menu should  be enlarged with:


Is w1 in H1?
Is a power of w1 in H1?
(In both cases we need to express w1 as a product
of the given generators of H1.)

.................................................

{\tt If two subgroups of an abelian group have been checked
in, the tools menu should have the following additional items:

Does H1 contain H2?
Does H2 contain H1?
Is H1=H2?
____________________________________________________________

Compute the join of H1 and H2
Compute the intersection of H1 and H2
______________________________________________________________

{\tt 1.3. Check-in a map (of an abelian group).}
\m\ni There are no checks or warnings in the check-in procedures
for maps. The system readily accepts incomplete definitions.
Perhaps the heading {\tt Define a map on A1} should
read 

\m\ni Define a map from A1 to A1

\m\ni in order to allow for additional possibilities
later. Having defined this map (keep in mind our
discussion of yesterday, which menas certain tools will have
to be deleted), we need a menu item:

\m\ni Does m1 extend to a endomorphism

\m\ni Notice the use of the word endomorphism. 
If it does then additional tools should be provided,
such as composition of endomorphisms and homomorphisms, 
provided the ranges and domains match, check to see if
an endomorphism is an automorphism, check to see if
an automorphism is inner, computation of the inverse of
an automorphism, check to see if an automorphism
is of finite order, facility to compute image of an
element under an endomorphism as well as an homomorphism,
check to see if a homomorphism is onto, if a homomorphism
is an isomorphism, computation of the inverse of an 
isomorphism. Of course these things do not have to be
implemented now, unless it is very easy and very fast, i.e.,
one day. 

\m\ni After checking in a word w1 the image of w1 under
a homomorphism should be reported as:

\m\ni $ w2 = a^4$, the image of w1 under the endomorphism m1
(remove the word ``Let'').

\m\ni
{\tt Homology.} In the case of one-relator groups
$$G =<X; r^e>,$$
where $r$ is not a proper power and $e \geq 1$,  the integral
homology groups take the following form:

\item {1.} if e=1, then
\itemitem {a.} $H_n(G)=0$ if $n>2$; 
 \itemitem {b.} $H_2(G)=0$ if $r$ is not in the
derived group of the ambient free group;
\itemitem {c.} $H_2(G)=Z$ if $r$ is in the derived group of the ambient free group.
\item {2.} if $e > 1$, then
\itemitem {a.} $H_{2n-1}(G)=Z_e$ for $n>1$
\itemitem {b.} $H_{2n}(G)=0 $ for $n>1$
\itemitem {c.} $H_2(G)=0$ if $r$ is not in the derived group
of the ambient free group.
\itemitem {d.} $H_2(G)=Z_e$ if $r$ is in the derived group
of the ambient free group.
\par

\m\ni
Additionally, for one-relator groups, it is appropriate
to report that a one-relator group is finite if and only if
it comes as a one-generator group with the relator a power.

*****************************************************************************
\m\ni
I want the help files to be chosen appropriately, i.e., for
example the abelian groups help file will apply directly to
abelian groups, etc.
\m\ni
When defining an abelian group, the help message should
read as follows:
\m\ni
There is no need, in defining an abelian group, to include
any commutativity relations (or relators) - the system
automatically supplies such relations (relators). So, for
example, <s,t> is a presentation of the free abvelian
group of rank two, while <s,t; s^2=1> presents the direct 
product of a group of order two with an infinite cyclic group. 
Explicitly adding some commuting relations does no harm. So 
the free abelian group of rank two can also be presented, when 
checked in as an abelian group, in the form <s,t; [s,t]>.
\m\ni
I checked in A2 = <x,y;[x,y]> and Magnus reported that it had
solved the word problem for A2, because the presentation was
one-relator. This message has to be removed.
\m\ni
Having checked in an abelian group A2, I invoked the tool for
computing the cyclic decomposition for A2. The report should
read:

The cyclic decomposition for A2 is:

$$A2 = Z + Z.$$

Many problems about A2, such as the order problem, the word
problem and so on, can be solved using this decomposition,
 

\m\ni 
It is better, for the most part, not to put anything into the GIC 
unless it is explicitly asked for.

\m\ni
When checking in a word in the generators of an abelian group,
the help file is inappropriate. It should read: The usual notation
for words can be used in working with abelian groups. There is no
need to recast them into any particular form - the system will
take care of everything.


\m\ni
The tools menu for a word in an abelian group should read:

Find the canonical decomposition of w1

Is w1 trivial in A1




\m\ni


\m\ni
\end



----- End Included Message -----



