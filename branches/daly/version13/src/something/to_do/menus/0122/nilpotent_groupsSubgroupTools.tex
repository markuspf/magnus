\noindent{\Large\bf Nilpotent groups: subgroup tools}

\bigskip
\hrule\hrule\hrule

\begin{enumerate}

\item Is H1=G1?

\item Is H1 of finite index in G1?

\item Is H1 isolated in G1?

\item Is H1 central G1?

\item Is H1 normal in G1?

\item Is H1 subnormal in G1??

\item Is H1 characteristic?

\item Is H1 fully invariant?


\bigskip
\hrule\hrule\hrule

\item Compute the index of H1 in G1

\item Compute a Schreier transversal of H1 in G1

\item Compute the isolator of H1 in G1

\bigskip
\hrule\hrule\hrule

\item Is H1 trivial?

\item Is H1 finite?

\item Is H1 abelian?

\item Is H1 free nilpotent?

\item Is H1 automatic?

\item Is H1 hyperbolic?

\bigskip
\hrule\hrule\hrule

\item Compute the order of H1

\item Compute the center of H1

\item Compute the torsion subgroup of H1

\item Compute the canonical decomposition of H1 abelianized

\item Compute the class of H1

\item Compute the derived length of H1

\item Compute the lower central series of H1

\item Compute the quotient of two successive terms of the lower central
series of H1

\item Compute the derived series of H1

\item Compute the quotient of two successive terms of the derived series
of H1

\item Compute the Hirsch number of H1

\item Compute a Malcev basis for H1

\item Compute the first n terms for a short-lex system of representatives for the
elements of H1

\item Compute an integral homology group of H1

\bigskip
\hrule\hrule\hrule

\item Find a polycyclic presentation of H1

\bigskip
\hrule\hrule\hrule

\item Find a permutation representation of G1
modulo H1 when the index of H1 in G1 is finite

\item Find a finite rewriting system for H1

\bigskip
\hrule\hrule\hrule


\item Delete selected objects and dependencies

\bigskip
\hrule\hrule\hrule

\item
Synopsis

\bigskip
\hrule\hrule\hrule

\end{enumerate}
