\noindent{\Large\bf One-relator groups: subgroup tools}

\bigskip
\hrule\hrule\hrule

\begin{enumerate}

\item Is H1=G1?

\item Is H1 of finite index in G1?

\item Is H1 isolated in G1?

\item Is H1 central in G1?

\item Is H1 normal in G1?

\item Is H1 subnormal in G1?

\item Is H1 characteristic in G1?

\item Is H1 fully invariant G1?

\bigskip
\hrule\hrule\hrule

\item Compute the index of H1 in G1

\item Compute a Schreier transversal of H1 in G1

\bigskip
\hrule\hrule\hrule

\item Is H1 trivial?

\item Is H1 abelian?

\item Is H1 nilpotent?

\item Is H1 polycyclic?

\item Is H1 solvable?

\item Is H1 perfect?

\item Is H1 abelian?

\item Is H1 finite?

\item Is H1 finitely related?

\item Is H1 automatic?

\item Is H1 hyperbolic?

\item Is H1 free?


\bigskip
\hrule\hrule\hrule


\item Compute the order of H1

\item Compute the center of H1

\item Compute the canonical decomposition of H1 abelianized

\item Compute a nilpotent quotient of H1

\bigskip
\hrule\hrule\hrule

\item Find a permutation representation of G1 modulo H1 when the index
of H1 in G1 is finite

\bigskip
\hrule\hrule\hrule

\item Enumerate relators of H1

\bigskip
\hrule\hrule\hrule

\item Delete selected objects and dependencies

\bigskip
\hrule\hrule\hrule

\item
Synopsis

\bigskip
\hrule\hrule\hrule

\end{enumerate}
