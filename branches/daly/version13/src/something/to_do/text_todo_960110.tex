From gilbert Wed Jan 10 16:02:28 1996
Date: Wed, 10 Jan 96 16:02:28 EST
From: gilbert ( Gilbert Baumslag)
To: bormotov
Content-Length: 3473

The message when Magnus starts up should read as follows:

This is a  beta release of Magnus, version 1.0.0. Release
date, Februay 11, 1996. Copyright (C) 1995 The New York 
Group Theory Cooperative. Magnus comes with NO WARRANTY 
WHATSOEVER. 

This software is free. You are welcome to redistribute
it under certain conditions - choose `Show GNU general public
license' in the `Help' menu for information about this version
of Magnus.

(The rest remains the same.)

************************************************************

The list of contributors should include:

E. Schneider

************************************************************

In the Magnus Manual, the date at the beginning and at the
end should be changed to 1/96.

************************************************************

In the Magnus Manual, the heading should be:

Editing text in Magnus


The text should be change slightly as follows:

Many windows in Magnus have text in them
which you can edit. Position the cursor in
the appropriate place, click with the left
mouse button and begin typing. If nothing
happens, the text is read-only (like this
text).

For tasks such as cut-copy-paste, you can use
any combination of the three methods below:

1. {\it the mouse}

2. {\it Macintosh-style edit menus}

3. {\it Emacs keychords}

***************************************************
In the Magnus Manual, Editing with the Mouse should
read as follows:

Editing with the mouse

You can only copy and paste with the mouse.
First select the text that you want to copy
with the left mouse button, either by dragging
it across the desired text or by double or 
triple clicking. Then click the left mouse 
button at the point where you want to paste
the selected text and click the middle button
to do the pasting.

*****************************************************

The GNU GENERAL PUBLIC LICENSE should be repositioned
on the page. In my version it looks terrible. But it
may be alright as it stands.

*******************************************************

Release Notes should be Release notes

Here is the new version of the Release notes:

This version of Magnus, beta Version 1.0.0, is still
in the process of development. You might want to look
at the section on our {\it Future plans} to see what we
have in mind for the future.  Although we have fixed
a number of bugs, some still remain. We apologise, in
advance, for any problems that you might be faced with.

You can always get the current version of Magnus, compiled for a Sun Workstation, from

zebra.sci.ccny.cuny.edu

in the directory  /pub/Magnus

by anonymous ftp. (Remember to set the file type to binary.)

You will get a gzipped, tarred version. Simply gunzip and extract
the files as usual.

The README files will explain how to install Magnus.

If you have difficulty installing Magnus, the following commands will help to pinpoint some of the problems.

% uname -a

Look for a string like "SunOS <machine name> 4.1.x <other stuff>". "SunOS" and "4.1" are the important parts.

% wish

If you get "wish: Command not found.", you may not have Tcl/Tk!

% info tclversion
 
This should say "7.3"

% puts $tk_version

This should say "3.6"

% info commands addinput

This should say "addinput"

% exit   (this gets you out of the "wish" program)

If you want the source code for Magnus, email 

rgr@groups.sci.ccny.cuny.edu.

Your feedback will help us to improve Magnus and will
be very much appreciated (see {\it  Contacting the 
authors} for how to send your comments).









