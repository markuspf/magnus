%% This is AMSLaTeX source.

\title{Abelian Groups Specification}
\author{The New York Group Theory Cooperative}
\date{27 November, 1995}

\documentstyle[11pt]{article}

%% some measurements to suit american and european paper:
%\parindent0pt
\textwidth17cm
\oddsidemargin-.7cm
\topmargin-1cm
\textheight22cm


\newcommand{\update}[1]{\marginpar{\footnotesize #1}}

\newcommand{\magnus}{{\sc magnus}}
\newcommand{\lskip}{\vspace{10pt}}


\newcommand{\subsubsubsection}[1]{\vspace{3pt}{\noindent\bf #1}\vspace{3pt}}


\newcommand{\wish}[3]{
\item{\em #1}\\
{\sc input}: #2\\
{\sc output}: #3
}

\newcommand{\wishtwo}[4]{
\item{\em #1}\\
{\sc input}: #2\\
{\sc primary output}: #3\\
{\sc secondary output}: #4
}

\newcommand{\wishthree}[5]{
\item{\em #1}\\
{\sc input}: #2\\
{\sc primary output}: #3\\
{\sc secondary output}: #4\\
{\sc tertiary output}: #5
}


\newcommand{\tick}{\makebox[1pt][r]{\raisebox{4pt}{$\surd$}}}
\newcommand{\notick}{\makebox[1pt]{}}
\newcommand{\square}{$\Box$}
\newcommand{\who}[1]{\makebox[2.0cm][l]{#1}}
\newcommand{\when}[1]{\makebox[1.7cm]{#1}}
\newcommand{\ckitem}[4]{\item\square#1\who{\kern1ex#2}\when{#3}\parbox[t]{12cm}{#4}}
%\newcommand{\subckitem}[4]{\item\square#1\who{\kern1ex#2}\when{#3}\parbox[t]{9.5cm}{#4}}
\newcommand{\subckitem}[4]{\item\square#1\who{\kern1ex#2}\parbox[t]{9.5cm}{#4}}


\newcommand{\done}{$\Box$\makebox[1pt][r]{\raisebox{4pt}{$\surd$}}}
\newcommand{\nearly}{$\Box$}
\newcommand{\planned}{$\heartsuit$}
\newcommand{\capa}[2]{\item#2\kern2ex#1}


\newcommand{\FE}{{\sf FE}}
\newcommand{\SM}{{\sf SM}}
\newcommand{\CM}{{\sf CM}}
\newcommand{\CMs}{{\sf CMs}}
\newcommand{\GCM}{{\sf GCM}}
\newcommand{\GIC}{{\sf GIC}}


\begin{document}

\maketitle


%%--------------------------------------------------------------------
\section{Introduction}

Let

$$\langle x_1,\dots, x_n\;;\; R \rangle$$

\noindent be an abelian presentation of an abelian group $G$. Then $G$ has an
abelian presentation of the form

$$\langle f_1,\dots, f_r, t_1,\dots, t_k\;;\; t_1^{p_1},\dots, t_k^{p_k}\rangle$$

\noindent where $0\le r\le n$, $0\le k\le n$, $r+k\le n$, and $p_i\mid
p_{i+1}$ for $1\le i<k$. Let $H$ be the group given by this
presentation, and let $\alpha:G\to H$ be the isomorphism given by
$x_i\mapsto w_i(\tilde{f},\tilde{t})$ for $1\le i\le n$.

\lskip

The minimum data needed to represent $G$ efficiently is $r$, the
vector $\hat{p}$, and the maps $\alpha$, $\alpha^{-1}$.

These data are found by the familiar integer matrix computation. It
can happen in practice that, even given `nice' input with a `nice
answer', the computation produces very large integers, and may exhaust
the resources of the machine or the patience of the user for this
reason.

Therefore we would like to extract partial information from a
computation which might fail or be too expensive. These may include
whether $G$ is finite, the torsion-free rank of $G$, and possibly {\em
any} cyclic decomposition of $G$ (not necessarily a canonical
one). The first two can be done as separate computations over the
rationals, for especially bad $G$.


%%--------------------------------------------------------------------
\section{Implementation}

We will need classes to encapsulate the computations of the Smith
normal form and the torsion-free rank over the rationals. We will need
classes to represent abelian group structure, words
$w(\tilde{f},\tilde{t})$, and subgroups defined by sets of such words.

%%....................................................................
\subsection{Class {\tt AbelianWord}}

A word $w(\tilde{f},\tilde{t})$ could be represented internally as a
{\tt VectorOf<Integer>} of length $r+k$. We should not use a `raw'
{\tt VectorOf}, for these reasons:

\begin{itemize}

\item
A {\tt VectorOf<Integer>} could be anything; the compiler cannot
detect cases where an inappropriate {\tt VectorOf<Integer>} is
mistakenly `interpreted' as a word.

\item
{\tt VectorOf} has public members which should be hidden.

\item
We may wish to allow for transparently compressed words, e.g., for
$k>10^3$ the word $2t_{692}-t_5$ could be represented by the vector
$[5,-1,692,2]$.

\end{itemize}

Therefore we derive a class {\tt AbelianWord} (private or protected)
from {\tt VectorOf}, and provide {\tt Word}-like members as
needed. This class could also have heuristics for automatically
compressing the {\tt VectorOf} representation.


%%....................................................................
\subsection{The Maps $\alpha$, $\alpha^{-1}$}

The maps $\alpha$ and $\alpha^{-1}$ can be thought of as inherently
part of the abelian group structure, so therefore should have no
formal existence outside of it. Thus it is adequate to represent
$\alpha^{-1}$ as a {\tt VectorOf<Word>}, and $\alpha$ as a {\tt
VectorOf<AbelianWord>}.


%%....................................................................
\subsection{Class {\tt Abelianization}}

The utility class {\tt Abelianization} represents the Smith normal
form computation. It has these public members:

\begin{enumerate}

\item
A constructor {\tt Abelianization(const FPGroup)}.

\item
The standard computation members

\begin{itemize}
\item
{\tt void startComputation( )}
\item
{\tt bool continueComputation( )}
\item
{\tt bool done( ) const}
\end{itemize}

\item
Accessors for partial results

\begin{itemize}
\item
{\tt Trichotomy knownToBeInfinite( ) const}
\item
{\tt bool haveTorsionFreeRank( ) const}

{\tt int getTorsionFreeRank( ) const}
\end{itemize}

\item
Accessors for final results

\begin{itemize}
\item
{\tt VectorOf<Integer> invariants( ) const}
\item
{\tt VectorOf<AbelianWord> getIsomorphism( ) const}
\item
{\tt VectorOf<Word> getInverseIsomorphism( ) const}
\end{itemize}

\end{enumerate}

All accessors throw an error when they are called prematurely.

We would like to be able to reuse this in cases where, e.g., we
already have a integer matrix, and want it's Smith normal form. This
is complicated by the fact that $\alpha$, $\alpha^{-1}$ may not be
necessary. Consider how to factor this functionality out.

%%....................................................................
\subsection{Class {\tt AbelianTorsionFreeRank}}

This is just a partial version of {\tt Abelianization}, which works
over the rationals.


%%....................................................................
\subsection{Class {\tt AbelianGroup}}

The utility class {\tt AbelianGroup} implicitly represents the data
$G$, $H$, $\alpha$.  It has these public members:

\begin{enumerate}

\item
A constructor {\tt AbelianGroup(const Abelianization\&)}.

This extracts $r$, $\hat{p}$, $\alpha$, and $\alpha^{-1}$ from the
{\tt Abelianization}.

\item
{\tt int torsionFreeRank( ) const}

\item
{\tt VectorOf<Integer> invariants( ) const}

\item
{\tt Integer order( ) const}

\item
{\tt AbelianWord reinterpret(const Word) const}

\item
{\tt Word reinterpret(const AbelianWord) const}

\item
{\tt Word maximalRoot(const Word) const}

\item
{\tt void printBasisOn( ostream\& )}

Outputs something like $f_i\mapsto u_i(\tilde{x})$, $t_j\mapsto v_j(\tilde{x})$.

\end{enumerate}


%%....................................................................
\subsection{Class {\tt AbelianSubgroup}}

Class {\tt AbelianSubgroup} could derive publicly from class {\tt
Subgroup}. It could have additional helping members, such as an {\tt
Abelianization*} and {\tt AbelianGroup*} to represent $G/N$, where $N$
is the subgroup.



%%--------------------------------------------------------------------
\section{End User Functionality}

The following are to be reflected in front end menu items. All of them
require data (e.g., the {\tt AbelianGroup}) which may not have been
computed yet. In this case, they must present problem views.


%%....................................................................
\subsection{Elements}

\begin{enumerate}

\ckitem{\notick}{}{}{
Word problem.

Use {\tt AbelianGroup::reinterpret}.
}

\ckitem{\notick}{}{}{
Equality problem.

Use {\tt AbelianGroup::reinterpret}.
}

\ckitem{\notick}{}{}{
Power problem.

Use {\tt AbelianGroup::maximalRoot}.
}

\ckitem{\notick}{}{}{
Maximal root.

Use {\tt AbelianGroup::maximalRoot}.
}

\ckitem{\notick}{}{}{
Rewrite $u(\tilde{x})$ as $v(\tilde{f},\tilde{t})$.

Use {\tt AbelianGroup::reinterpret}.
}

\ckitem{\notick}{}{}{
Rewrite $u(\tilde{f},\tilde{t})$ as $v(\tilde{x})$.

Use {\tt AbelianGroup::reinterpret}.
}

\end{enumerate}


%%....................................................................
\subsection{Elements in Subgroups}

\begin{enumerate}

\ckitem{\notick}{}{}{
Subgroup membership problem.

Possibly map into $G/N$.
}

\ckitem{\notick}{}{}{
Power in a subgroup.

Possibly map into $G/N$, and do some division.
}

\ckitem{\notick}{}{}{
Rewriting an element of a subgroup in the generators of the subgroup.

(Need details)
}

\end{enumerate}


%%....................................................................
\subsection{Subgroups}

Subgroups may be generated by words $u(\tilde{x})$ or by words
$v(\tilde{f},\tilde{t})$, but not both kinds.

\begin{enumerate}

\ckitem{\notick}{}{}{
Subgroup containment problem.

This is an iterated subgroup membership problem.
}

\ckitem{\notick}{}{}{
Subgroup equality problem.

This is two containment problems.
}

\ckitem{\notick}{}{}{
Index of a subgroup.

The order of $G/N$.
}

\ckitem{\notick}{}{}{
Intersection.

(Need details)
}

\ckitem{\notick}{}{}{
Join.

Trivial.
}

\ckitem{\notick}{}{}{
Does the subgroup coincide with the group?

Index = 1.
}

\ckitem{\notick}{}{}{
Enumerate relations of the subgroup.

(What is meant? Do anything special for finite index?)
}

\ckitem{\notick}{}{}{
Find a virtual free abelian complement of the subgroup.

(Need details)
}

\ckitem{\notick}{}{}{
Find a basis for the group compatible with the one for the subgroup.

(Need details)
}

\end{enumerate}


%%....................................................................
\subsection{The Group}

\begin{enumerate}

\ckitem{\notick}{}{}{
The torsion-free rank.

Output {\tt AbelianGroup::torsionFreeRank()}.
}

\ckitem{\notick}{}{}{
The invariants.

Output {\tt AbelianGroup::invariants()}.
}

\ckitem{\notick}{}{}{
A basis.

Call {\tt AbelianGroup::printBasisOn(ostream\&)}.
}

\ckitem{\notick}{}{}{
Is the group free?

Iff {\tt AbelianGroup::invariants().length() == 0}.
}

\ckitem{\notick}{}{}{
The order problem.

By means of selectively running {\tt AbelianTorsionFreeRank} and/or
{\tt Abelianization}.
}

\ckitem{\notick}{}{}{
The isomorphism problem.

By means of selectively running {\tt AbelianTorsionFreeRank} and/or
{\tt Abelianization}.

We may consider allowing for {\tt Abelianization} to be told to
`pause' when it has its relation matrix in Hermite normal form (as
defined by Sims). Or, a compressed form of the Hermite normal form
could be part of an {\tt AbelianGroup}, for use in difficult
isomorphism tests.
}

\ckitem{\notick}{}{}{
The integral homology.

(Need formula)
}

\end{enumerate}



%%....................................................................
\subsection{I/O}

A word $w(\tilde{f},\tilde{t})$ may be written by the end user in the
usual form, such as ``{\tt 0}'', ``{\tt f\_2 - 3 t\_1}'', etc.

Therefore, we may as well have a parser for `integer linear
combination of `named things' which are indexed by non-zero integers
(the $f$'s and $t$'s, in this case)'. Then {\tt AbelianGroup} can have
members to read such words, and (ordered) sets of them.


\end{document}
