\documentclass[12pt]{article}

\usepackage{latexsym}
\usepackage{amsfonts}
\usepackage{epsfig}

\parindent=0pt
\parskip=8pt
\def\magnus{MAGNUS }
\def\nsmagnus{MAGNUS}

\begin{document}


\section{Adding packages}

\nsmagnus' Package Mechanism was developed to provide a simple and
convenient way of integrating second party software into \nsmagnus'
graphical user interface.
Integrated packages communicate with \magnus by writing simple
messages to  the standard output and reading messages from the
standard input.

Packages are required to  understand simplified
descriptions (representations) of algebraic  objects accepted in
\magnus software package. The formats existing in \magnus were
slightly changed in order to simplify parsing of the descriptions
of objects sent to a package from \nsmagnus.
For example, many kinds of brackets are eliminated when sent to a
package so that one can read definitions symbol by symbol.

The following is a description of conventions used to represent 
different objects in \magnus and in messages as well: 

\textbf{ Algebraic objects:}
\begin{enumerate}
\item \textbf{Word}:
There are a number of conventions concerning word notation.
Generator names can be any letter, such as 
\verb+a, b, ..., x, y, z+
or any letter followed by a sequence of digits such as 
\verb+a1, b23 +
and so on. In putting generators together to form words, there
must be white space between them unless other punctuation is
used. The \verb+^+ is used to denote exponentiation. By default, the
exponents are integers only.

Example: \verb+a b^2 c c d^-4 a^3+

{\em Note}: When sending a word definition to \magnus the
following additional conventions are accepted: we allow \verb+a (b c)+
in place of \verb+ a b c+. We sometimes use capital letters to denote
inverses of generators. So the inverses of \verb+a, b, c11, ..+ are
denoted by \verb+A, B, C11,...+ and also by \verb+a^-1, b^-1, c11^-1 +and so
on.  The exponents are allowed to be group elements as well as
integers, with \verb+a^b+ is defined to be the conjugate \verb+B a b+ of 
\verb+a+ by \verb+b+.
Thus \verb+ a^b = B a b = b^-1 a b+.

Our convention is that \verb+^+ is left-associative. So 
\verb+ a^b^2 = (a^b)^2+.

We use square brackets to denote commutators:\\ 
\verb+    [a,b]= a^-1 b^-1 a b = A B a b+.\\
We then define, for $n > 2$, \\
\verb+    [a1,a2,...,an]= [[a1,a2],a3,...,an]+.\\
Thus multiple commutators are "left-normed".

\item \textbf{Tuple} :
Words in a tuple are separated  with
commas and the list is enclosed in \{\}'s. For
example:

\verb+{a, a, a^2, b^a, a, b^2}+.


\item \textbf{Subgroup} : 
A subgroups is described by listing its generators, each such
generator given by a word in the generators of the supergroup. We
separate these words by commas. For example, the subgroup of a
group with generators \verb+a+ and \verb+b+ generated 
by \verb+a^2 b+ and \verb+[a,b]+ is
described by

\verb+    { a^2 b, a^-1 b^-1 a b }+


\item \textbf{ Map}:
To define a map from the generators
\verb+{ x, y, ... }+
of a group G to a group H with generators
\verb+{ a, b, ... }+
assign to each of \verb+x, y, ...+ its image as a word in the generators
of $H$, enclosing the description by \verb+{}+'s. For example:
\begin{verbatim}
{
  x -> a^b,
  y -> b^-1
}
\end{verbatim}
If you omit the image of a generator, Magnus assigns its image to
be the identity.

\item \textbf{Group} : 
We use customary notation in describing a group by means of a
presentation : $< x_1, ..., x_n ; r_1, ..., r_n>$, where $x_i$ -
generators of a group ant $r_i$ - relators.

In case of a free group, symbol ``;'' is ommited. Presentation of
a nilpotent group starts with a nilpotency class, followed by a
presentation in the form described above.

Here are some examples of how to enter a presentation:

\begin{verbatim}
< x, y; x^2  y^3, x y x y >

<a,b; a^-1 b^-1 a b >

< b, t ; t b^2 t^-1 = b b b >

< a,b,c,d,e>

4 <a,b; a b c >
\end{verbatim}
\end{enumerate}


Detailed description of the messages' format together with
examples of their use are given below.

\textbf{How to add a package}

When \magnus is started, you will see ``Packages" item in the main
menu. It has two sub-items: ``Add a package" and ``Edit the list of packages".

When ``Add a package" menu is selected a Dialog Window (Figure
\ref{fig:addwin}) will pop-up. This is the window where all
information required to add a package is entered. There are 5
fields that must be filled in:


\begin{figure}[htbp]
%\centerline{\epsfig{file=addwindow.eps}}
\caption{``Add a new  package''  dialog window.}
\label{fig:addwin}
\end{figure}


\begin{enumerate}
\item \emph{Enter project name}: enter a name of the package. This
string will appear as an item in the menu. There is a limit of
20 characters.
\item \emph{Command line}: enter the command which will invoke
the package. You can type it in the text window or select an
executable from the "Open File" dialog window by clicking
"Browse" button. This field is limited to 128 characters.
\item \emph{Select checkin type}: select the type of a group
your package will work with. Together with selected algebraic
objects, this field defines conditions for packages menu item to
appear (see below for details).
\item \emph{Select object}: select algebraic objects
that your package will be applied to. When you  highlight
selected in this field objects which also belong to a group of the
type selected in the previous field the menu item for you package
will appear in the ``Packages" menu. Then you can run the package
by selecting corresponding menu item.
\item \emph{Parameters}: You can define a list of parameters that
can be edited and sent to the package before it is started. Press
``Parameters" button for the ``List of parameters" dialog window. It is
empty at the beginning. You can add a new parameter by pressing
``Add" button. Two fields have to be filled: the {\em name} of the parameter
which will appear as a parameter description in the package's
problem window, and the parameter {\em type} (integer, text or boolean).
\end{enumerate}


The example in Figure \ref{fig:addwin} shows package named 
\verb+Coset Enumerator (PEACE)+ 
which is invoked with \verb+peace -h+ command (here we assume that 
\verb+peace+ lies in the search path and so that the full path name is not 
needed). 
This package is applied to a ``Finitely Presented" group


\textbf{Messages description}: 
Usually communication between Magnus and an integrated package 
starts by sending a description of selected objects and values
of parameters (if necessary )  from Magnus to the package.
After that, the package sends its output to Magnus using messages as
specified below. Notice that if no special command precedes the output
string, it will be shown in the linked file and its link is automatically 
called ``Click here to see the details from \emph{package\_name}''.


The following is a list of messages currently available for
interchange between external packages and Magnus together with examples of
their use.

\textbf{From Magnus to a package}:

The following describes messages Magnus can send to a package.
All messages of this type are of the following format:
\begin{quote}\em
object\_description ( oid\_number )
\end{quote}
where \emph{object\_description} is a string which describes the
object (group presentation, word definition and so on ) as
described above. \emph{oid\_number} is a unique integer
identifier, automatically assigned to each object created in the
Magnus workspace. One can use \emph{oid\_number} to
refer to a particular object, when sending messages to Magnus
Session Manager (see below).

\begin{enumerate}
\item \textbf{Group}: \magnus sends a presentation of a group which 
the package should read: \\
 \emph{ group\_presentation ( oid ) }.\\
Example:
\begin{verbatim}
<a,b,c,d ; a^2, b c > (3)
\end{verbatim}
\item \textbf{Nilpotent Group}: \\
\emph{ nilpotency\_class group\_presentation   ( oid ) } \\
Example:
\begin{verbatim}
5 <a,b,c,d ; a^2, b c > (3)
\end{verbatim}
\item \textbf{Word}:  \\
 \emph{ group\_presentation ( oid ) } \\
\emph{  word ; ( oid ) } \\
Example:
\begin{verbatim}
<a,b,c,d ; a^2, b c > (3)
a^-1 b^-1 a b; (12)
\end{verbatim}
\item \textbf{Subgroup} : \\
 \emph{ group\_presentation ( oid ) } \\
\emph{  subgroup\_generators ; ( oid ) } \\
Example:
\begin{verbatim}
<a,b,c,d ; a^2, b c > (3)
{a^-1 b^-1 a b, a^2 c^2}; (12)
\end{verbatim}
\item \textbf{Tuple} : \\
 \emph{ group\_presentation ( oid ) } \\
\emph{  words\_in\_tuple ; ( oid ) } \\
Example:
\begin{verbatim}
<a,b,c,d ; a^2, b c > (3)
{a^-1 b^-1 a b, a^2 c^2}; (12)
\end{verbatim}
\item \textbf{Map} : \\
 \emph{ domain\_group\_presentation ( oid ) } \\
 \emph{ range\_group\_presentation ( oid ) } \\
\emph{  map  ( oid ) } \\
Example:
\begin{verbatim}
<a,b,c,d ; a^2, b c > (3)
<a,b,c,d ; a^2, b c > (3)
{ a->b, b->c, c->d, d->a } (15)
\end{verbatim}
\item \textbf{Homomorphism} : \\
 \emph{ domain\_group\_presentation ( oid ) } \\
 \emph{ range\_group\_presentation ( oid ) } \\
\emph{  homomorphism ; ( oid ) } \\
Example:
\begin{verbatim}
<a,b,c > (0)
<x,y,z > (3)
{ a->x, b->y, c->z } (15)
\end{verbatim}
\item \textbf{Two words (w1,w2)} :  \\
 \emph{ group\_presentation ( oid ) } \\
\emph{  word\_w1 ; ( oid ) } \\
\emph{  word\_w2 ; ( oid ) } \\
Example:
\begin{verbatim}
<a,b,c,d ; a^2, b c > (3)
a^-1 b^-1 a b; (12)
a^-1; (11)
\end{verbatim}
\item \textbf{Two subgroups (H1,H2)} :  \\
 \emph{ group\_presentation ( oid ) } \\
\emph{  subgroup\_H1 ; ( oid ) } \\
\emph{  subgroup\_H2 ; ( oid ) } \\
Example:
\begin{verbatim}
<a,b,c,d ; a^2, b c > (3)
{a^-1 b^-1,  a b }; (12)
{a^-1, b}; (11)
\end{verbatim}
\item \textbf{Word and Subgroup (w,H)} :  \\
 \emph{ group\_presentation ( oid ) } \\
\emph{  subgroup\_H ; ( oid ) } \\
\emph{  word\_w ; ( oid ) } \\
Example:
\begin{verbatim}
<a,b,c,d ; a^2, b c > (3)
{a^-1 b^-1,  a b }; (12)
a^-1 (11)
\end{verbatim}
\item \textbf{Parameters} : \\
\emph{Integer ( name ) (value) } \\
\emph{String ( name ) (value) } \\
\emph{Boolean ( name ) (value) } \\
Example:
\begin{verbatim}
Integer (Maximal number of elements) (1000)
Boolean (Enumerate all) (1)
\end{verbatim}
\end{enumerate}

\textbf{ From Package to Magnus }
\begin{enumerate}
\item Output text into session log : \\
\emph{Text \{ text \} } \\
Example:
\begin{verbatim}
. . .
printf(``Text { Package has started its work } \n'');
. . .
\end{verbatim}

\item Create a link to a file : \\
\emph{Link \{ link\_name , file\_name \} } \\
Example:
\begin{verbatim}
. . .
printf(``Link { Click to see some details , /tmp/detail_file } \n'');
. . .
\end{verbatim}
\item Create a group object in the Workspace : \\
\emph{Create Group \{ group\_presentation \} } \\
Example:
\begin{verbatim}
. . .
printf(``Create Group { <a,b,c,d ; a^2, b^3 c a > } \n'');
. . .
\end{verbatim}
\item Create a word object for a group in the Workspace : \\
\emph{Create Word \{ word $|$ group\_oid \} } \\
Example:
\begin{verbatim}
. . .
printf(``Create Word { a B c b | 3 } \n'');
. . .
\end{verbatim}
\item Create a subgroup in the Workspace : \\
\emph{Create Subgroup \{ list\_of\_subgroup\_generators $|$ group\_oid \} }\\
Example:
\begin{verbatim}
. . .
printf(``Create Subgroup { a b, b c , a d, c^7 | 5 } \n'');
. . .
\end{verbatim}
\item Create a tuple in the Workspace : \\
\emph{Create Tuple \{ words\_in\_tuple $|$ group\_oid \} }\\
Example:
\begin{verbatim}
. . .
printf(``Create Tuple { a b, b c , a d, c^7 | 5 } \n'');
. . .
\end{verbatim}
\item Create a map : \\
\emph{Create Map \{ map $|$ domain\_group\_oid $|$ range\_group\_oid \} }\\
Example:
\begin{verbatim}
. . .
printf(``Create Map { a - > x, b->y^2 | 8 | 14 } \n'');
. . .
\end{verbatim}
\end{enumerate}


Here is an example of a simple package which is written in Perl.
It takes a word from Magnus and returns its $n$-th power where
$n$ is a parameter to be specified.

First the package is added to the ``Packages" menu in \nsmagnus. The name,
command to invoke the package, the type and the fact that an integer
parameter is required are all entered via a dialog window.

To use the package, check-in a finitely presented group and a word.
With the word highlighted, select the package from the menu. Enter
the power as a parameter via the menu in the ``Start" dialog window 
that appears.

Here is the Perl script which returns the appropriate results:

\begin{verbatim}
#!/usr/bin/perl

#
#  Expect 3 lines on STDIN as follows:
#
# <a, b, c ; a^2, b^3 c^-4 >  (5)       # presentation and its oid
# a^-1 b^-1 a^2 b^3;  (12)              # word with its oid
# Integer (power) (17)                  # power to be raised
#

$lineone = <STDIN>;
@parts = split(/\D+/, $lineone);        # split line into numbers
$groupoid = $parts[$#parts];            # oid = the last number

$linetwo = <STDIN>;
@parts = split(/;/,$linetwo);           # split line at semi-colon ;
$theword = $parts[0];                   # initial part is the word

$linethree = <STDIN>;
@parts = split(/\D+/,$linethree);       # split line into numbers
$power = $parts[$#parts];               # power = the last number

print "Text {About to compute the ${power}-th power of $theword } \n";

$powerstr = "";

for ( $i = 0; $i < $power; $i++ ){
   $powerstr = $powerstr." $theword";
}

print "Create Word {$powerstr | $groupoid } \n";
\end{verbatim}




\section{Adding new algorithms to the Tools Menu}


  \magnus consists of two major parts: the Front End (FE) and  the Back End (BE). 
  The FE is implemented in Tcl/Tk . It is responsible for representing information 
  (obtained from the BE) on the screen and transferring the input from the user to 
  the BE. 
  Almost all transactions of the Front End are controlled by a Back End process
  called the 
  Session Manager. Developers  that are not involved in changing of \nsmagnus' GUI design 
  usually do not have to work with the Front End. A more detailed description of the 
  communications between the Front End and the Back End can be found in Magnus 
  Reference Manual 1. The goal of this text is to show the steps that are
  needed to add new  algorithms into \nsmagnus.

  When adding a new algorithm to Magnus  one has to work with the Back End 
  class structure  which  is implemented  in standard C++.
  Source code for the \nsmagnus' Back End is located in magnus/back\_end/ directory 
  which consists of directories containing class' descriptions. 



Usually it is obvious from the directory name where to look for a particular 
class or a group of classes. Each directory has the following structure: 
  \begin{itemize}
  \item include/ - directory containing the definition files;
  \item src/ - contains implementation files;
  \item test/ -  source code for the local executables which, after the 
  compilation,
   usually can be found in test/bin.
  \item Makefile - makefile for the current directory.
  \end{itemize}

 
  \subsection{Generalities about menus} Menus in \magnus are contextual
in the sense that their contents depend on which items are selected
in the Workspace window.  If no icons are selected, then only the Check-in
menu for new objects and a few default items are available. If one or more
items are selected, then a variety of relevant tools become available.
For instance if a subgroup H1 and a word w1 are selected, the tools menu 
will contain a number of algorithms for answer questions about their
relationship, for instance ``Is w1 an delement of H1?".


  In most cases we can predict the running time of a problem in advance.
  We say that a problem is ``fast''  if it runs approximately less than a
  second on a modern computer. Usually these problems are those that
  can be run in constant time or linear time where upper bound on the 
  input size is known. When we cannot assure fast execution of the algorithm
  than the we should create a {\em Problem Window} with a list of {\em computation
  managers} (see below) which is used to distribute the machine resources.


     The Magnus Session Manager should be able to require information from 
  the front end i.e. names of the selected objects, their relations
  and so on.  Additionally the Magnus Front End must be able to reflect the state 
  of the Session Manager.  Therefore, we predefine a small number of events, 
  in response to
   which the front end is to take some action which reflects the session manager's
   state. The session manager describes to the front end what to do in each event, 
  via a simple message  protocol. 
  Most of such events  are for static properties of algebraic objects which
    are needed to determine whether a menu item, etc., should be posted.

  The following is a list  of classes which implement interaction events
  available from the EData objects:
  \begin{enumerate}
  \item  \emph{struct Text}. Used in session manager to deliver a text message
  to the front end.
  \item \emph{ struct CheckinType}. Correspond to a ``type'' of the group in the 
  front end.
  Usually used to set up conditions in the front end. For example:
  \footnotesize
  \begin{verbatim}
   M.startItemGroup( CheckinType( 1 ) == SMFPGroup::ABELIAN );
  \end{verbatim}
  \normalsize
  Here \emph{  CheckinType( 1 ) == SMFPGroup::ABELIAN} is true in the front end if 
  the
  ``type'' of the first selected in the Workspace ``group'' object is Abelian.
  \item \emph{ struct IsHomo}. True in the front end if the  parameter object is a 
  homomorphism.
  \item \emph{ struct IsIso}. True in the front end if the  parameter object is an
  isomorphism.
  \item \emph{ struct IsAuto}. True in the front end if the  parameter object is 
  an
  automorphism.
  \item \emph{ struct Parent}. Detects the parent for a given object.
  \item \emph{ struct Domain}. Gives the domain group of a given object.
  \item \emph{struct Range}. Gives the range group of a given object.
  \item \emph{struct Oid}. Gives  unique object ID for a given object.
  \item \emph{ struct Name}. Gives a name in the Workspace for corresponding 
  object.
  \item \emph{struct Link}. A special kind of a text message. Allows to create a
  link to the specified file in  the Workspace logs.
  \end{enumerate}

  Events are send by means of a simple messages. Most of the messages are derived
  from the ostream. To send a message a special method \emph{void send()} has to
  be called.  The following are most useful for the user messages available
  in session manager:
  \begin{enumerate}
  \item \emph{class LogMessage : public OutMessage, protected FEData, public 
  ostrstream}.
  Sends a message to the specified log window in the front end.
  This is the most commonly used type of the messages. All reports, links to the 
  files
  and so on are delivered by means of this message.
  \item \emph{class Warning : public OutMessage, protected FEData, public 
  ostrstream}.
  Creates a warning window with the specified text.

  \item \emph{class Message : public OutMessage, protected FEData, public 
  ostrstream}. Posts
  a message in the separate dialog window.
  \end{enumerate}

  Example:

  \footnotesize
  \begin{verbatim}
  LogMessage msg( *this, theWord );
  msg << Name( theWord ) << " is not trivial.";
  msg.send();
  // Print in the main log window and the log window corresponding 
  // to ``theWord''  the following message: ``w1 is not trivial.''

  LogMessage msg1(*this,theGroup);
  msg1 << Link(Chars( "Click here to see the canonical decomposition of ")+
                     Text(Name( theGroup )),
                    "CyclicDecomposition",
                    ``can_decomp.txt'' );
  msg1.send();   
  // Create a link to the file can_decomp.txt in the main log window 
  // and the log window for ``theGroup''
  // Text of the link: "Click here to see the canonical decomposition of G1''

  \end{verbatim}
  \normalsize

  \subsection{Creating a menu item}
  Even though  fast and slow problems  involve different methods of 
  executing the algorithms, the corresponding menu items are created in the 
  uniform fashion.
   
  There is a group of template classes which are used to create menus.
  Menu itself is described by one of the classes inherited from \emph{class Menu} 
  (it can be
  found in magnus/back\_end/SessionManager/include/Menu.C).
  Here is a short description of its methods:
  \begin{itemize}
   
  \item \emph{ enum MenuKind \{ CHECKIN, TOOLS, MAKE, PACKAGES, DATABASE, TESTING 
  \}} - 
          Main menu items' types. 
    
  \item \emph{  void startItemGroup( ) } - Start an item group with no condition.

   
  \item \emph{ void startItemGroup(const Expression\& condition)} - Start an item 
  group with condition.
    Every item in the menu that has been added in this item group
      will appear in the menu only if "condition" is true 


  \item \emph{  void startCascade(const Text\& text)} - Start cascade menu

  \item \emph{  void closeCascade( ) }- Close cascade menu

  \item \emph{ void addDisabled(const Text\& text)} - Add a disabled (grayed) menu 
  item

  \item \emph{  void separator( ) } -  Add a separator in the menu

  \item \emph{  void done( )} - Close current item group
  \end{itemize}

  From  \emph{Menu} we derive classes of the form:
  \begin{center}
  \emph{template $<$ list of classes $>$ class Menu\textbf{N} : public Menu}
  \end{center}
  where \emph{N} is the number of objects selected in the Workspace when this
  menu is activated and  \emph{list of classes} is a list of parameter classes 
  which correspond to the selected objects in the Workspace. 

  In addition to the inherited methods, all \emph{Menu\textbf{N}} classes
  have  
  \begin{center}
  \emph{void add(const CtorArgs\textbf{N} $<$list of classes$>$\& c, const Text\& 
  t, const Action\&a ) }
  \end{center}
  which actually adds an item to the menu and associates menu with a class
  which will execute the corresponding action. Here, $t$ is a text in the menu,
  $a$ some additional action, usually left as default and $c$ is an instance of
  one of the  special classes \emph{Ctor\textbf{N}} - a
   `constructor wrapper', i.e., an appropriate derivative of class Ctor.
   These arrange, in cooperation with class ObjectSmith, for a callback
   when the menu item is selected. 

  Definition of \emph{Ctor\textbf{N} } is somehow similar to one for 
  \emph{Menu\textbf{N} }:
  \begin{center}
  \emph{template $<$ T, list of object classes $>$ class Ctor\textbf{N} : public 
  CtorArgs\textbf{N}
   $<$list of object classes$>$}. 
  \end{center}
  Here, again, \emph{N} is the number of objects
   selected in the Workspace, \emph{list of classes} - types of corresponding 
  objects and \emph{T}  is a class (usually derived from \emph{Supervisor} or 
  \emph{ComputationManager} classes) which actually defines the action (see 
  below).


  Examples of defining menu items are shown below:
  \scriptsize
  \begin{verbatim}
    {
      Menu1<SMFPGroup> M( Menu::TOOLS );
      // Create a menu object which correspond to one
      // selected ``group'' object in the Workspace and 
      // ``Tools'' item of the main menu


      M.startItemGroup( CheckinType( 1 ) == SMFPGroup::ABELIAN );
      // Start group for Abelian groups only

      M.startCascade( Text("Is") + Name(1) + Text("...") );
      // Start a cascade menu ``Is A1 ...''

      M.add( Ctor1<IsTrivialProblem, SMFPGroup>(),
             Text("Is") + Name(1) + Text("trivial ?")
             );
      // Insert item ``Is A1 trivial?'' and assign IsTrivialProblem class
      // to perform an action when this menu item is selected

      M.add( Ctor1<IsFiniteProblem, SMFPGroup>(),
             Text("Is") + Name(1) + Text("finite ?")
             );
      // Insert item ``Is A1 finite?'' and assign IsFiniteProblem class
      // to perform an action when this menu item is selected

      M.closeCascade();
      // We done with current cascade menu

      M.startCascade( Text("Compute ...") );
      // Start a new cascade menu ``Compute ...''
      
      M.add( Ctor1<AbelianOrderProblem, SMFPGroup>(),
             Text("Compute the order of") + Name(1) );
      // Insert item ``Compute the order of A1?'' and assign
      // AbelianOrderProblem class
      // to perform an action when this menu item is selected

      .   .   .

      M.closeCascade();
      
      M.done()
      // Done with the  menu object
   }

   {
     Menu2<SMWord,SMWord> M( Menu::TOOLS );
     // Create a menu object which correspond to two
     // selected ``word objects in the Workspace and 
     // ``Tools'' item of the main menu

      M.startItemGroup( Parent( 1 ) == Parent( 2 ) &&
                        CheckinType( Parent(1) ) == SMFPGroup::ABELIAN
                        );
      // Condition - both words correspond to the same ``group'' object
      // and this group is abelian
      
      M.add( Ctor2<ProductInAbelianForm, SMWord, SMWord>(),
             Text("Compute the product") + Name(1) + Name(2)
             + Text(", in abelian form")
             );
      // Insert item ``Compute the product w1w2 in the abelian form?'' 
      // and assign ProductInAbelianForm class
      // to compute the product

      ...

      M.startItemGroup( Parent( 1 ) == Parent( 2 ) &&
                        CheckinType( Parent(1) ) != SMFPGroup::ABELIAN
                        );

     // Condition - both words correspond to the same ``group'' object
      // and this group is NOT abelian
      
      M.add( Ctor2<FormalProductOfWords,SMWord,SMWord>(),
             Text("Compute the product") + Name(1) + " * " + Name(2)
             );
      
      M.separator();
      // Put a separator line in the menu
      ...

      M.done()
   }
          
  \end{verbatim}
  \normalsize

  \subsection{ Adding fast problems to the \nsmagnus' menu}
  Now we know how modify menu items. Next step is to create actions
  invoked from the menu. We start with the simplest case when the action
  is almost instantaneous (so called ``Fast Problem''). 

  All ``fast problems'' classes are derived from \emph{FastComputation} class
  which is derived from \emph{ComputationManager}. 
  User would need to provide a constructor, which  receives
  the object to which an action is applied, and override a special function
  \emph{ void takeControl( )} in which we actually implement the invoked action.
  This function is called by session manager every time the corresponding 
  menu item is selected.

  After a derived class is implemented we need to add a corresponding menu 
  item by changing magnus/back\_end/SMApps/src/menuDefns.C.
  First, find a proper Menu object and call \emph{add()} method to 
  add the problem as was described above.

  An example below shows  source code for adding an item 
  to  solve the word problem in a free group.

  \scriptsize
  \begin{verbatim}

  // --------------------- menuDefns.C -------------------------- //
  // the following text occurs in the menuDefns.C:
   M.add( Ctor1<WordProblemInFree,SMWord>(),
             Text("Is") + Name(1) + "trivial ?" 
             );



  //Part of  fastProblems.h corresponding to the word problem in a free group
  // --------------------- WordProblemInFree -------------------- //

  // the following class implements an action invoked by
  // selecting ``Is w1 trivial?'' menu item for a free group, i.e.
  // our action is to receive a word and solve word problem for 
  // this word in the parent free group

  class WordProblemInFree : public FastComputation
  {
  public:

    // Receive the reference to the  word and save it
    WordProblemInFree(const class SMWord& w) : theWord( w ) { }

    // implement the action
    void takeControl( ) //overrides ComputationManager
    {
          Word u, v;

          // get the parent free group
          FreeGroup F = theWord.getParent().getFreePreimage();

          // Create log message
          LogMessage msg( theWord );
          msg << Name( theWord );

          // if given word is trivial
          if ( F.wordProblem( theWord.getWord() ) ) {
             msg << " is trivial";
          } else {
             msg << " is not trivial";
          }

          // finish and send the message
          msg << " in " << Name( Parent( theWord ) ) << ".";
          msg.send();
    }
    
  private:

    const class SMWord& theWord;
  };
  \end{verbatim}
  \normalsize
 
  \subsection{Adding a problem with ``Problem Window'' to \nsmagnus' menu}

  In general problems we dealing with in \magnus are more complicated than
  fast problems. Usually we can't say 
  in advance when we'll get the answer (sometimes we can't even say if we'll get
   the answer at all). Some problems could be solved in different ways and it is 
  often 
  not clear which method is better. All this requires a smarter managing system. 
  \magnus
  faces these problems by running algorithms in parallel and allowing the user to 
  distribute resources between the algorithms. 

  Brielfy, a process of solving a problem can be described in the 
  following steps. When user selects an item from the menu a corresponding
  Supervisor object (inhereted from the Supervisor class) is created and
  a Problem Window will come up.
  After user clicks ``Start'' button, the \emph{takeControl()} methods of the 
  Supervisor
  object and Computation Managers (defined in the \emph{viewStructure()} method of
  the Supervisor) are  executed one by one. Usually each of them spends one time
  unit (arc) and gives control tho the next Computation Manager. This process 
  continues until the termination conditions are met. These  conditions are:
  either all Computation Managers (including Supervisor) have been stoped and 
  \emph{adminTerminate()} method was called, or user terminated them by clicking
  ``Terminate'' button of the problem window.  


  Let's see what the steps of adding a problem  by following an example (notice 
  that
  the source code was slitely changed to make example easier).
  Consider the problem of computing Abelian Invariants (Cyclic Decomposition) for 
  the
   Abelian group.  Interface and implementation for  Abelian Groups class can be 
  found in:
  magnus/back\_end/Groups/. There are two files: include/AbelianGroup.h and 
   src/AbelianGroup.C.  There are many different methods  in AbelianGroup but we 
  are
   interested in those  which correspond to computing of the Cyclic Decomposition. 
  In AbelianGroup.h we will find definitions for   two methods:   
  \emph{computeCyclicDecomposition()} and  \emph{invariants()}.
  First does the actual computations and the second returns the result 
  (invariants) as a
   vector of integer numbers.  
  Example below shows  a segment of the source code from AvelianGroup.h

  \footnotesize
  \begin{verbatim}
  // --------------- AbelianGroup --------------- //

  class AbelianGroup : public ObjectOf<AbelianGroupRep>
  {

  public:

   . . .

   void computeCyclicDecomposition( ) {
      change()->computeCyclicDecomposition();
    }

   . . . 
   
    VectorOf<Integer> invariants( ) const {
      return look()->invariants();
    }

  }
  \end{verbatim}
  \normalsize


          Now the question is how to make  \emph{computeCyclicDecomposition()}  
  available from
   the menu. This would involve writing of a couple of classes which are 
  correspond to a
   certain standard and most of them could be created by copying and changing 
   lines from similar classes.
   Usually these classes are  defined in files which are located in 
  magnus/back\_end/SMApps
   (Session Manager Applications). Classes for our problem defined and implemented 
  in
   magnus/back\_end/SMApps/include/AbelianInvariants.h and 
  magnus/back\_end/SMApps/src/AbelianInvariants.C.

          The first type of classes to be created are those that inherited from 
  the class
   called ARCer (located in magnus/back\_end/SessionManager/...). Its purpose is 
  to control
   the execution of a child process that will do actual computations and to obtain 
  the
   result for the corresponding ComputationManager (see below). 
  This is the class that will call our
   \emph{AbelianGroup::computeCyclicDecomposition()}.  Definition of such class 
  from
  AbelianInvariants.h is shown below. 
  
   \scriptsize
  \begin{verbatim}
  // ---------------------- AbelianInvariantsARCer --------------------------- //

  class AbelianInvariantsARCer : public ARCer
  {
  public:
    ////////////////////////////////////////////////////////////////////////
    // Constructors:                                                      //
    ////////////////////////////////////////////////////////////////////////

    AbelianInvariantsARCer( ComputationManager& boss ) 
      : ARCer( boss ), AG( 0 ) { } 

    ~AbelianInvariantsARCer( ) { delete AG; } 

    /////////////////////////////////////////////////////////////////////////
    // Accessors:                                                          //
    /////////////////////////////////////////////////////////////////////////
    // set parameters. Usually used before calling takeControl()
    void setArguments( const FPGroup&, bool );

    // Inherited from ARCer:
    // bool takeControl( );  
    // when called for the first time, forks a child process and starts 
  computation 
    // returns true when  finished with computations, returns false if one
    // unit of time (arc) was used but algortihm has not finished yet.

    AbelianGroup getCyclicDecomposition();
    // returns Abelian group with computed decomposition  

    /////////////////////////////////////////////////////////////////////////
    // IPC interaction members:                                            //
    /////////////////////////////////////////////////////////////////////////

    void runComputation( ); 
    // called from takeControl() after a child process started
    // all time consuming computations should go here 

    void writeResults( ostream& ); 
    // write data that should be available for the SessionManager
    //  to a stream. Executed in the child process after  runComputation( )
    // is finished and before the child process quits. 
    
    void readResults( istream& );  
    // read data from a stream Called from the parent process. Used to 
    // read results saved by writeResults(...)

  private:

    /////////////////////////////////////////////////////////////////////////
    // Data Members:                                                       //
    /////////////////////////////////////////////////////////////////////////
    FPGroup G;  
    bool bMakeFile;
    AbelianGroup* AG;
  };
  \end{verbatim}
  \normalsize

 Implementation is quite simple and idea
   is usually the same. The folowing is shows the implementation of methods 
  for 
  AbelianInvariantsARCer. It should be clear from the  comments what the purpose 
  of
  each method in the class. 


  \scriptsize
  \begin{verbatim}
  // ---------------------- AbelianInvariantsARCer --------------------------- //

  void AbelianInvariantsARCer::setArguments( const FPGroup& group,   bool makeFile 
  )
  // set/reset arguments before running the computation 
  {
    delete AG;
    AG = new AbelianGroup( group, makeFile );
  }

  void AbelianInvariantsARCer::runComputation( )
  // this function will be called from the child process
  {
  #if SAFETY > 0
    if( AG == 0 )
      error( "void AbelianInvariantsARCer::runComputation( ) : "
             "Have to initialize the group first");
  #endif

    AG->computeCyclicDecomposition(); // do the computations
  }


  void AbelianInvariantsARCer::writeResults( ostream& out ) 
  // write the result  from the child's process (here AbelianGroup
  //  with computed invariants) after the  computation is finished 
  {
    out < *AG;
  }


  void AbelianInvariantsARCer::readResults( istream& in )
  // read the result by the parent process after the child process quit
  {
    in > *AG;
  }
  \end{verbatim}
  \normalsize


          Next type of classes are  ComputationManagers. These classes inherited 
  from 
  ComputationManager class which  is located in magnus/back\_end/SessionManager/.
  Objects of such classes provide communication between SessionManager and child 
  processes
   forked from an ARCer object. 
  For example, it is ComputationManager's responsibility to send
   start, stop, suspend or terminate commands to the corresponding ARCer object. 
  ComputationManager receives the amount of time (number of arcs) that the child 
  process
   allowed to work before the control will be given to another ComputationManager 
  (and 
  therefore another child process) and makes sure that the child process uses this 
  time
   correctly. SessionManager receives from ComputationManager information to be 
  represented
   on the screen or to be saved in Information Centers.
          
   
  Code below shows the definition and implementation of 
  the main methods of the ComputationManager for computing Abelian Invariants.
   Definitions of ComputationManagers usually very simple and do not vary much.

  \scriptsize
  \begin{verbatim}
  // Declarations in AbelinaInvariants.h
  // -------------------------- AbelianInvariants ---------------------------- //

  class AbelianInvariants : public ComputationManager
  {
  public:

    /////////////////////////////////////////////////////////////////////////
    // Constructors:                                                      //
    /////////////////////////////////////////////////////////////////////////

    AbelianInvariants(class AbelianInvariantsProb&);
   
         *    *    * 

  private:

    /////////////////////////////////////////////////////////////////////////
    // Data Members:                                                   //
    /////////////////////////////////////////////////////////////////////////

    class AbelianInvariantsProb& theSupervisor; // parent supervisor (see below)
    
    AbelianInvariantsARCer arcer; // ARCer object
  };
  \end{verbatim}
  \normalsize

  \scriptsize
  \begin{verbatim}
  // Implementation in AbelinaInvariants.C
  // -------------------------- AbelianInvariants ---------------------------- //

  AbelianInvariants::AbelianInvariants(AbelianInvariantsProb& prob)
    : theSupervisor( prob ),    
      arcer( *this )
  {
    // initialize ARCer parameters
    arcer.setArguments( theSupervisor.getSMFPGroup().getFPGroup(), true ); 

    adminStart();  // start computation manager
  }


  void AbelianInvariants::takeControl( )
  {
    if ( freeARCs() > 0 )    // if there are free arcs (time units) try to do 
  computations 

      // When called for the first time, arcer.takeControl() forks a child process
      //  and starts computation, returns true if finished with computations
      // false if one arc is used but computation has not been finished
      if( arcer.takeControl() ) { 
       
        // put cyclic decomposition into Group information center 
        theSupervisor.getSMFPGroup().gic.putHaveCyclicDecomposition    
          ( arcer.getCyclicDecomposition() );                          
        
        adminTerminate();   // terminate the computation manager 
      }
      else
        usedOneARC();  //  tell SessionManager  that one ARC was used
  }

  \end{verbatim}
  \normalsize

  There is one type of ComputationManagers that needs special attention. 
  They are called Supervisors. The difference between Supervisor and regular 
  computation
   manager is that instead of managing directly child processes started from 
  ARCers, 
  supervisor manages a number of other Computation Managers. In addition 
  Supervisor has
   a method called viewStructure(...) which defines the corresponding Problem 
  Window which
  user actually sees on the screen.

  \begin{figure}[h]
   %\centerline{ \epsfig{file=OrderProbWin.eps} }
    \centerline{ Missing EPS file }
  \caption{An example of the problem window.}
  \label{OPWin}
  \end{figure}                             

   Figure \ref{OPWin} shows the Problem  Window for the "Find the order of a 
  group" 
  problem. Regular Computation Managers correspond to the lines with algorithm 
  names and
   little scroll bars on the right side. The Supervisor for this problem 
  correspond to
   the whole window. It is his responsibility to distribute free arcs (time units)
   between the  Computation Managers, define title and text of the window and so 
  on. 
  Implementation of the methods for AbelianInvariants Problem Supervisor is given 
  below

  \scriptsize
  \begin{verbatim}
  // ------------------------- AbelInvariantsProb ---------------------------- //
  AbelInvariantsProb::AbelInvariantsProb(SMFPGroup& G)
    : Supervisor( true ),
           theGroup( G ),
           // AbelianInvariants Computation Manager:
           abelianInvariants( *this )  
  {
    // check if group information center has the decomposition
    // already. If it does than set the fast result flag
    if( theGroup.gic.haveCyclicDecomposition() )  
       resultIsFast();                             
  }

  void AbelInvariantsProb::takeControl( )
  {
    // if result is fast just print the decomposition on the screen and terminate
    if( fastResult() ) {   
      LogMessage( theGroup.gic.getCyclicDecompositionMessage() ).send();
      adminTerminate();
      return;
    }

    // if our computation manager stoped
    if ( theGroup.gcm().abelianInvariants->state() == TERMINATED ) { 
      LogMessage( *this, theGroup.gic.getCyclicDecompositionMessage(), 2 ).send(); 
      // put the answer on the screen
      adminTerminate();
    }
  }

  void AbelInvariantsProb::viewStructure(ostream& ostr) const
  {
    char *temp;
    if( theGroup.getCheckinType() == SMFPGroup::ABELIAN )
      temp = "";
    else 
      temp = " abelianized";
      
    // define the problem window
    ProblemView pv( ostr, oid(), 
                   "Compute the canonical decomposition",  // title of the window
                   Text("Compute the canonical decomposition of")//text below the 
  title
                    + Name( theGroup ) + Text(temp),
                    helpID( "AbelInvariantsProb", theGroup ), // help file id
                    "Can dec" // text that will appear below the icon in
                              // the workspace  which correspond
                              // to this problem window
                  );                    
    pv.startItemGroup();  

    // insert computation manager: title-"For the use of this problem"
    // time unit resources ID - abelianInvariants.arcSlotID()
    // default number of arcs - 100
    pv.add("For the use of this problem", abelianInvariants.arcSlotID(), 100); 
    pv.done();
  }

  \end{verbatim}
  \normalsize

  The  only thing that left is to  make  our problem available from the menu.  
  As in the case of a fast problem  we need to change  
  magnus/back\_end/SMApps/src/menuDefns.C.
  First, find a Menu which corresponds to the Tools Menu for Abelian
  groups and call \emph{add()} method with a proper Ctor object as was described 
  above.
  The corresponding part of menuDefns.C with comments  is shown below

  \footnotesize
  \begin{verbatim}

   Menu1<SMFPGroup> M( Menu::TOOLS ); // create menu object for one selected 
  object of type
                                      // SMFPGroup  in the workspace and 
                                      // with items located in the Tool menu 

    . . .

     M.startItemGroup( CheckinType( 1 ) == SMFPGroup::ABELIAN ); 
     // start group of items which correspond
     // to abelian groups
   
    . . .

     M.add( Ctor1<AbelInvariantsProb,SMFPGroup>(),  
            Text("Compute the canonical decomposition of") + Name(1) );    

    // add an item, called Compute the canonical 
    // decomposition of G, where G is the
    // name of the group object in the
    // workspace and create an object of
    // AbellianInvariantsProb class when 
    // this item selected from the menu 

    . . .
    
    M.done();

  \end{verbatim}
  \normalsize

  After all these steps and compilation of \nsmagnus' SessionManager 
  AbelianInvariants Problem should be available from  the menu.

\subsection{Adding Parameters}

Sometimes it is usefull to have additional user defined parameters 
passed to the algorithms. It is possible to create a problem window
with  a predefined set of parameters which could be changed during 
the execution of the algorithm. Parameters could be changed only at 
the idle state of the problem (i.e. before it started or when the 
\textbf{<Suspend>} button  was pressed). 

When creating a problem window with parameters, a slightly different
procedure used to define window itself. Such windows defined by creating
obects of class \emph{EnumeratorProblemView} instead of usual \emph{ProblemView}.
Similar to what has been described above, we create an object of 
class \emph{EnumeratorProblemView} inside the body of 
\emph{Supervisor::viewStructure(ostream ostr) const}. Bars in the window, 
corresponding to omputation managers are
added by calling \emph{EnumeratorProblemView::addProblem(const Subproblem)}
method. To add parameters one will have to call 
\emph{EnumeratorProblemView::addParameter(const ViewContent)}.
Here \emph{ViewContent} is a parent class for all classes which define
 parameter objects. For now the following classes are available
for defining parameters in the problem window:
\begin{itemize}
\item \emph{class EditInteger}. Creates a field for entering an integer
number. Constructor:\\
 \emph{EditInteger( const Chars name,const Chars text,int change = 0,int def = 0, int minV = 0,int maxV = MAXINT)} \\
Here \emph{name} - unique name (ID) of the parameter that will be used later for
referencing , \emph{text} - text that will appear infront of the
edit field,\emph{change} - if true will alow to change the parameter
after the problem was started, \emph{def} - default value,
 \emph{minV} - minimal value, \emph{maxV} - maximal value.
\item \emph{class EditBool}. Creates a check box which if checked corespond
to the truth value of the parameter. Constructor: \\
\emph{ EditBool( const Chars name,const Chars text,int change = 0,int def = 0)} \\
Again, \emph{name} - unique referencing ID, \emph{text} - corresponding text in the
window, \emph{change} - if true will allow to change parameter after
the problem has started, \emph{def} - default value (true, false).
\item \emph{class EditText}. Creates a text edit field. Constructor: \\
\emph{EditText( const Chars name,const Chars text,int change = 0,Chars def = "")}\\
Parameters have the same meanings as above
\item \emph{class EditWord.} Creates a field where a word in geerators
of a given free group can be entered. Constructor: \\
\emph{EditWord( const Chars name,const Chars text,int change = 0,const FreeGroup f = FreeGroup(), const Word w = Word() )} \\
\emph{name}, \emph{text} and \emph{change} same as above. Free group \emph{f}
defines a set of generators (actually their names) used to parse the 
entered word. \emph{w} - default word.

\end{itemize}

Parameters above could be grouped together. There are three  classes 
which allow you to group parameters:
\begin{itemize}
\item \emph{class  GroupWindow}. Defines a group
object. Parameters are added through a sequence of calls of 
\emph{GroupWindow::add(const ViewContent)} method.
\item \emph{class RadioButton.} Creates a radio buton object in the
window. Usually used in groups to select something from a number of choices.
Constructor:\\
\emph{ RadioButton(const Chars name, const Chars text, int change = 0)} \\
Class is ihereted form \emph{GroupWindow}.
Parameters grouped by the object of this
class wll appear on the screen only when the corresponding
radio button is selected. If no parameters added to the group then
object behaves as a regular radio button.
\item \emph{class  RadioButtonGroup}. This class
is inhereted from \emph{GroupWindow} and groups a number
of objects of the \emph{RadioButton} class.
\end{itemize}

Current values of parameters for a particular window can be obtained
by executing  \emph{Supervisor::getParameters()} method
of the corresponding \emph{Supervisor} object.
It returns a Parameter Structure which keeps all parameters together
with their current values. Value for  a particular parameter can
be obtained through \emph{ParameterStructure::getValue(const Char name)}
method, where \emph{name} is a unique reference ID of the parameter.


When a problem window apears on the screen, user can open a  parameter
window by clicking  on the \textbf{<Parameters>} menu item in the problem window.

To read values of the modified parameters, we will usually need to 
overload \emph{start()} and \emph{resume()} virtual methods of the
\emph{ComputationManager} class. These methods are called right after
the  \textbf{<Start>}(\textbf{<Resume>}) buttons of the problem window
were selected. It will allow us to check values of 
modified parameters and update problem's settings when the process executing 
the algorithm is idle.

Example of the supervisor which corresponds to the problem window with
parameters is given below

  \footnotesize
  \begin{verbatim}
// ------------------- SupervisorWithParameters ------------------- //

void SupervisorWithParameters::takeControl( )
{
 	// some code here

     .  .  .

}


void SupervisorWithParameters::viewStructure(ostream& ostr) const
{
  
   EnumertatorProblemView pv(oid(),"Supervisor with parameters",
			     Text("Select to define some parameters) ,
			     helpID("SP",theGroup),
			      Text("sp")
			     );

   // add an integer parameter, name ``I1'' with default
   // value 10. This parameter can be changed after the resume
   // button is selected
   pv.addParameter(EditInteger("I1","Integer parameter",1,10,0,100));

   // reate a radio button  object 
   RadioButton RB1("RB1","Radio buttom 1");
   RadioButton RB2("RB2","Radio button 2");

   // add integer parameters to each radio button above
   RB1.add(EditInteger("RB1_I1","Integer for radio button 1",0,10,1,10000));
   RB2.add(EditInteger("RB2_I1","Integer for radio button 2",0,10,1,10000));

   // group radio buttons
   RadioButtonGroup rbGroup("rbGroup");
   rbGroup.add(RB1);
   rbGroup.add(RB2);
      
   // add group to the parameters
   pv.addParameter(rbGroup);


   // add a computation manager
   pv.addProblem( Subproblem("Sp1",ThisARCSlotID(),
			     "For the use of this problem",40));
   
  
  // submit configuration to the Session Manager
   submit(ostr, pv);

}

// this method is called when the Start button of the
// corresponding window is pressed but before actual 
// computation is started
void SupervisorWithParameters::start( ) {              
	
  // Get parameters:
  // first integer
  int I1 = getParameters().getValue("I1");
       .  .  . // do something with it


  // get which of the radio buttons was selected  
  Chars t = getParameters().getValue("rbGroup");

   int buttonInteger;
  // assign to buttonInteger value of the integer
  // parameter for the selected button
  if (t == Chars("RB1")){
    	buttonInteger = getParameters().getValue("RB1_I1");
    } else {
    	buttonInteger = getParameters().getValue("RB2_I1");
    }
	.  .  . // do something here
  
}

void WordEnumeratorProblem::resume( )
{              
   // only first integer parameter was defined in a such
   // way that it is possible to change its value after 
   // the Suspend button is pressed.

   int I1 = getParameters().getValue("I1");
     .  .  .
}


  \end{verbatim}
  \normalsize

\appendix
\section{A fast problem example}
Let $H$ be a subgroup and $w$ be a word in some free group $F$.
Suppose we want to make a subgroup $H_C$ which is $H$ conjugated by $w$.
Let's add this problem to \nsmagnus. If $H$ is generated by elements
$<h_1, h_2, ..., h_n>$ then $H_C$ will be generated by $<h_1^w, h_2^w, ..., h_n^w>$.

First, we need to find a place where to put  the algorithm itself. 
Probably the most sencible way is to add it as a property of
the subgroup of a free group. In \emph{magnus/back\_end/Subgroup} we can find
definition and implementation files for subgroups of free groups.
Subgroups are implemented using representation template and therefore implementation
of the algorithm will go into the Rep class which is located in  \emph{SGofFreeGroup.C}.

We add the following procedure to \emph{SGofFreeGroupRep} class:

\footnotesize
\begin{verbatim}
SGofFreeGroupRep* SGofFreeGroupRep::conjugateBy(const Word& w ) const 
{
   
  VectorOf<Word> result(theGenerators.length());

  // conjugate each generator by the word
  for( int i = 0; i < theGenerators.length(); ++i ) 
    result[i] = Word( Word(theGenerators[i]).conjugateBy(w) ).freelyReduce();

  // return representation for the new subgroup
  return new SGofFreeGroupRep(theParentGroup,result);
}
\end{verbatim}
\normalsize

Now we have to add this algorithm to the interface class \emph{SGofFreeGroup}
by introducing a new method:
 
\footnotesize
\begin{verbatim}
SGofFreeGroup SGofFreeGroup::conjugateBy(const Word& w) {
    // create SGofFreeGroup from the representation object
    // returned by (SGofFreeGroupRep*)look()->conjugateBy(w)
    return SGofFreeGroup(look()->conjugateBy(w));
}
\end{verbatim}
\normalsize

Second step,  is to add implemented algorithm to the \nsmagnus' menu.
We can see that conjugating a set of elements by a word is very
fast procedure, therefore we will add it as a fast problem.
Files that contain classes for fast problems are:
\emph{magnus/back\_end/SMApps/include/fastProblems.h} and 
\emph{magnus/back\_end/SMApps/src/fastProblems.C}.

We declare \emph{SubgroupConjugateBy} class in \emph{fastProblems.h}:
\footnotesize
\begin{verbatim}
// it is inhereted from FastComputation manager
class SubgroupConjugateBy : public FastComputation
{
public:

  // constructor accepts two parameters: selected subgroup
  // and selected word by which it will be conjugated
  SubgroupConjugateBy(const class SMSubgroup& S, const class SMWord& W)
    : theSubgroup( S ), theWord( W )
  { }

  void takeControl( );              // overrides ComputationManager

private:

  // saved references to the objects
  const class SMSubgroup& theSubgroup;
  const class SMWord& theWord;
};
\end{verbatim}
\normalsize

The part above almost the same for all fast problems. The only thing
that needs to be changed is the name of the class and objects it will
act on. The main action occurs in  \emph{takeControl()} method. We implement
it in \emph{fastProblems.C} file:
\footnotesize
\begin{verbatim}
void SubgroupConjugateBy::takeControl( )
{
  // get subgroup object from the session manager's representation object
  SGofFreeGroup S = theSubgroup.getSubgroup();

  // create a new session manager's object for the
  // conjugator of theSubgroup
  SMObject* smo =
    new SMSubgroup( theSubgroup.getParent(),
		    theSubgroup.getSubgroup().conjugateBy( theWord.getWord() ),
		    Text(Name( theSubgroup)) + Text(``conjugated by'') + 
			Name( theWord )
		    );
  
  // adjust dependencies. This will ensure the correspondence
  // between the new subgroup and it's parent group
  ListOf<OID> dependencies( theSubgroup.getParent() );
  CheckinMessage( *smo, "", dependencies ).send();
}
\end{verbatim}
\normalsize 

The last thing left is to tell front end where and when to execute
this problem. Sinve we are just creating a new object problem 
should go to the \textbf{Make} menu.
 This is done by adding to \emph{magnus/back\_end/SMApps/src/menuDefns.C}
the following statements:
\footnotesize
\begin{verbatim}
 {
   // define a ``Make'' menu for two selected objects
   // first of type SMSubgroup and second SMWord
   Menu2<SMSubgroup, SMWord> M( Menu::MAKE );
   
   // start a new item group
   // make sure that both objects belong to the same free group
   // here, numbers correspond to the order of the parameters, i.e.
   // 1 - for the subgroup and 2 - for the word
   M.startItemGroup(CheckinType( Parent(1) ) == SMFPGroup::FREE &&
		     Parent( 1 ) == Parent( 2 )
		   );
   
   // add aour class to the menu.
   //  Text will say something like ``H1 conjugated by w1''
   M.add( Ctor2<SubgroupConjugateBy,SMSubgroup,SMWord>(),
	  Text(Name(1)) + Text("conjugated by") + Name(2)
	  );

   // close the menu description 
   M.done();
 }
\end{verbatim}
\normalsize 
\end{document}



\end{document}

