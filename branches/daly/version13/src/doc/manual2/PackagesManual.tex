\documentclass[12pt]{article}

\usepackage{latexsym}
\usepackage{amsfonts}
\usepackage{epsfig}

\parindent=0pt
\parskip=8pt


\begin{document}

\begin{center}
\bf\LARGE MAGNUS' Package Mechanism:   \\ User manual
\end{center}
\medskip

MAGNUS' Package Mechanism was developed to provide a simple and
convenient way of integrating second party software into MAGNUS'
graphical user interface.
Integrated packages communicate with MAGNUS by writing simple
messages to  the standard output and reading messages from the
standard input.

Packages are required to  understand simplified
descriptions (representations) of algebraic  objects accepted in
MAGNUS software package. The formats existing in MAGNUS were
slightly changed in order to simplify parsing of the descriptions
of objects sent to a package from MAGNUS.
For example, many kinds of brackets are eliminated when sent to a
package so that one can read definitions symbol by symbol.

The following is a description of conventions used to represent 
different objects in MAGNUS and in messages as well: 

\textbf{ Algebraic objects:}
\begin{enumerate}
\item \textbf{Word}:
There are a number of conventions concerning word notation.
Generator names can be any letter, such as 
\verb+a, b, ..., x, y, z+
or any letter followed by a sequence of digits such as 
\verb+a1, b23 +
and so on. In putting generators together to form words, there
must be white space between them unless other punctuation is
used. The \verb+^+ is used to denote exponentiation. By default, the
exponents are integers only.

Example: \verb+a b^2 c c d^-4 a^3+

{\em Note}: When sending a word definition to MAGNUS the
following additional conventions are accepted: we allow \verb+a (b c)+
in place of \verb+ a b c+. We sometimes use capital letters to denote
inverses of generators. So the inverses of \verb+a, b, c11, ..+ are
denoted by \verb+A, B, C11,...+ and also by \verb+a^-1, b^-1, c11^-1 +and so
on.  The exponents are allowed to be group elements as well as
integers, with \verb+a^b+ is defined to be the conjugate \verb+B a b+ of 
\verb+a+ by \verb+b+.
Thus \verb+ a^b = B a b = b^-1 a b+.

Our convention is that \verb+^+ is left-associative. So 
\verb+ a^b^2 = (a^b)^2+.

We use square brackets to denote commutators:\\ 
\verb+    [a,b]= a^-1 b^-1 a b = A B a b+.\\
We then define, for $n > 2$, \\
\verb+    [a1,a2,...,an]= [[a1,a2],a3,...,an]+.\\
Thus multiple commutators are "left-normed".

\item \textbf{Tuple} :
Words in a tuple are separated  with
commas and the list is enclosed in \{\}'s. For
example:

\verb+{a, a, a^2, b^a, a, b^2}+.


\item \textbf{Subgroup} : 
A subgroups is described by listing its generators, each such
generator given by a word in the generators of the supergroup. We
separate these words by commas. For example, the subgroup of a
group with generators \verb+a+ and \verb+b+ generated 
by \verb+a^2 b+ and \verb+[a,b]+ is
described by

\verb+    { a^2 b, a^-1 b^-1 a b }+


\item \textbf{ Map}:
To define a map from the generators
\verb+{ x, y, ... }+
of a group G to a group H with generators
\verb+{ a, b, ... }+
assign to each of \verb+x, y, ...+ its image as a word in the generators
of $H$, enclosing the description by \verb+{}+'s. For example:
\begin{verbatim}
{
  x -> a^b,
  y -> b^-1
}
\end{verbatim}
If you omit the image of a generator, Magnus assigns its image to
be the identity.

\item \textbf{Group} : 
We use customary notation in describing a group by means of a
presentation : $< x_1, ..., x_n ; r_1, ..., r_n>$, where $x_i$ -
generators of a group ant $r_i$ - relators.

In case of a free group, symbol ``;'' is ommited. Presentation of
a nilpotent group starts with a nilpotency class, followed by a
presentation in the form described above.

Here are some examples of how to enter a presentation:

\begin{verbatim}
< x, y; x^2  y^3, x y x y >

<a,b; a^-1 b^-1 a b >

< b, t ; t b^2 t^-1 = b b b >

< a,b,c,d,e>

4 <a,b; a b c >
\end{verbatim}
\end{enumerate}


Detailed description of the messages' format together with
examples of their use are given below.

\textbf{How to add a package}

When MAGNUS is started, you will see ``Packages" item in the main
menu. It has two sub-items: ``Add a package" and ``Edit the list of packages".

When ``Add a package" menu is selected a Dialog Window (Figure
\ref{fig:addwin}) will pop-up. This is the window where all
information required to add a package is entered. There are 5
fields that must be filled in:


\begin{figure}[htbp]
%\centerline{\epsfig{file=addwindow.eps}}
\caption{``Add a new  package''  dialog window.}
\label{fig:addwin}
\end{figure}


\begin{enumerate}
\item \emph{Enter project name}: enter a name of the package. This
string will appear as an item in the menu. There is a limit of
20 characters.
\item \emph{Command line}: enter the command which will invoke
the package. You can type it in the text window or select an
executable from the "Open File" dialog window by clicking
"Browse" button. This field is limited to 128 characters.
\item \emph{Select checkin type}: select the type of a group
your package will work with. Together with selected algebraic
objects, this field defines conditions for packages menu item to
appear (see below for details).
\item \emph{Select object}: select algebraic objects
that your package will be applied to. When you  highlight
selected in this field objects which also belong to a group of the
type selected in the previous field the menu item for you package
will appear in the ``Packages" menu. Then you can run the package
by selecting corresponding menu item.
\item \emph{Parameters}: You can define a list of parameters that
can be edited and sent to the package before it is started. Press
``Parameters" button for the ``List of parameters" dialog window. It is
empty at the beginning. You can add a new parameter by pressing
``Add" button. Two fields have to be filled: the {\em name} of the parameter
which will appear as a parameter description in the package's
problem window, and the parameter {\em type} (integer, text or boolean).
\end{enumerate}


The example in Figure \ref{fig:addwin} shows package named 
\verb+Coset Enumerator (PEACE)+ 
which is invoked with \verb+peace -h+ command (here we assume that 
\verb+peace+ lies in the search path and so that the full path name is not 
needed). 
This package is applied to a ``Finitely Presented" group


\textbf{Messages description}: 
Usually communication between Magnus and an integrated package 
starts by sending a description of selected objects and values
of parameters (if necessary )  from Magnus to the package.
After that, the package sends its output to Magnus using messages as
specified below. Notice that if no special command precedes the output
string, it will be shown in the linked file and its link is automatically 
called ``Click here to see the details from \emph{package\_name}''.


The following is a list of messages currently available for
interchange between external packages and Magnus together with examples of
their use.

\textbf{From Magnus to a package}:

The following describes messages Magnus can send to a package.
All messages of this type are of the following format:
\begin{quote}\em
object\_description ( oid\_number )
\end{quote}
where \emph{object\_description} is a string which describes the
object (group presentation, word definition and so on ) as
described above. \emph{oid\_number} is a unique integer
identifier, automatically assigned to each object created in the
Magnus workspace. One can use \emph{oid\_number} to
refer to a particular object, when sending messages to Magnus
Session Manager (see below).

\begin{enumerate}
\item \textbf{Group}: MAGNUS sends a presentation of a group which 
the package should read: \\
 \emph{ group\_presentation ( oid ) }.\\
Example:
\begin{verbatim}
<a,b,c,d ; a^2, b c > (3)
\end{verbatim}
\item \textbf{Nilpotent Group}: \\
\emph{ nilpotency\_class group\_presentation   ( oid ) } \\
Example:
\begin{verbatim}
5 <a,b,c,d ; a^2, b c > (3)
\end{verbatim}
\item \textbf{Word}:  \\
 \emph{ group\_presentation ( oid ) } \\
\emph{  word ; ( oid ) } \\
Example:
\begin{verbatim}
<a,b,c,d ; a^2, b c > (3)
a^-1 b^-1 a b; (12)
\end{verbatim}
\item \textbf{Subgroup} : \\
 \emph{ group\_presentation ( oid ) } \\
\emph{  subgroup\_generators ; ( oid ) } \\
Example:
\begin{verbatim}
<a,b,c,d ; a^2, b c > (3)
{a^-1 b^-1 a b, a^2 c^2}; (12)
\end{verbatim}
\item \textbf{Tuple} : \\
 \emph{ group\_presentation ( oid ) } \\
\emph{  words\_in\_tuple ; ( oid ) } \\
Example:
\begin{verbatim}
<a,b,c,d ; a^2, b c > (3)
{a^-1 b^-1 a b, a^2 c^2}; (12)
\end{verbatim}
\item \textbf{Map} : \\
 \emph{ domain\_group\_presentation ( oid ) } \\
 \emph{ range\_group\_presentation ( oid ) } \\
\emph{  map  ( oid ) } \\
Example:
\begin{verbatim}
<a,b,c,d ; a^2, b c > (3)
<a,b,c,d ; a^2, b c > (3)
{ a->b, b->c, c->d, d->a } (15)
\end{verbatim}
\item \textbf{Homomorphism} : \\
 \emph{ domain\_group\_presentation ( oid ) } \\
 \emph{ range\_group\_presentation ( oid ) } \\
\emph{  homomorphism ; ( oid ) } \\
Example:
\begin{verbatim}
<a,b,c > (0)
<x,y,z > (3)
{ a->x, b->y, c->z } (15)
\end{verbatim}
\item \textbf{Two words (w1,w2)} :  \\
 \emph{ group\_presentation ( oid ) } \\
\emph{  word\_w1 ; ( oid ) } \\
\emph{  word\_w2 ; ( oid ) } \\
Example:
\begin{verbatim}
<a,b,c,d ; a^2, b c > (3)
a^-1 b^-1 a b; (12)
a^-1; (11)
\end{verbatim}
\item \textbf{Two subgroups (H1,H2)} :  \\
 \emph{ group\_presentation ( oid ) } \\
\emph{  subgroup\_H1 ; ( oid ) } \\
\emph{  subgroup\_H2 ; ( oid ) } \\
Example:
\begin{verbatim}
<a,b,c,d ; a^2, b c > (3)
{a^-1 b^-1,  a b }; (12)
{a^-1, b}; (11)
\end{verbatim}
\item \textbf{Word and Subgroup (w,H)} :  \\
 \emph{ group\_presentation ( oid ) } \\
\emph{  subgroup\_H ; ( oid ) } \\
\emph{  word\_w ; ( oid ) } \\
Example:
\begin{verbatim}
<a,b,c,d ; a^2, b c > (3)
{a^-1 b^-1,  a b }; (12)
a^-1 (11)
\end{verbatim}
\item \textbf{Parameters} : \\
\emph{Integer ( name ) (value) } \\
\emph{String ( name ) (value) } \\
\emph{Boolean ( name ) (value) } \\
Example:
\begin{verbatim}
Integer (Maximal number of elements) (1000)
Boolean (Enumerate all) (1)
\end{verbatim}
\end{enumerate}

\textbf{ From Package to Magnus }
\begin{enumerate}
\item Output text into session log : \\
\emph{Text \{ text \} } \\
Example:
\begin{verbatim}
. . .
printf(``Text { Package has started its work } \n'');
. . .
\end{verbatim}

\item Create a link to a file : \\
\emph{Link \{ link\_name , file\_name \} } \\
Example:
\begin{verbatim}
. . .
printf(``Link { Click to see some details , /tmp/detail_file } \n'');
. . .
\end{verbatim}
\item Create a group object in the Workspace : \\
\emph{Create Group \{ group\_presentation \} } \\
Example:
\begin{verbatim}
. . .
printf(``Create Group { <a,b,c,d ; a^2, b^3 c a > } \n'');
. . .
\end{verbatim}
\item Create a word object for a group in the Workspace : \\
\emph{Create Word \{ word $|$ group\_oid \} } \\
Example:
\begin{verbatim}
. . .
printf(``Create Word { a B c b | 3 } \n'');
. . .
\end{verbatim}
\item Create a subgroup in the Workspace : \\
\emph{Create Subgroup \{ list\_of\_subgroup\_generators $|$ group\_oid \} }\\
Example:
\begin{verbatim}
. . .
printf(``Create Subgroup { a b, b c , a d, c^7 | 5 } \n'');
. . .
\end{verbatim}
\item Create a tuple in the Workspace : \\
\emph{Create Tuple \{ words\_in\_tuple $|$ group\_oid \} }\\
Example:
\begin{verbatim}
. . .
printf(``Create Tuple { a b, b c , a d, c^7 | 5 } \n'');
. . .
\end{verbatim}
\item Create a map : \\
\emph{Create Map \{ map $|$ domain\_group\_oid $|$ range\_group\_oid \} }\\
Example:
\begin{verbatim}
. . .
printf(``Create Map { a - > x, b->y^2 | 8 | 14 } \n'');
. . .
\end{verbatim}
\end{enumerate}


Here is an example of a simple package which is written in Perl.
It takes a word from Magnus and returns its $n$-th power where
$n$ is a parameter to be specified.

First the package is added to the ``Packages" menu in Magnus. The name,
command to invoke the package, the type and the fact that an integer
parameter is required are all entered via a dialog window.

To use the package, check-in a finitely presented group and a word.
With the word highlighted, select the package from the menu. Enter
the power as a parameter via the menu in the ``Start" dialog window 
that appears.

Here is the Perl script which returns the appropriate results:

\begin{verbatim}
#!/usr/bin/perl

#
#  Expect 3 lines on STDIN as follows:
#
# <a, b, c ; a^2, b^3 c^-4 >  (5)       # presentation and its oid
# a^-1 b^-1 a^2 b^3;  (12)              # word with its oid
# Integer (power) (17)                  # power to be raised
#

$lineone = <STDIN>;
@parts = split(/\D+/, $lineone);        # split line into numbers
$groupoid = $parts[$#parts];            # oid = the last number

$linetwo = <STDIN>;
@parts = split(/;/,$linetwo);           # split line at semi-colon ;
$theword = $parts[0];                   # initial part is the word

$linethree = <STDIN>;
@parts = split(/\D+/,$linethree);       # split line into numbers
$power = $parts[$#parts];               # power = the last number

print "Text {About to compute the ${power}-th power of $theword } \n";

$powerstr = "";

for ( $i = 0; $i < $power; $i++ ){
   $powerstr = $powerstr." $theword";
}

print "Create Word {$powerstr | $groupoid } \n";
\end{verbatim}

\end{document}

