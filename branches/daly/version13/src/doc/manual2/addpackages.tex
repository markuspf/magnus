BABYL OPTIONS: -*- rmail -*-
Version: 5
Labels:
Note:   This is the header of an rmail file.
Note:   If you are seeing it in rmail,
Note:    it means the file has no messages in it.

1,,
X-Coding-System: undecided-unix
Mail-from: From cfm@localhost.localdomain  Mon Oct 25 14:47:06 2004
Return-Path: <cfm@localhost.localdomain>
Received: from localhost.localdomain (g1 [127.0.0.1])
	by localhost.localdomain (8.12.10/8.12.10) with ESMTP id i9PIl6pG020203
	for <daly@localhost.localdomain>; Mon, 25 Oct 2004 14:47:06 -0400
Received: (from cfm@localhost)
	by localhost.localdomain (8.12.10/8.12.10/Submit) id i9PIl69Y020201
	for daly; Mon, 25 Oct 2004 14:47:06 -0400
Date: Mon, 25 Oct 2004 14:47:06 -0400
From: cfmiller <cfm@localhost.localdomain>
Message-Id: <200410251847.i9PIl69Y020201@localhost.localdomain>
To: daly@localhost.localdomain

*** EOOH ***
Date: Mon, 25 Oct 2004 14:47:06 -0400
From: cfmiller <cfm@localhost.localdomain>
To: daly@localhost.localdomain

\documentclass[12pt]{article}

\usepackage{latexsym}
\usepackage{amsfonts}
\usepackage{epsfig}

\parindent=0pt
\parskip=8pt
\def\magnus{MAGNUS }
\def\nsmagnus{MAGNUS}

%\textheight=23cm

\begin{document}


\section{Adding packages}

\nsmagnus' Package Mechanism was developed to provide a simple and
convenient way of integrating third party software into \nsmagnus'
graphical user interface.
Integrated packages communicate with \magnus by writing simple
messages to  the standard output and reading messages from the
standard input.

Packages are required to  understand simplified
descriptions (representations) of algebraic  objects accepted in
\magnus software package. The formats existing in \magnus were
slightly changed in order to simplify parsing of the descriptions
of objects sent to a package from \nsmagnus.
For example, many kinds of brackets are eliminated when sent to a
package so that one can read definitions symbol by symbol.

The following is a description of conventions used to represent
different objects in \magnus and in messages as well:

\textbf{ Algebraic objects:}
\begin{enumerate}
\item \textbf{Word}:
There are a number of conventions concerning word notation.
Generator names can be any letter, such as
\verb+a, b, ..., x, y, z+
or any letter followed by a sequence of digits such as
\verb+a1, b23 +
and so on. In putting generators together to form words, there
must be white space between them unless other punctuation is
used. The \verb+^+ is used to denote exponentiation. By default, the
exponents are integers only.

Example: \verb+a b^2 c c d^-4 a^3+

{\em Note}: When sending a word definition to \magnus the
following additional conventions are accepted: we allow \verb+a (b c)+
in place of \verb+ a b c+. We sometimes use capital letters to denote
inverses of generators. So the inverses of \verb+a, b, c11, ..+ are
denoted by \verb+A, B, C11,...+ and also by \verb+a^-1, b^-1, c11^-1 +and so
on.  The exponents are allowed to be group elements as well as
integers, with \verb+a^b+ is defined to be the conjugate \verb+B a b+ of
\verb+a+ by \verb+b+.
Thus \verb+ a^b = B a b = b^-1 a b+.

Our convention is that \verb+^+ is left-associative. So
\verb+ a^b^2 = (a^b)^2+.

We use square brackets to denote commutators:\\
\verb+    [a,b]= a^-1 b^-1 a b = A B a b+.\\
We then define, for $n > 2$, \\
\verb+    [a1,a2,...,an]= [[a1,a2],a3,...,an]+.\\
Thus multiple commutators are "left-normed".

\item \textbf{Tuple} :
Words in a tuple are separated  with
commas and the list is enclosed in \{\}'s. For
example:

\verb+{a, a, a^2, b^a, a, b^2}+.


\item \textbf{Subgroup} :
A subgroups is described by listing its generators, each such
generator given by a word in the generators of the supergroup. We
separate these words by commas. For example, the subgroup of a
group with generators \verb+a+ and \verb+b+ generated
by \verb+a^2 b+ and \verb+[a,b]+ is
described by

\verb+    { a^2 b, a^-1 b^-1 a b }+


\item \textbf{ Map}:
To define a map from the generators
\verb+{ x, y, ... }+
of a group G to a group H with generators
\verb+{ a, b, ... }+
assign to each of \verb+x, y, ...+ its image as a word in the generators
of $H$, enclosing the description by \verb+{}+'s. For example:
\begin{verbatim}
{
    x -> a^b,
    y -> b^-1
}
\end{verbatim}
If you omit the image of a generator, \magnus assigns its image to
be the identity.

\item \textbf{Group} :
We use customary notation in describing a group by means of a
presentation : $< x_1, ..., x_n ; r_1, ..., r_n>$, where $x_i$ -
generators of a group ant $r_i$ - relators.

In case of a free group, symbol ``;'' is ommited. Presentation of
a nilpotent group starts with a nilpotency class, followed by a
presentation in the form described above.

Here are some examples of how to enter a presentation:

\begin{verbatim}
< x, y; x^2  y^3, x y x y >

<a,b; a^-1 b^-1 a b >

< b, t ; t b^2 t^-1 = b b b >

< a,b,c,d,e>

4 <a,b; a b c >
\end{verbatim}
\end{enumerate}


Detailed description of the messages' format together with
examples of their use are given below.

\textbf{How to add a package}

When \magnus is started, you will see ``Packages" item in the main
menu. It has two sub-items: ``Add a package" and ``Edit the list of packages".

When ``Add a package" menu is selected a Dialog Window (Figure
\ref{fig:addwin}) will pop-up. This is the window where all
information required to add a package is entered. There are 5
fields that must be filled in:


\begin{figure}[htbp]
%\centerline{\epsfig{file=addwindow.eps}}
\caption{``Add a new  package''  dialog window.}
\label{fig:addwin}
\end{figure}


\begin{enumerate}
\item {\em Enter project name}: enter a name of the package. This
string will appear as an item in the menu. There is a limit of
20 characters.
\item {\em Command line}: enter the command which will invoke
the package. You can type it in the text window or select an
executable from the ``Open File" dialog window by clicking
``Browse" button. This field is limited to 128 characters.
\item {\em Select checkin type}: select the type of a group
your package will work with. Together with selected algebraic
objects, this field defines conditions for packages menu item to
appear (see below for details).
\item {\em Select object}: select algebraic objects
that your package will be applied to. When you  highlight
selected in this field objects which also belong to a group of the
type selected in the previous field the menu item for you package
will appear in the ``Packages" menu. Then you can run the package
by selecting corresponding menu item.
\item {\em Parameters}: You can define a list of parameters that
can be edited and sent to the package before it is started. Press
``Parameters" button for the ``List of parameters" dialog window. It is
empty at the beginning. You can add a new parameter by pressing
``Add" button. Two fields have to be filled: the {\em name} of the parameter
which will appear as a parameter description in the package's
problem window, and the parameter {\em type} (integer, text or boolean).
\end{enumerate}


The example in Figure \ref{fig:addwin} shows package named
\verb+Coset Enumerator (PEACE)+
which is invoked with \verb+peace -h+ command (here we assume that
\verb+peace+ lies in the search path and so that the full path name is not
needed).
This package is applied to a ``Finitely Presented" group


\textbf{Messages description}:
Usually communication between \magnus and an integrated package
starts by sending a description of selected objects and values
of parameters (if necessary )  from \magnus to the package.
After that, the package sends its output to \magnus using messages as
specified below. Notice that if no special command precedes the output
string, it will be shown in the linked file and its link is automatically
called ``Click here to see the details from {\em package\_name}''.


The following is a list of messages currently available for
interchange between external packages and \magnus together with examples of
their use.

\textbf{From \magnus to a package}:

The following describes messages \magnus can send to a package.
All messages of this type are of the following format:
\begin{quote}\em
object\_description ( oid\_number )
\end{quote}
where {\em object\_description} is a string which describes the
object (group presentation, word definition and so on ) as
described above. {\em oid\_number} is a unique integer
identifier, automatically assigned to each object created in the
\magnus workspace. One can use {\em oid\_number} to
refer to a particular object, when sending messages to \magnus
session manager (see below).

\begin{enumerate}
\item \textbf{Group}: \magnus sends a presentation of a group which
the package should read: \\
   {\em  group\_presentation ( oid ) }.\\
Example:
\begin{verbatim}
<a,b,c,d ; a^2, b c > (3)
\end{verbatim}
\item \textbf{Nilpotent Group}: \\
{\em  nilpotency\_class group\_presentation   ( oid ) } \\
Example:
\begin{verbatim}
5 <a,b,c,d ; a^2, b c > (3)
\end{verbatim}
\item \textbf{Word}:  \\
   {\em  group\_presentation ( oid ) } \\
{\em   word ; ( oid ) } \\
Example:
\begin{verbatim}
<a,b,c,d ; a^2, b c > (3)
a^-1 b^-1 a b; (12)
\end{verbatim}
\item \textbf{Subgroup} : \\
   {\em  group\_presentation ( oid ) } \\
{\em   subgroup\_generators ; ( oid ) } \\
Example:
\begin{verbatim}
<a,b,c,d ; a^2, b c > (3)
{a^-1 b^-1 a b, a^2 c^2}; (12)
\end{verbatim}
\item \textbf{Tuple} : \\
   {\em  group\_presentation ( oid ) } \\
{\em   words\_in\_tuple ; ( oid ) } \\
Example:
\begin{verbatim}
<a,b,c,d ; a^2, b c > (3)
{a^-1 b^-1 a b, a^2 c^2}; (12)
\end{verbatim}
\item \textbf{Map} : \\
   {\em  domain\_group\_presentation ( oid ) } \\
   {\em  range\_group\_presentation ( oid ) } \\
{\em   map  ( oid ) } \\
Example:
\begin{verbatim}
<a,b,c,d ; a^2, b c > (3)
<a,b,c,d ; a^2, b c > (3)
{ a->b, b->c, c->d, d->a } (15)
\end{verbatim}
\item \textbf{Homomorphism} : \\
   {\em  domain\_group\_presentation ( oid ) } \\
   {\em  range\_group\_presentation ( oid ) } \\
{\em   homomorphism ; ( oid ) } \\
Example:
\begin{verbatim}
<a,b,c > (0)
<x,y,z > (3)
{ a->x, b->y, c->z } (15)
\end{verbatim}
\item \textbf{Two words (w1,w2)} :  \\
   {\em  group\_presentation ( oid ) } \\
{\em   word\_w1 ; ( oid ) } \\
{\em   word\_w2 ; ( oid ) } \\
Example:
\begin{verbatim}
<a,b,c,d ; a^2, b c > (3)
a^-1 b^-1 a b; (12)
a^-1; (11)
\end{verbatim}
\item \textbf{Two subgroups (H1,H2)} :  \\
   {\em  group\_presentation ( oid ) } \\
{\em   subgroup\_H1 ; ( oid ) } \\
{\em   subgroup\_H2 ; ( oid ) } \\
Example:
\begin{verbatim}
<a,b,c,d ; a^2, b c > (3)
{a^-1 b^-1,  a b }; (12)
{a^-1, b}; (11)
\end{verbatim}
\item \textbf{Word and Subgroup (w,H)} :  \\
   {\em  group\_presentation ( oid ) } \\
{\em   subgroup\_H ; ( oid ) } \\
{\em   word\_w ; ( oid ) } \\
Example:
\begin{verbatim}
<a,b,c,d ; a^2, b c > (3)
{a^-1 b^-1,  a b }; (12)
a^-1 (11)
\end{verbatim}
\item \textbf{Parameters} : \\
{\em Integer ( name ) (value) } \\
{\em String ( name ) (value) } \\
{\em Boolean ( name ) (value) } \\
Example:
\begin{verbatim}
Integer (Maximal number of elements) (1000)
Boolean (Enumerate all) (1)
\end{verbatim}
\end{enumerate}

\textbf{ From Package to \magnus }
\begin{enumerate}
\item Output text into session log : \\
{\em Text \{ text \} } \\
Example:
\begin{verbatim}
. . .
printf(``Text { Package has started its work } \n'');
. . .
\end{verbatim}

\item Create a link to a file : \\
{\em Link \{ link\_name , file\_name \} } \\
Example:
\begin{verbatim}
. . .
printf(``Link { Click to see some details , /tmp/detail_file } \n'');
. . .
\end{verbatim}
\item Create a group object in the Workspace : \\
{\em Create Group \{ group\_presentation \} } \\
Example:
\begin{verbatim}
. . .
printf(``Create Group { <a,b,c,d ; a^2, b^3 c a > } \n'');
. . .
\end{verbatim}
\item Create a word object for a group in the Workspace : \\
{\em Create Word \{ word $|$ group\_oid \} } \\
Example:
\begin{verbatim}
. . .
printf(``Create Word { a B c b | 3 } \n'');
. . .
\end{verbatim}
\item Create a subgroup in the Workspace : \\
{\em Create Subgroup \{ list\_of\_subgroup\_generators $|$ group\_oid \} }\\
Example:
\begin{verbatim}
. . .
printf(``Create Subgroup { a b, b c , a d, c^7 | 5 } \n'');
. . .
\end{verbatim}
\item Create a tuple in the Workspace : \\
{\em Create Tuple \{ words\_in\_tuple $|$ group\_oid \} }\\
Example:
\begin{verbatim}
. . .
printf(``Create Tuple { a b, b c , a d, c^7 | 5 } \n'');
. . .
\end{verbatim}
\item Create a map : \\
{\em Create Map \{ map $|$ domain\_group\_oid $|$ range\_group\_oid \} }\\
Example:
\begin{verbatim}
. . .
printf(``Create Map { a - > x, b->y^2 | 8 | 14 } \n'');
. . .
\end{verbatim}
\end{enumerate}


Here is an example of a simple package which is written in Perl.
It takes a word from \magnus and returns its $n$-th power where
$n$ is a parameter to be specified.

First the package is added to the ``Packages" menu in \nsmagnus. The name,
command to invoke the package, the type and the fact that an integer
parameter is required are all entered via a dialog window.

To use the package, check-in a finitely presented group and a word.
With the word highlighted, select the package from the menu. Enter
the power as a parameter via the menu in the ``Start" dialog window
that appears.

Here is the Perl script which returns the appropriate results:

\begin{verbatim}
#!/usr/bin/perl

#
#  Expect 3 lines on STDIN as follows:
#
# <a, b, c ; a^2, b^3 c^-4 >  (5)       # presentation and its oid
# a^-1 b^-1 a^2 b^3;  (12)              # word with its oid
# Integer (power) (17)                  # power to be raised
#

$lineone = <STDIN>;
@parts = split(/\D+/, $lineone);        # split line into numbers
$groupoid = $parts[$#parts];            # oid = the last number

$linetwo = <STDIN>;
@parts = split(/;/,$linetwo);           # split line at semi-colon ;
$theword = $parts[0];                   # initial part is the word

$linethree = <STDIN>;
@parts = split(/\D+/,$linethree);       # split line into numbers
$power = $parts[$#parts];               # power = the last number

print "Text {About to compute the ${power}-th power of $theword } \n";

$powerstr = "";

for ( $i = 0; $i < $power; $i++ ){
     $powerstr = $powerstr." $theword";
}

print "Create Word {$powerstr | $groupoid } \n";
\end{verbatim}




\section{Adding new algorithms to \magnus}


\magnus consists of two major parts: the front end (FE) and  the back end (BE).
The FE is implemented in Tcl/Tk . It is responsible for representing
information
(obtained from the BE) on the screen and transferring the input from
the user to the BE.
Almost all transactions of the front end are controlled by a back end process
called the session manager. Developers  that are not involved in changing of
\nsmagnus' GUI design
usually do not have to work with the front end. A more detailed
description of the
communications between the front end and the back end can be found in \magnus
Reference Manual 1. In this section we describe the steps that are
needed to add new  algorithms into \nsmagnus.

Adding a new algorithm to \magnus involves adding code to the back end
class structure  which  is implemented  in standard C++.
Source code for the \nsmagnus' back end is located in
the {\tt magnus/back\_end/} directory
which consists of directories containing descriptions of the various classes.



Usually the directory names indicate where to look for a particular
class or a group of classes. Each directory has the following structure:
\begin{itemize}
\item include/ - directory containing the definition files;
\item src/ - contains implementation files;
\item test/ -  source code for the local executables which, after the
compilation,
usually can be found in test/bin.
\item Makefile - makefile for the current directory.
\end{itemize}

\subsection{Communicating with the front end}
The \magnus session manager can require information from
the front end, for example the names of the selected objects, their relations,
and so on.  In addition the \magnus front end must be able to
reflect the state
of the session manager.  So a small number of events are pre-defined
in response to
which the front end is to take some action which reflects the session manager's
state. The session manager describes to the front end what to do in each event,
via a simple message  protocol.
Most of such events  are for static properties of algebraic objects which
are needed to determine whether a menu item, etc., should be posted.

The following is a list  of classes which implement interaction events
available from the EData objects:
\begin{enumerate}
\item  {\tt struct Text}. Used in session manager to deliver a text message
to the front end.
\item {\tt  struct CheckinType}. Correspond to a ``type'' of the group in the
front end.
Usually used to set up conditions in the front end. For example:
\footnotesize
\begin{verbatim}
M.startItemGroup( CheckinType( 1 ) == SMFPGroup::ABELIAN );
\end{verbatim}
\normalsize
Here {\tt   CheckinType( 1 ) == SMFPGroup::ABELIAN} is true in the
front end if
the
``type'' of the first selected in the Workspace ``group'' object is Abelian.
\item {\tt  struct IsHomo}. True in the front end if the  parameter
object is a
homomorphism.
\item {\tt  struct IsIso}. True in the front end if the  parameter
object is an
isomorphism.
\item {\tt  struct IsAuto}. True in the front end if the  parameter object is
an
automorphism.
\item {\tt  struct Parent}. Detects the parent for a given object.
\item {\tt  struct Domain}. Gives the domain group of a given object.
\item {\tt struct Range}. Gives the range group of a given object.
\item {\tt struct Oid}. Gives  unique object ID for a given object.
\item {\tt  struct Name}. Gives a name in the Workspace for corresponding
object.
\item {\tt struct Link}. A special kind of a text message. Allows to create a
link to the specified file in  the Workspace logs.
\end{enumerate}

Events are sent by means of a simple messages. Most of the messages are derived
from the ostream. To send a message a special method {\tt void send()} has to
be called.  The following are most useful for the user messages available
in session manager:
\begin{enumerate}
\item {\tt class LogMessage : public OutMessage, protected FEData, public
ostrstream}.
Sends a message to the specified log window in the front end.
This is the most commonly used type of the messages. All reports, links to the
files and so on are delivered by means of this message.

\item {\tt class Warning : public OutMessage, protected FEData, public
ostrstream}.
Creates a warning window with the specified text.

\item {\tt class Message : public OutMessage, protected FEData, public
ostrstream}. Posts
a message in the separate dialog window.
\end{enumerate}

Example:

\footnotesize
\begin{verbatim}
LogMessage msg( *this, theWord );
msg << Name( theWord ) << " is not trivial.";
msg.send();
// Print in the main log window and the log window corresponding
// to ``theWord''  the following message: ``w1 is not trivial.''

LogMessage msg1(*this,theGroup);
msg1 << Link(Chars( "Click here to see the canonical decomposition of ")+
                  Text(Name( theGroup )),
                 "CyclicDecomposition",
                 ``can_decomp.txt'' );
msg1.send();
// Create a link to the file can_decomp.txt in the main log window
// and the log window for ``theGroup''
// Text of the link: ``Click here to see the canonical decomposition of G1''

\end{verbatim}
\normalsize

\subsection{Generalities about menus} Menus in \magnus are contextual
in the sense that their contents depend on which items are selected
in the Workspace window.  If no icons are selected, then only the Check-in
menu for new objects and a few default items are available. If one or more
items are selected, then a variety of relevant tools become available.
For instance if a subgroup H1 and a word w1 are selected, the tools menu
will contain a number of algorithms for answer questions about their
relationship, for instance ``Is w1 an delement of H1?".


In most cases one can predict the running time of a problem in advance.
We say that a problem is ``fast''  if it runs approximately less than a
second on a modern computer. Usually these problems are those that
can be run in constant time or linear time where upper bound on the
input size is known. When we cannot assure fast execution of the algorithm,
we must create a {\em problem window} with a list of {\em
computation managers} (see below) which are used to distribute
the machine resources.


\subsection{Creating a menu item}
Even though  fast and slow problems  involve different methods of
executing the algorithms, the corresponding menu items are created in the
uniform fashion.

There is a group of template classes which are used to create menus.
Menu itself is described by one of the classes inherited from {\tt class Menu}
which can be
found in
{\tt magnus/back\_end/SessionManager/include/Menu.C}.
Here is a short description of its methods:
\begin{itemize}
\item {\tt  enum MenuKind \{CHECKIN, TOOLS, MAKE, PACKAGES,\\ 
DATABASE, TESTING\}} - Main menu items' types.

\item {\tt   void startItemGroup( ) } - Start an item group with no condition.


\item {\tt  void startItemGroup(const Expression\& condition)} - Start an item 
group with condition. Every item in the menu that has been added in this item group 
will appear in the menu only if ``condition" is true


\item {\tt  void startCascade(const Text\& text)} - Start cascade menu

\item {\tt   void closeCascade( ) }- Close cascade menu

\item {\tt  void addDisabled(const Text\& text)} - Add a disabled
(grayed) menu
item

\item {\tt   void separator( ) } -  Add a separator in the menu

\item {\tt   void done( )} - Close current item group
\end{itemize}

  From  {\tt Menu} we derive classes of the form:
\begin{center}
{\tt template $<$ list of classes $>$ class Menu\textbf{N} : public Menu}
\end{center}
where {\bf N} is the number of objects selected in the Workspace when this
menu is activated and  {\tt list of classes} is a list of parameter classes
which correspond to the selected objects in the Workspace.

In addition to the inherited methods, all {\tt Menu\textbf{N}} classes
have
\begin{center}
{\tt void add(const CtorArgs\textbf{N} $<$list of classes$>$\& c, const Text\&
t, const Action \&a ) }
\end{center}
which actually adds an item to the menu and associates menu with a class
which will execute the corresponding action. Here, {\tt t} is a text 
in the menu,
{\tt a} some additional action, usually left as default and {\tt c} 
is an instance of
one of the  special classes {\tt Ctor\textbf{N}} - a
{\em constructor wrapper}, that is, an appropriate derivative of class Ctor.
These arrange, in cooperation with class ObjectSmith, for a callback
when the menu item is selected.

The definition of {\tt Ctor\textbf{N} } is similar to one for
{\tt Menu\textbf{N} }:
\begin{center}
{\tt template $<$ T, list of object classes $>$ class Ctor\textbf{N} : public
CtorArgs\textbf{N}
$<$list of object classes$>$}.
\end{center}
Here, again, {\bf N} is the number of objects
selected in the Workspace, {\tt list of classes} specifies the types 
of corresponding
objects and {\tt T}  is a class (usually derived from {\em supervisor} or
{\em computation manager} classes) which actually defines the action (see
below).


Examples of defining menu items are shown below:
\footnotesize
\begin{verbatim}
{
    Menu1<SMFPGroup> M( Menu::TOOLS );
    // Create a menu object which correspond to one
    // selected ``group'' object in the Workspace and
    // ``Tools'' item of the main menu


    M.startItemGroup( CheckinType( 1 ) == SMFPGroup::ABELIAN );
    // Start group for Abelian groups only

    M.startCascade( Text("Is") + Name(1) + Text("...") );
    // Start a cascade menu ``Is A1 ...''

    M.add( Ctor1<IsTrivialProblem, SMFPGroup>(),
          Text("Is") + Name(1) + Text("trivial ?")
          );
    // Insert item ``Is A1 trivial?'' and assign IsTrivialProblem class
    // to perform an action when this menu item is selected

    M.add( Ctor1<IsFiniteProblem, SMFPGroup>(),
          Text("Is") + Name(1) + Text("finite ?")
          );
    // Insert item ``Is A1 finite?'' and assign IsFiniteProblem class
    // to perform an action when this menu item is selected

    M.closeCascade();
    // We done with current cascade menu

    M.startCascade( Text("Compute ...") );
    // Start a new cascade menu ``Compute ...''

    M.add( Ctor1<AbelianOrderProblem, SMFPGroup>(),
          Text("Compute the order of") + Name(1) );
    // Insert item ``Compute the order of A1?'' and assign
    // AbelianOrderProblem class
    // to perform an action when this menu item is selected

    .   .   .

    M.closeCascade();

    M.done()
    // Done with the  menu object
}

{
   Menu2<SMWord,SMWord> M( Menu::TOOLS );
   // Create a menu object which correspond to two
   // selected ``word objects in the Workspace and
   // ``Tools'' item of the main menu

    M.startItemGroup( Parent( 1 ) == Parent( 2 ) &&
                     CheckinType( Parent(1) ) == SMFPGroup::ABELIAN
                     );
    // Condition - both words correspond to the same ``group'' object
    // and this group is abelian

    M.add( Ctor2<ProductInAbelianForm, SMWord, SMWord>(),
          Text("Compute the product") + Name(1) + Name(2)
          + Text(", in abelian form")
          );
    // Insert item ``Compute the product w1w2 in the abelian form?''
    // and assign ProductInAbelianForm class
    // to compute the product

    ...

    M.startItemGroup( Parent( 1 ) == Parent( 2 ) &&
                     CheckinType( Parent(1) ) != SMFPGroup::ABELIAN
                     );

    // Condition - both words correspond to the same ``group'' object
    // and this group is NOT abelian

    M.add( Ctor2<FormalProductOfWords,SMWord,SMWord>(),
          Text("Compute the product") + Name(1) + " * " + Name(2)
          );

    M.separator();
    // Put a separator line in the menu
    ...

    M.done()
}

\end{verbatim}
\normalsize

\subsection{ Adding fast problems to a \magnus menu}
The above explains how to modify menus and add items.
The next step is to code the algorithms or actions
invoked from the menu. We start with the simplest case when the action
is almost instantaneous (so called ``fast problem'').

All ``fast problems'' classes are derived from the {\em FastComputation} class
which is derived from {\em ComputationManager}.
We must provide a constructor, which  receives
the object to which an action is applied, and override a special function
{\tt  void takeControl( )} in which we actually implement the desired action.
This function is called by the session manager every time the corresponding
menu item is selected.

After a derived class is implemented we need to add a corresponding menu
item by changing {\tt magnus/back\_end/SMApps/src/menuDefns.C}.
First, find the proper Menu object and call {\tt add()} method to
add the problem as was described above.

The example below shows the source code for adding an item
to solve the word problem in a free group.  Another more complete example is
provided in the next section.

\footnotesize
\begin{verbatim}

// --------------------- menuDefns.C -------------------------- //
// the following text occurs in the menuDefns.C:
M.add( Ctor1<WordProblemInFree,SMWord>(),
          Text("Is") + Name(1) + "trivial ?"
          );

//Part of  fastProblems.h corresponding to the word problem in a free group
// --------------------- WordProblemInFree -------------------- //

// the following class implements an action invoked by
// selecting ``Is w1 trivial?'' menu item for a free group, i.e.
// our action is to receive a word and solve word problem for
// this word in the parent free group

class WordProblemInFree : public FastComputation
{
public:

// Receive the reference to the  word and save it
WordProblemInFree(const class SMWord& w) : theWord( w ) { }

// implement the action
void takeControl( ) //overrides ComputationManager
{
       Word u, v;

       // get the parent free group
       FreeGroup F = theWord.getParent().getFreePreimage();

       // Create log message
       LogMessage msg( theWord );
       msg << Name( theWord );

       // if given word is trivial
       if ( F.wordProblem( theWord.getWord() ) ) {
          msg << " is trivial";
       } else {
          msg << " is not trivial";
       }

       // finish and send the message
       msg << " in " << Name( Parent( theWord ) ) << ".";
       msg.send();
}

private:

const class SMWord& theWord;
};
\end{verbatim}
\normalsize

\subsection{An example of adding a fast problem} Here is a fairly
complete example of adding a ``fast problem" algorithm together with
its menu item.

Let $H$ be a subgroup and $w$ be a word in some free group $F$.
Suppose we want to make a subgroup $H_C$ which is $H$ conjugated by $w$.
Let's add this problem to \nsmagnus. If $H$ is generated by elements
$<h_1, h_2, ..., h_n>$ then $H_C$ will be generated by $<h_1^w,
h_2^w, ..., h_n^w>$.

First we need to find a place where to put the algorithm itself.
Probably the most sensible way is to add it as a property of
the subgroup of a free group. In {\tt magnus/back\_end/Subgroup} we can find
definition and implementation files for subgroups of free groups.
Subgroups are implemented using a representation template and therefore
implementation
of the algorithm will go into the Rep class which is located in
{\tt SGofFreeGroup.C}.

We add the following procedure to {\tt SGofFreeGroupRep} class:

\footnotesize
\begin{verbatim}
SGofFreeGroupRep* SGofFreeGroupRep::conjugateBy(const Word& w ) const
{

    VectorOf<Word> result(theGenerators.length());

    // conjugate each generator by the word
    for( int i = 0; i < theGenerators.length(); ++i )
      result[i] = Word( Word(theGenerators[i]).conjugateBy(w) ).freelyReduce();

    // return representation for the new subgroup
    return new SGofFreeGroupRep(theParentGroup,result);
}
\end{verbatim}
\normalsize

Now we have to add this algorithm to the interface class {\tt SGofFreeGroup}
by introducing a new method:

\footnotesize
\begin{verbatim}
SGofFreeGroup SGofFreeGroup::conjugateBy(const Word& w) {
      // create SGofFreeGroup from the representation object
      // returned by (SGofFreeGroupRep*)look()->conjugateBy(w)
      return SGofFreeGroup(look()->conjugateBy(w));
}
\end{verbatim}
\normalsize

The second step is to add the implemented algorithm to the \magnus menu.
We know that conjugating a set of elements by a word is very
fast procedure, therefore we will add it as a fast problem.
Files that contain classes for fast problems are:
{\tt magnus/back\_end/SMApps/include/fastProblems.h} and
{\tt magnus/back\_end/SMApps/src/fastProblems.C}.

We declare {\tt SubgroupConjugateBy} class in {\tt fastProblems.h}:
\footnotesize
\begin{verbatim}
// it is inhereted from FastComputation manager
class SubgroupConjugateBy : public FastComputation
{
public:

    // constructor accepts two parameters: selected subgroup
    // and selected word by which it will be conjugated
    SubgroupConjugateBy(const class SMSubgroup& S, const class SMWord& W)
      : theSubgroup( S ), theWord( W )
    { }

    void takeControl( );              // overrides ComputationManager

private:

    // saved references to the objects
    const class SMSubgroup& theSubgroup;
    const class SMWord& theWord;
};
\end{verbatim}
\normalsize

The segment above is almost the same for all fast problems. The only thing
that needs to be changed is the name of the class and the objects it will
act on. The main action occurs in the {\tt takeControl()} method. We implement
it in the {\tt fastProblems.C} file:
\footnotesize
\begin{verbatim}
void SubgroupConjugateBy::takeControl( )
{
    // get subgroup object from the session manager's representation object
    SGofFreeGroup S = theSubgroup.getSubgroup();

    // create a new session manager's object for the
    // conjugator of theSubgroup
    SMObject* smo =
      new SMSubgroup( theSubgroup.getParent(),
             theSubgroup.getSubgroup().conjugateBy( theWord.getWord() ),
             Text(Name( theSubgroup)) + Text(``conjugated by'') +
             Name( theWord )
             );

    // adjust dependencies. This will ensure the correspondence
    // between the new subgroup and it's parent group
    ListOf<OID> dependencies( theSubgroup.getParent() );
    CheckinMessage( *smo, "", dependencies ).send();
}
\end{verbatim}
\normalsize

Finally we must tell the front end where and when to execute
this algorithm. Since we are just creating a new object problem
should go to the \textbf{Make} menu.
This is done by adding to {\tt magnus/back\_end/SMApps/src/menuDefns.C}
the following statements:
\footnotesize
\begin{verbatim}
   {
     // define a ``Make'' menu for two selected objects
     // first of type SMSubgroup and second SMWord
     Menu2<SMSubgroup, SMWord> M( Menu::MAKE );

     // start a new item group
     // make sure that both objects belong to the same free group
     // here, numbers correspond to the order of the parameters, i.e.
     // 1 - for the subgroup and 2 - for the word
     M.startItemGroup(CheckinType( Parent(1) ) == SMFPGroup::FREE &&
              Parent( 1 ) == Parent( 2 )
              );

     // add aour class to the menu.
     //  Text will say something like ``H1 conjugated by w1''
     M.add( Ctor2<SubgroupConjugateBy,SMSubgroup,SMWord>(),
            Text(Name(1)) + Text("conjugated by") + Name(2)
            );

     // close the menu description
     M.done();
   }
\end{verbatim}
\normalsize




\subsection{Adding a slow problem to \magnus}

Many problems \magnus deals with are more complicated than
fast problems. Usually we can't say
in advance when we'll get the answer (sometimes we can't even say if we'll get
the answer at all). Some problems could be solved in different ways and it is
often
not clear which method is better. All this requires a smarter managing system.
\magnus
faces these problems by running algorithms in parallel and allowing the user to
distribute resources among the algorithms.

Brielfy, a process of solving a problem can be described in the
following steps. When user selects an item from the menu a corresponding
Supervisor object (inhereted from the {\tt Supervisor} class) is created and
a problem window will come up.
After user clicks ``Start'' button, the {\tt takeControl()} methods of the
{\tt Supervisor}
object and computation managers (defined in the
{\tt viewStructure()} method of
the {\tt Supervisor}) are  executed one by one. Usually each of them spends one time
unit (arc) and gives control tho the next computation manager. This process
continues until the termination conditions are met. These  conditions are:
either all computation managers (including the {\tt Supervisor}) have been stoped and
the {\tt adminTerminate()} method was called, or user terminated them by clicking
``Terminate'' button of the problem window.


Let's see what the steps of adding a problem  by following an example (notice
that the source code was slightly changed to make example easier).
Consider the problem of computing abelian invariants (cyclic decomposition) for
an abelian group.  The interface and implementation for the {\tt AbelianGroup} class can be
found in: {\tt magnus/back\_end/Groups/}.
There are two files: {\tt include/AbelianGroup.h} and
{\tt src/AbelianGroup.C}.
There are many different methods in {\tt AbelianGroup} but we are
interested in those  which correspond to computing of the cyclic decomposition.
In {\tt AbelianGroup.h} we will find definitions for   two methods:
{\tt computeCyclicDecomposition()} and  {\tt invariants()}.
The first does the actual computations and the second returns the result
(invariants) as avector of integer numbers.
The example below shows  a segment of the source code from {\tt AbelianGroup.h}

\footnotesize
\begin{verbatim}
// --------------- AbelianGroup --------------- //

class AbelianGroup : public ObjectOf<AbelianGroupRep>
{

public:

. . .

void computeCyclicDecomposition( ) {
    change()->computeCyclicDecomposition();
}

. . .

VectorOf<Integer> invariants( ) const {
    return look()->invariants();
}

}
\end{verbatim}
\normalsize


Now the question is how to make  {\tt computeCyclicDecomposition()}
available from
the menu. This would involve writing of a couple of classes which are
correspond to a
certain standard and most of them could be created by copying and changing
lines from similar classes.
Usually these classes are  defined in files which are located in
{\tt magnus/back\_end/SMApps} which contains the
session manager applications. Classes for our problem are defined and implemented
in
{\tt magnus/back\_end/SMApps/include/AbelianInvariants.h} and
{\tt magnus/back\_end/SMApps/src/AbelianInvariants.C}.

The first type of classes to be created are those that inherited from
the class called ARCer
located in {\tt magnus/back\_end/SessionManager/...}. Its purpose is
to control
the execution of a child process that will do actual computations and to obtain
the
result for the corresponding computation manager (see below).
This is the class that in turn calls the class that implements our algorithm
{\tt AbelianGroup::computeCyclicDecomposition()}.  The definition of 
such a class
from {\tt AbelianInvariants.h} is shown below.

\footnotesize
\begin{verbatim}
// ---------------------- AbelianInvariantsARCer --------------------------- //

class AbelianInvariantsARCer : public ARCer
{
public:
////////////////////////////////////////////////////////////////////////
// Constructors:                                                      //
////////////////////////////////////////////////////////////////////////

AbelianInvariantsARCer( ComputationManager& boss )
    : ARCer( boss ), AG( 0 ) { }

~AbelianInvariantsARCer( ) { delete AG; }

/////////////////////////////////////////////////////////////////////////
// Accessors:                                                          //
/////////////////////////////////////////////////////////////////////////
// set parameters. Usually used before calling takeControl()
void setArguments( const FPGroup&, bool );

// Inherited from ARCer:
// bool takeControl( );
// when called for the first time, forks a child process and starts
computation
// returns true when  finished with computations, returns false if one
// unit of time (arc) was used but algortihm has not finished yet.

AbelianGroup getCyclicDecomposition();
// returns Abelian group with computed decomposition

/////////////////////////////////////////////////////////////////////////
// IPC interaction members:                                            //
/////////////////////////////////////////////////////////////////////////

void runComputation( );
// called from takeControl() after a child process started
// all time consuming computations should go here

void writeResults( ostream& );
// write data that should be available for the SessionManager
//  to a stream. Executed in the child process after  runComputation( )
// is finished and before the child process quits.

void readResults( istream& );
// read data from a stream Called from the parent process. Used to
// read results saved by writeResults(...)

private:

/////////////////////////////////////////////////////////////////////////
// Data Members:                                                       //
/////////////////////////////////////////////////////////////////////////
FPGroup G;
bool bMakeFile;
AbelianGroup* AG;
};
\end{verbatim}
\normalsize

The implementation is quite simple and usually follows the same pattern.
The following is shows the methods for {\tt AbelianInvariantsARCer}.
The purpose of each method in the class should be clear from the  comments.


\footnotesize
\begin{verbatim}
// ---------------------- AbelianInvariantsARCer --------------------------- //

void AbelianInvariantsARCer::setArguments( const FPGroup& group,
bool makeFile
)
// set/reset arguments before running the computation
{
delete AG;
AG = new AbelianGroup( group, makeFile );
}

void AbelianInvariantsARCer::runComputation( )
// this function will be called from the child process
{
#if SAFETY > 0
if( AG == 0 )
    error( "void AbelianInvariantsARCer::runComputation( ) : "
          "Have to initialize the group first");
#endif

AG->computeCyclicDecomposition(); // do the computations
}


void AbelianInvariantsARCer::writeResults( ostream& out )
// write the result  from the child's process (here AbelianGroup
//  with computed invariants) after the  computation is finished
{
out < *AG;
}


void AbelianInvariantsARCer::readResults( istream& in )
// read the result by the parent process after the child process quit
{
in > *AG;
}
\end{verbatim}
\normalsize


The next type of classes are computation managers. These are
classes inherited from the
{\tt ComputationManager} class which  is located in directory
{\tt magnus/back\_end/SessionManager/}.
The objects of such classes provide communication between
the session manager and child
processes forked from an ARCer object.
For example, it is the computation manager's responsibility to send
start, stop, suspend or terminate commands to the corresponding ARCer object.
The computation manager receives the amount of time (number of arcs) 
that the child
process is allowed to work on the problem
before the control will be given to another computation manager
(and therefore another child process) and makes sure that the child
process uses this time
correctly. The session manager receives information from
the computation manager to be
represented on the screen or to be saved in information centers.


The code below shows the definition and implementation of
the main methods of the computation manager for computing abelian invariants.
The definitions of computation managers are usually very simple
and do not vary much.

\footnotesize
\begin{verbatim}
// Declarations in AbelinaInvariants.h
// -------------------------- AbelianInvariants ---------------------------- //

class AbelianInvariants : public ComputationManager
{
public:

/////////////////////////////////////////////////////////////////////////
// Constructors:                                                      //
/////////////////////////////////////////////////////////////////////////

AbelianInvariants(class AbelianInvariantsProb&);

      *    *    *

private:

/////////////////////////////////////////////////////////////////////////
// Data Members:                                                   //
/////////////////////////////////////////////////////////////////////////

class AbelianInvariantsProb& theSupervisor; // parent supervisor (see below)

AbelianInvariantsARCer arcer; // ARCer object
};
\end{verbatim}
\normalsize

\scriptsize
\begin{verbatim}
// Implementation in AbelinaInvariants.C
// -------------------------- AbelianInvariants ---------------------------- //

AbelianInvariants::AbelianInvariants(AbelianInvariantsProb& prob):
    theSupervisor( prob ),
    arcer( *this )
{
// initialize ARCer parameters
arcer.setArguments( theSupervisor.getSMFPGroup().getFPGroup(), true );

adminStart();  // start computation manager
}


void AbelianInvariants::takeControl( )
{
if ( freeARCs() > 0 )    
    // if there are free arcs (time units) try to do computations
    //
    // When called for the first time, arcer.takeControl() forks a child process
    //  and starts computation, returns true if finished with computations
    // false if one arc is used but computation has not been finished
    if( arcer.takeControl() ) {

     // put cyclic decomposition into Group information center
     theSupervisor.getSMFPGroup().gic.putHaveCyclicDecomposition
       ( arcer.getCyclicDecomposition() );

     adminTerminate();   // terminate the computation manager
    }
    else
     usedOneARC();  //  tell SessionManager  that one ARC was used
}

\end{verbatim}
\normalsize

There is one type of computation manager that needs special attention.
These are called {\em supervisors}.
The difference between a supervisor and a regular
computation manager is that,
instead of managing directly child processes started from ARCers,
a supervisor manages a number of other computation managers. In addition
a supervisor has
a method called {\tt viewStructure(...)} which defines the corresponding
{\em problem window} which user actually sees on the screen.

\begin{figure}[h]
%\centerline{ \epsfig{file=OrderProbWin.eps} }
\centerline{ Missing EPS file }
\caption{An example of the problem window.}
\label{OPWin}
\end{figure}

Figure \ref{OPWin} shows the problem window for the ``Find the order of a
group" problem. Regular computation managers correspond to the lines 
with algorithm
names and little scroll bars on the right side. The supervisor for this problem
corresponds to the whole window.
It is the supervisor's responsibility to distribute free arcs (time units)
between the computation managers, define the title and text of the 
window and so
on.  The implementation of the methods for the abelian invariants problem
supervisor is given below.

\footnotesize
\begin{verbatim}
// ------------------------- AbelInvariantsProb ---------------------------- //
AbelInvariantsProb::AbelInvariantsProb(SMFPGroup& G): 
        Supervisor( true ),
        theGroup( G ),
        // AbelianInvariants Computation Manager:
        abelianInvariants( *this )
{
// check if group information center has the decomposition
// already. If it does than set the fast result flag
if( theGroup.gic.haveCyclicDecomposition() )
     resultIsFast();
}

void AbelInvariantsProb::takeControl( )
{
// if result is fast just print the decomposition on the screen and terminate
if( fastResult() ) {
    LogMessage( theGroup.gic.getCyclicDecompositionMessage() ).send();
    adminTerminate();
    return;
}

// if our computation manager stoped
if ( theGroup.gcm().abelianInvariants->state() == TERMINATED ) {
    LogMessage( *this, theGroup.gic.getCyclicDecompositionMessage(), 2 ).send();
    // put the answer on the screen
    adminTerminate();
}
}

void AbelInvariantsProb::viewStructure(ostream& ostr) const
{
char *temp;
if( theGroup.getCheckinType() == SMFPGroup::ABELIAN )
    temp = "";
else
    temp = " abelianized";

// define the problem window
ProblemView pv( ostr, oid(),
                "Compute the canonical decomposition",  
                //title of the window
                Text("Compute the canonical decomposition of")
                //text below the title
                 + Name( theGroup ) + Text(temp),
                 helpID( "AbelInvariantsProb", theGroup ), 
                // help file id
                 "Can dec" 
                // text that will appear below the icon in
                // the workspace which correspond
                // to this problem window
               );
pv.startItemGroup();

// insert computation manager: title-"For the use of this problem"
// time unit resources ID - abelianInvariants.arcSlotID()
// default number of arcs - 100
pv.add("For the use of this problem", abelianInvariants.arcSlotID(), 100);
pv.done();
}

\end{verbatim}
\normalsize

It remains to  make  our algorithm available from the menu.
As in the case of a fast problem  we need to change
{\tt magnus/back\_end/SMApps/src/menuDefns.C}.
First, find the menu which corresponds to the Tools menu for abelian
groups and provide an {\tt add()} method with a
proper {\tt Ctor} object as was described above.
The corresponding part of {\tt menuDefns.C} with comments  is shown below.

\footnotesize
\begin{verbatim}
Menu1<SMFPGroup> M( Menu::TOOLS );
      // create menu object for one selected object of type
      // SMFPGroup  in the workspace and
      // with items located in the Tool menu

. . .

   M.startItemGroup( CheckinType( 1 ) == SMFPGroup::ABELIAN );
     // start group of items which correspond
     // to abelian groups

. . .

   M.add( Ctor1<AbelInvariantsProb,SMFPGroup>(),
         Text("Compute the canonical decomposition of") + Name(1) );

     // add an item, called Compute the canonical
     // decomposition of G, where G is the
     // name of the group object in the
     // workspace and create an object of
     // AbellianInvariantsProb class when
     // this item selected from the menu

. . .

   M.done();
\end{verbatim}
\normalsize

After all these steps and compilation of \nsmagnus' session manager
the abelian invariants algorithm should be available from the menu.




\subsection{Adding Parameters}

Sometimes it is useful to have additional user defined parameters
passed to algorithms. It is possible to create a problem window
with  a predefined set of parameters which can be changed during
pauses in the execution of the algorithm. 
Parameters can be changed only in
an idle state of the problem (that is, before it started or when the
``Suspend" button  was pressed).

When creating a problem window with parameters, a slightly different
procedure is used to define the window itself. 
Such windows are defined by creating
obects of class {\tt EnumeratorProblemView} instead of usual
{\tt ProblemView}.
Similar to the above, we create an object of
class {\tt EnumeratorProblemView} inside the body of
{\tt Supervisor::viewStructure(ostream ostr) const}. 

Slidable bars in the window,
corresponding to computation managers are
added by calling the method 
{\tt EnumeratorProblemView::addProblem(const Subproblem)}.
To add parameters we call
{\tt EnumeratorProblemView::add\-Parameter(const ViewContent)}.
Here {\tt ViewContent} is a parent class for all classes which define
   parameter objects. For now the following classes are available
for defining parameters in the problem window:
\begin{itemize}
\item {\tt class EditInteger}. Creates a field for entering an integer
number. Constructor:\\
   {\tt EditInteger( const Chars name,const Chars text,int change =
0,int def = 0, int minV = 0,int maxV = MAXINT)} \\
Here {\tt name} is the unique name (ID) of the parameter that will be
used later for
referencing , {\tt text} is the text that will appear infront of the
edit field,{\tt change} is a boolean which 
if true will alow to change the parameter
after the problem was started, {\tt def} is the default value,
   {\tt minV} is the minimal value, and {\tt maxV} is the maximal value.
\item {\tt class EditBool}. Creates a check box which if checked corespond
to the truth value of the parameter. Constructor: \\
{\tt  EditBool( const Chars name,const Chars text,int change = 0,int
def = 0)} \\
Again, {\tt name} - unique referencing ID, {\tt text} -
corresponding text in the
window, {\tt change} - if true will allow to change parameter after
the problem has started, {\tt def} - default value (true, false).
\item {\tt class EditText}. Creates a text edit field. Constructor: \\
{\tt EditText( const Chars name,const Chars text,int change =
0,Chars def = "")}\\
Parameters have the same meanings as above
\item {\tt class EditWord.} Creates a field where a word in geerators
of a given free group can be entered. Constructor: \\
{\tt EditWord( const Chars name,const Chars text,int change =
0,const FreeGroup f = FreeGroup(), const Word w = Word() )} \\
{\tt name}, {\tt text} and {\tt change} same as above. Free group {\tt f}
defines a set of generators (actually their names) used to parse the
entered word. {\tt w} - default word.

\end{itemize}

Parameters above could be grouped together. There are three  classes
which allow you to group parameters:
\begin{itemize}
\item {\tt class  GroupWindow}. Defines a group
object. Parameters are added through a sequence of calls of
{\tt GroupWindow::add(const ViewContent)} method.
\item {\tt class RadioButton.} Creates a radio buton object in the
window. Usually used in groups to select something from a number of choices.
Constructor:\\
{\tt  RadioButton(const Chars name, const Chars text, int change = 0)} \\
Class is ihereted form {\tt GroupWindow}.
Parameters grouped by the object of this
class wll appear on the screen only when the corresponding
radio button is selected. If no parameters added to the group then
object behaves as a regular radio button.
\item {\tt class  RadioButtonGroup}. This class
is inhereted from {\tt GroupWindow} and groups a number
of objects of the {\tt RadioButton} class.
\end{itemize}

Current values of parameters for a particular window can be obtained
by executing  {\tt Supervisor::getParameters()} method
of the corresponding {\tt Supervisor} object.
It returns a Parameter Structure which keeps all parameters together
with their current values. Value for  a particular parameter can
be obtained through {\tt ParameterStructure::getValue(const Char name)}
method, where {\tt name} is a unique reference ID of the parameter.


When a problem window apears on the screen, user can open a  parameter
window by clicking  on the {\bf Parameters} menu item in the
problem window.

To read values of the modified parameters, we will usually need to
overload {\tt start()} and {\tt resume()} virtual methods of the
{\tt ComputationManager} class. These methods are called right after
the  {\bf Start} or {\bf Resume} buttons of the problem window
are selected. This will allow us to check values of
modified parameters and update problem's settings when the process executing
the algorithm is idle.

Example of the supervisor which corresponds to the problem window with
parameters is given below

    \footnotesize
    \begin{verbatim}
// ------------------- SupervisorWithParameters ------------------- //

void SupervisorWithParameters::takeControl( )
{
     // some code here

       .  .  .

}


void SupervisorWithParameters::viewStructure(ostream& ostr) const
{

     EnumertatorProblemView pv(oid(),"Supervisor with parameters",
                  Text("Select to define some parameters) ,
                  helpID("SP",theGroup),
                   Text("sp")
                  );

     // add an integer parameter, name ``I1'' with default
     // value 10. This parameter can be changed after the resume
     // button is selected
     pv.addParameter(EditInteger("I1","Integer parameter",1,10,0,100));

     // reate a radio button  object
     RadioButton RB1("RB1","Radio buttom 1");
     RadioButton RB2("RB2","Radio button 2");

     // add integer parameters to each radio button above
     RB1.add(EditInteger("RB1_I1","Integer for radio button 1",0,10,1,10000));
     RB2.add(EditInteger("RB2_I1","Integer for radio button 2",0,10,1,10000));

     // group radio buttons
     RadioButtonGroup rbGroup("rbGroup");
     rbGroup.add(RB1);
     rbGroup.add(RB2);

     // add group to the parameters
     pv.addParameter(rbGroup);


     // add a computation manager
     pv.addProblem( Subproblem("Sp1",ThisARCSlotID(),
                  "For the use of this problem",40));


    // submit configuration to the session manager
     submit(ostr, pv);

}

// this method is called when the Start button of the
// corresponding window is pressed but before actual
// computation is started
void SupervisorWithParameters::start( ) {

    // Get parameters:
    // first integer
    int I1 = getParameters().getValue("I1");
         .  .  . // do something with it


    // get which of the radio buttons was selected
    Chars t = getParameters().getValue("rbGroup");

     int buttonInteger;
    // assign to buttonInteger value of the integer
    // parameter for the selected button
    if (t == Chars("RB1")){
         buttonInteger = getParameters().getValue("RB1_I1");
      } else {
         buttonInteger = getParameters().getValue("RB2_I1");
      }
     .  .  . // do something here

}

void WordEnumeratorProblem::resume( )
{
     // only first integer parameter was defined in a such
     // way that it is possible to change its value after
     // the Suspend button is pressed.

     int I1 = getParameters().getValue("I1");
       .  .  .
}


    \end{verbatim}
    \normalsize



\subsection{Object's information centers}

There are many  algorithms which compute  information that can also
be used to solve other problems about a particular algebraic object.
If these are time consuming computations, then we want to avoid repetition
and save the computed information (if we can) so it could be used
later on. 
This is the purpose of the {\em Information Centers}.

Almost every algebraic object has an associated class  which
keeps the previously computed information. The definitions of such classes
could
be found in {\tt back\_end/SessionManager/include/} and their names
usually consist
of three letters. First indicate the object and second two are {\em IC}
for 
``Information Center'' (i.e. {\em GIC} for ``Group Information Center'',
{\em SIC}
for ``Subgroup Information Center'' and so on).

Data members of the information center classes are   object specific. The
decision
whether to put some information into a {\em IC}  is usually 
 based on how difficult it is to compute such information. 
 For a given piece of information, there are up to three kinds of
functions needed:
\begin{itemize}
\item The member function through which the information is reported 
\item The member function via which information's existence is queried 
\item The member function via which the data is accessed
\end{itemize}

Functions are named according to the following scheme:
\begin{itemize}
\item for Trichotomy the report function name is\\
{\em putIs$<$property\_name$>$}.
Existence query - {\em is$<$property\_name$>$}.
\item for boolean properties the report function name is\\
{\em putHave$<$property\_name$>$}.
Existence query - 
{\em have$<$property\_name$>$}.
\item properties which contain a special data structures have the
following associated function names:
{\em putHave$<$property\_name$>$} - report a
property; {\em have$<$property\_name$>$} - 
query the existence; {\em get$<$property\_name$>$} - acquire the property
data.
\end{itemize}


Below  is an example of how  {\em GIC} is used to access the canonical
decomposition of an Abelian group.
\footnotesize	
\begin{verbatim}
// the code below waits until the canonical decomposition
// is computed and computs the order of the given group


AbelianOrderProblem::AbelianOrderProblem(SMFPGroup& G)
    : Supervisor( ),
      theGroup( G ), // save the group object
      abelianInvariants( *this )
{
  // if gic for our group has computed cyclic decomposition or the order
  // of the group then result is fast, i.e. no problem window
  // will appear on the screen 
  if ( theGroup.gic.haveOrder() ||
theGroup.gic.haveCyclicDecomposition() )
    resultIsFast();
}


void AbelianOrderProblem::takeControl( )
{
  // takeControl() is called periodically by the Session Manager
  // if order of the group is availbale from the GIC then print 
  // it to log and exit
  if( theGroup.gic.haveOrder() ) {
    LogMessage( theGroup.gic.getOrderMessage() ).send();
    adminTerminate();
    return;
  }

  // if canonical decomposition is available in GIC, then compute
  // the order (it is fast if we have invariants), save the order in the
GIC
  // report and exit 
  if( theGroup.gic.haveCyclicDecomposition() ) {

    AbelianGroup& A =
(AbelianGroup&)(theGroup.gic.getCyclicDecomposition());

    Chars explanation = Text("this follows from the canonical
decomposition of")
      + Name(theGroup);
    theGroup.gic.putHaveOrder(A.order(), explanation);
    
    if( !fastResult() )
      LogMessage(*this, theGroup.gic.getOrderMessage(), 2).send();
    
    adminTerminate();
  }
}

\end{verbatim}
\normalsize


\subsection{Object's computation managers}

Since Magnus allows one to execute several tasks simultaneously, it can
happen that the same computation is started for two separate problems.
For example, both ``Compute the order of an Abelian group'' 
and ``Compute the order of an element of an Abelian group'' will require
the canonical decomposition for that group. If we start the two problems
simultaneously, we will have two instances of the same algorithm which
is wasting of the  memory  and cpu time. 

To avoid such situations, all
computation managers (called {\em Subordinate computation Managers} )
that can be used in different Supervisors (i.e.
started in several Problem Windows) are assembled in intermediate
Supervisor
objects, called ``object's computation managers'' (i.e. {\em GCM} 
for ``Group computation manager'', {\em SCM} - Subgroup
computation manager and so on). When Magnus starts there only one instance
of {\em GCM} ({\em SCM}, {\em WCM} ...) created. 

All supervisors reference
the same instance of the desired computation manager which assures
that there is only one instance of the particular algorithm is running at
the same time. Moreover, if the algorithm was started for one of the
problems, all the problems started after that will use the same longest
running computation for their purposes.

The following is the list of computation managers that are common
for many problems and are contained in  the {\em GCM}:

\footnotesize
\begin{verbatim}
//--------------------- GCM-------------------------//

class GCM : public Supervisor
{
public:

  GCM( class SMFPGroup& G );

  .  .  .

  // Subordinate Computation Managers:                                  

  Subordinate<GCM,NormalClosure> normalClosure;

  Subordinate<GCM,AbelianInvariants> abelianInvariants;

  Subordinate<GCM,AbelianPrimes> abelianPrimes;

  Subordinate<GCM,AbelianRank> abelianRank;

  Subordinate<GCM,KBSupervisor> kbSupervisor;

  Subordinate<GCM,AGSupervisor> agSupervisor;

  Subordinate<GCM,NilpotentQuotients> nilpotentQuotients;

  Subordinate<GCM,NGcomputeBasis> computeBasis;

  Subordinate<GCM,ToddCoxeter> theToddCoxeter;

  Subordinate<GCM,HToddCoxeter> ghToddCoxeter;

  .  .  .

};
\end{verbatim}
\normalsize

The use of subordinate computation managers is almost the same as for
regular
computation managers, except that the constructor requires both a
reference
to the calling Supervisor object and a reference to the subordinate 
computation manager in the object's computation managers class.

Example:
\footnotesize
\begin{verbatim}
// definition of the subordinate computation manager in 
// the Abelian Invariants
//

class AbelInvariantsProb : public Supervisor
{
public:
  .  .  .

  MirrorSubordinate abelianInvariants;

};


// initialization of the subordinate computation manager
AbelInvariantsProb::AbelInvariantsProb(SMFPGroup& G)
  : Supervisor( true ),
	theGroup( G ),
 	abelianInvariants( *this, G.gcm().abelianInvariants )
{
  if( theGroup.gic.haveCyclicDecomposition() )
    resultIsFast();
}


// everython else is the same as for regular computation manager
void AbelInvariantsProb::takeControl( )
{
  if( fastResult() ) {
    LogMessage( theGroup.gic.getCyclicDecompositionMessage() ).send();
    adminTerminate();
    return;
  }

  // if CM abelianInvariants in GCM has stoped then report the
  // cyclic decomposition
  if ( theGroup.gcm().abelianInvariants->state() == TERMINATED ) {
    LogMessage( *this, theGroup.gic.getCyclicDecompositionMessage(), 2
).send();
    adminTerminate();
  }
}

\end{verbatim}
\normalsize


\end{document}



\end{document}

