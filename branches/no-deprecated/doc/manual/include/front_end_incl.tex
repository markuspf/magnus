%% This is AMSLaTeX source, included by manual_incl.tex.


%%--------------------------------------------------------------------
\section{The Front End}

%%....................................................................
\subsection{Messages}\label{FEMessages}

\newcommand{\msg}[2]{
\vspace{10pt}
\verb/#1/ {\em #2}
\vspace{10pt}
}

\newcommand{\marg}[1]{{\em #1\/}}


The \FE\ and \SM\ exchange messages. The syntax and semantics of the
messages which the \FE\ can accept are fixed; they are described in
this section.

In contrast, the \FE\ knows as little as possible about the messages it
sends to the \SM. Rather, the \FE\ is supplied by the \SM\ with {\em
message templates} for each kind of message it will need to
send. These templates contain {\em formal parameters} which the \FE\
replaces, at the time the message is sent, with {\em actual
parameters}. The syntax and semantics of formal parameters are also
fixed, and are described below along with each message type.

A message from the \SM\ to the \FE\ is an ASCII string containing no
non-printable characters, newlines, linefeeds, or tabs. It consists of
a {\em message name} followed by an ordered sequence of zero or more
arguments. A message name matches {\tt '[a-zA-Z][a-zA-Z\_]*\_'}. Each
argument is preceded by at least one space. The argument syntax is
described with each message, below.

The following sections describe each message that the \FE\ can accept,
grouped by function.  Because of the semantic interconnections between
the various messages, the sections are not in the temporal order in
which the messages are sent.

Except where noted, the arguments of messages to the \FE\ are keyword
arguments. For example, suppose the message named {\tt foo\_} takes
two arguments: an integer and a constant string. The syntax is
specified like:

\msg{foo\_}{theInt theString}

In this case \marg{theInt} and \marg{theString} are keyword
identifiers, so the actual message could be, for example:

{\tt foo\_ theInt 7 theString \{Hello world\}}

{\tt foo\_ theString \{Hello world\} theInt \{7\}}


%% . . . . . . . . . . . . . . . . . . . . . . . . . . . . . . . . . .
\subsubsection{Checking in an Object}\label{checking_in}

The \SM\ can pass a message which tells the \FE\ to create an image in
the workspace of an object which exists in the \SM. The message has
the form

\msg{create\_}{objectID typeID iconID name dependencies properties viewStructure
definition}

An \marg{objectID} is a unique positive integer assigned to the object
by the \SM. It is the only definitive reference to the object shared
by the \FE\ and \SM.

A \marg{typeID} is a positive integer assigned to the object by the
\SM. It is used to identify what actions may be taken on a set of objects.

An \marg{iconID} is one of the symbols: {\tt group}, {\tt subgroup},
{\tt elt}, {\tt map}, {\tt enumerator}, {\tt problem}.

A \marg{name} is the text which is to be displayed beneath the icon in
the workspace. It must be delimited by {\tt\{\}} if it contains
whitespace.

The \marg{dependencies} is a list of the form
{\tt\{}\marg{objectID}{\tt *\}}. In this case \marg{objectID} is not a
keyword identifier.

The \marg{properties} is a list of the form {\tt\{\{}\marg{key}
\marg{datum}{\tt\}+\}}, where \marg{key} and \marg{datum} are strings
matching {\tt '[a-zA-Z0-9(), ]+'}.  By convention, all spaces are
stripped out by the \FE.  The \marg{key-datum} pairs are stored by the
\FE\ on behalf of the \SM\ for later use. The purpose is to allow the
\SM\ to endow the \FE\ with certain knowledge (about objects, in this
case), which the \FE\ can use as directed without knowing anything
about it.  See \S\ref{hottext} for how to use this stored information.

A \marg{viewStructure} has the form: {\tt\{}\marg{viewBuilder}
\marg{viewParameters}{\tt\}}.  A \marg{viewBuilder} is the name of a
predefined view window builder (\S\ref{predefined_views}). The
\marg{viewParameters} are the parameters expected by the
\marg{viewBuilder}. The \marg{viewBuilder} may be {\tt\{\}} if no view
is defined.

The \marg{definition} is supplied only when the \marg{name} is the
empty string, and the object is not a problem object. This is the case
when a new object is constructed by the \SM\ in response to an end
user query.


%% . . . . . . . . . . . . . . . . . . . . . . . . . . . . . . . . . .
\subsubsection{{\sf Hot} text}\label{hottext}

The \FE\ accepts the message

\msg{define\_}{key datum}

\noindent
which tells it to associate \marg{datum} with \marg{key} in the same
sense as the \marg{properties} argument of the {\tt create\_}
message. The syntax and semantics are the same.

Many of the messages in the following sections may contain substrings
which we call {\em{\sf Hot} text}. This means that the substring may
contain substrings of the form {\tt [\_value\_ }\marg{key}{\tt]}, which
evaluate to the datum most recently associated with the \marg{key} in
a \marg{properties} argument of a {\tt create\_} message, or a {\tt
define\_} message, or the value {\tt 0} if there is no such
association. {\sf Hot} text may also contain substrings
{\tt[\_name\_ }\marg{expr}{\tt]}, where \marg{expr} evaluates to an
\marg{objectID}. This evaluates to the end user supplied name of the
object.

{\sf Hot} text may not contain characters {\tt[]} except as above,
{\tt""} except as outer delimiters, or {\tt\$\{\}$\backslash$} except
as part of variable references or delimiters, discussed below.


%% . . . . . . . . . . . . . . . . . . . . . . . . . . . . . . . . . .
\subsubsection{Menu Definitions}\label{menu_defn}

This section describes the messages accepted by the \FE\ for defining
workspace menus. We need a number of preliminaries first.

Given a selection in the workspace, the front end provides the end
user with a (possibly hierarchical) menu of actions which are
appropriate to that selection.

A given action has a {\em signature}, which is a bag of \marg{typeIDs}
$\{t_1,\dots t_n\}$. Let the bag of the \marg{typeIDs} of the selected
objects be $\{t'_1,\dots t'_n\}$. The action cannot be appropriate to
the selection unless there is a permutation $\sigma\in S_n$ such that
$t_{i}=t'_{\sigma(i)}$ for $1\le i\le n$.

In addition, the action may not be appropriate unless some condition
on the selected objects is met. The information needed to check these
conditions was stored in the \FE\ by the \SM\ via the
\marg{properties} argument of the {\tt create\_} message for each
object.

If an action is appropriate, the \FE\ must post a menu item for it,
and send a corresponding message to the \SM\ if that item is
chosen. The text of the menu item will need to refer to various of the
selected objects by \marg{name}, and the message will refer to them by
\marg{objectID}. Now suppose the signature of the action is
$\{t_1,\dots t_n\}$, and that the selected objects can be indexed
$\{O_1,\dots O_n\}$ so that the \marg{typeID} of $O_i$ is $t_i$. We
may consider an index $1\le i\le n$ to be a formal parameter of the
action, and $O_i$ to be the corresponding actual parameter.

The \SM\ may refer to the \marg{name} of the $i$th actual parameter in
{\sf Hot} text with the symbol {\tt \$\{name}$i${\tt\}}. It may refer
to the \marg{objectID} with the symbol {\tt \$\{OID}$i${\tt\}}.

At last, we have the syntax of a menu definition message:

\msg{menu\_defn\_}{menu signature itemGroups}

A \marg{menu} is one of the symbols {\tt checkin}, {\tt tools}. It
indicates under which toplevel workspace menu the given menu items are
to be posted.

A \marg{signature} has the form {\tt\{}\marg{typeID}{\tt*\}}. The
\marg{typeID}s are sorted in increasing order. In this case
\marg{typeID} is not a keyword identifier.

An \marg{itemGroups} has the form {\tt\{}\marg{itemGroup}{\tt+\}}.
 There are no keyword arguments at this level or below.

An \marg{itemGroup} has the form {\tt\{}\marg{condition}
{\tt\{}\marg{item}{\tt+\}\}}.

A \marg{condition} is any string which can be accepted by the {\tt
Tcl} procedure {\tt expr}. It is {\sf Hot} text. It may contain
substrings ``{\tt\$\{OID}$i${\tt\}}''. Note that {\tt expr} accepts
the operators {\tt \&\&}, {\tt ||}, {\tt !}, {\tt ==}, {\tt !=}, which
have the same meaning and precedence as in {\tt C}. It allows
parentheses. It represents true and false by {\tt 1} and {\tt 0}.

An \marg{item} has one the forms
\marg{realItem}, {\tt s}, or
{\tt\{c} \marg{itemText} {\tt\{}\marg{realItem}{\tt+\}\}}.
The {\tt s} \marg{item} is a menu item separator, and the {\tt\{c...\}}
\marg{item} is a cascaded menu.

A \marg{realItem} has the form
{\tt\{}\marg{itemText} \marg{action} \marg{message}{\tt\}}.

An \marg{itemText} is {\sf Hot} text, delimited by {\tt\{\}}'s, and may
contain substrings ``{\tt \$\{name}$i${\tt\}}''. Example:\break
{\tt\{Is \$\{name1\} trivial in [\_name\_ [\_value\_ parent(\$\{OID1\})]]?\}}.

An \marg{action} is the name of one of a number of predefined
procedures (documented in \S\ref{predefined_procedures}) which is to
be called before the \marg{message} is sent to the \SM. Such
procedures usually have the side-effect of defining new variables
which can be referenced in the \marg{message} body. The \marg{action}
may be {\tt\{\}}.

A \marg{message} is {\sf Hot} text delimited by {\tt\{\}}'s, and may
contain substrings ``{\tt \$\{OID}$i${\tt\}}'' and any other variable
references which are valid as a result of the \marg{action}. If the
\marg{message} is empty, i.e. {\tt\{\}}, the menu item is posted but
disabled.


%% . . . . . . . . . . . . . . . . . . . . . . . . . . . . . . . . . .
\subsubsection{Global Message Templates}\label{global_defns}

The \FE\ responds to a number of events by sending essentially the
same message to the \SM. The \SM\ may initialize templates for these
messsages at startup with the following messages to the \FE. The
\marg{message} is always delimited by {\tt\{\}} if it contains
whitespace:

\msg{init\_quit\_msg\_}{message}

The \marg{message} is to be sent to the \SM\ when the end user wishes
to quit. The front end does not exit until it gets the
{\tt confirm\_quit\_} message, described below.

\msg{init\_view\_req\_msg\_}{message}

The \marg{message} is to be sent to the \SM\ when the end user
requests a view on a subproblem (see \S\ref{predefined_views}).  It is
{\sf Hot} text, and may contain the references {\tt\$\{viewID\}} to
the enclosing view, and {\tt\$\{probViewID\}} to the subproblem for
which the view was requested.

\msg{init\_arc\_msg\_}{message}

The \marg{message} is to be sent to the \SM\ when the ARCs given to
any subproblem may have been changed by the end user.  It can only be
sent when the enclosing problem is in a suspended state, or has not
been started yet. It is {\sf Hot} text, and may contain the references
{\tt\$\{probViewID\}}, {\tt\$\{ARCSlotID\}}, {\tt\$\{value\}}.

\msg{init\_start\_msg\_}{message}

The \marg{message} is to be sent to the \SM\ when the end user starts
a problem.  It is {\sf Hot} text, and may contain the reference
{\tt\$\{viewID\}} to the \marg{viewID} of the problem.

\msg{init\_suspend\_msg\_}{message}

The \marg{message} is to be sent to the \SM\ when the end user
suspends a problem.  It is {\sf Hot} text, and may contain the reference
{\tt\$\{viewID\}} to the \marg{viewID} of the problem.

\msg{init\_resume\_msg\_}{message}

The \marg{message} is to be sent to the \SM\ when the end user resumes
a problem.  It is {\sf Hot} text, and may contain the reference
{\tt\$\{viewID\}} to the \marg{viewID} of the problem.

\msg{init\_terminate\_msg\_}{message}

The \marg{message} is to be sent to the \SM\ when the end user terminates
a problem.  It is {\sf Hot} text, and may contain the reference
{\tt\$\{viewID\}} to the \marg{viewID} of the problem.

\msg{init\_delete\_msg\_}{message}

The \marg{message} is the prefix of the message to be sent to the \SM\
when the end user deletes one or more objects. The \FE\ appends the
\marg{objectID}s of the objects to be deleted, separated by
spaces. The object list is closed under dependency.


%% . . . . . . . . . . . . . . . . . . . . . . . . . . . . . . . . . .
\subsubsection{General Messages}\label{general_msgs}

The \FE\ accepts the following general-purpose messages:

\msg{fatal\_error\_}{message}

A fatal error, described in the \marg{message}, has been detected by
the \SM.

\msg{confirm\_quit\_}{}

This tells the \FE\ that the \SM\ has completed its shutdown, so
the \FE\ may now exit.

\msg{confirm\_start\_}{viewID}

This tells the \FE\ that the \SM\ has started the problem with
\marg{viewID}.

\msg{confirm\_suspend\_}{viewID}

This tells the \FE\ that the \SM\ has suspended the problem with
\marg{viewID}. The \FE\ may get this message even when the end user
has not suspended the problem, e.g., when all of the ARCs for the
problem have been used up.

\msg{confirm\_resume\_}{viewID}

This tells the \FE\ that the \SM\ has resumed the problem with
\marg{viewID}.

\msg{terminate\_problem\_}{viewID}

This tells the \FE\ to put the problem in a state similar to
suspended, except that it cannot be resumed. This message is sent when
the problem has finished its work.

\msg{confirm\_delete\_}{{\tt\{}objectID{\tt+\}}}

This tells the \FE\ that the \SM\ has deleted the objects with the
given \marg{objectID}s. In this case \marg{objectID} is not a keyword
argument.

\msg{arc\_update\_}{viewID ARCSlotID value}

This tells the \FE\ that the ARC value of \marg{ARCSlotID} in
\marg{viewID} has changed to \marg{value}. These messages are
correctly interpreted whether the view is open or not, and are quietly
ignored when there is no view at all with \marg{viewID} (as happens,
e.g., when a purely internal active object sends an ARC update).

\msg{syntax\_error\_in\_defn\_}{name errMesg}

This tells the \FE\ that a previous attempt to check in an object with
name \marg{name} failed due to a syntax error in the definition of the
object. The error message is in \marg{errMesg}. The \FE\ is
responsible for reposting the same kind of dialog, with the same
entries as when the dialog was last confirmed, along with the error
message. If the new dialog is confirmed, with any or all entries
possibly different, the \FE\ must use the same message template as in
the previous check in request.

The \marg{errMesg} has the form {\tt\{$n$ \{{\em descr}\}\}}, where
$n$ is the index of the problem, and {\em descr} says what is wrong.


\msg{post\_to\_log\_}{viewID text level}

This tells the \FE\ to append the {\sf Hot} \marg{text} at the end of
the log of \marg{viewID}. When \marg{level} has value $1$, the
\marg{text} is posted to both the view log and the session log; when
\marg{level} is $2$, only the former.

\msg{type\_name\_}{name prefix typeID}

This is an experimental message, sent at startup, via which the \SM\
associates human readable text, the \marg{name}, with a \marg{typeID}.
The \marg{prefix} is the preferred default name prefix for an object
of the given type. Currently, the arguments are not keyword arguments.


%%....................................................................
\subsection{Predefined Procedures}\label{predefined_procedures}

The \FE\ provides a number of procedures which can be used as
\marg{action} arguments in menu definitions (\S\ref{menu_defn}). If
procedure {\tt\_foo\_} takes arguments {\tt bar$_1$}, $\ldots$, {\tt
bar$_n$}, then the \marg{action} must be written

{\tt\{\_foo\_ bar$_1$...bar$_n$\}}.

The arguments are {\em not} keyword arguments. Each of these
procedures has the side-effect that when they exit, they have defined
certain variables for use as substrings in the corresponding
\marg{message} argument of the \marg{realItem} in the menu
definition. The names of the variables are listed with each procedure.

%% . . . . . . . . . . . . . . . . . . . . . . . . . . . . . . . . . .
\subsubsection{Definition Dialogs}

The procedures {\tt\_defineFPGroup\_}~\marg{title},
{\tt\_defineSubgroup\_}~\marg{title}, {\tt\_defineMap\_}~\marg{title},
and {\tt\_defineWord\_}~\marg{title} each bring up a dialog in which
the end user can define the named kind of object. The {\sf Hot} text
\marg{title}, delimited by {\tt\{\}}, says what is being asked
for. Note that, e.g., {\tt\_defineFPGroup\_} could be used for several
group types. The variables defined by each upon exit are
$\backslash${\tt\$\{name\}} and $\backslash${\tt\$\{defn\}}.


%% . . . . . . . . . . . . . . . . . . . . . . . . . . . . . . . . . .
\subsubsection{Parameter Dialogs}

The procedure {\tt\_defineInt\_}~\marg{title}~\marg{low}~\marg{high}
is for prompting the user for an integer parameter to an action such
as `Compute a power of...'. The {\sf Hot} text \marg{title}, delimited
by {\tt\{\}}, says what the integer is for. The arguments \marg{low},
\marg{high} give inclusive bounds on acceptable integers; either or
both may be {\tt\{\}} if there is no bound.  This defines the variable
$\backslash${\tt\$\{n\}} upon exit.


%%....................................................................
\subsection{Predefined Views}\label{predefined_views}


%% . . . . . . . . . . . . . . . . . . . . . . . . . . . . . . . . . .
\subsubsection{The Problem View}

The \FE\ provides a procedure named {\tt problemView} for use as a
\marg{viewBuilder} in a \marg{viewStructure} argument to the {\tt
create\_} command (\S\ref{checking_in}). The \marg{viewParameters} it
expects has the form:

{\tt\{}\marg{viewID} \marg{viewTitle} \marg{viewBanner} \marg{totalARCs}
\marg{subproblems}{\tt\}}

The \marg{viewID} is a positive integer which is the definitive
reference shared by the \FE\ and \SM\ to the problem view, but parts
of the view are independently addressable via other id numbers.

The \marg{viewTitle} is {\sf Hot} text, delimited by {\tt\{\}}, which
titles the problem.

The \marg{viewBanner} is {\sf Hot} text, delimited by {\tt\{\}}, which
describes the problem. The view sizes its banner to display all of
\marg{viewBanner} by counting newlines.

The \marg{totalARCs} has the form {\tt\{}\marg{ARCSlotID}
\marg{value}{\tt\}}, where \marg{ARCSlotID} is a positive integer
which uniquely references a slot for the total number of ARCs consumed
by the problem, and \marg{value} is its initial value. At present, the
\marg{ARCSlotID} may as well be the \marg{objectID} of the problem; as
there are no facilities in the \SM\ (that I can see) for sending ARC
updates to this \marg{ARCSlotID}, the \FE\ computes the new value when
it gets other ARC updates.

The \marg{subproblems} has the form {\tt\{\{}\marg{probViewID}
\marg{name} \marg{state} \marg{ARCSlotID} \marg{value}{\tt\}+\}},
where \marg{probViewID} is a positive integer which uniquely
references the subproblem, \marg{name} is {\sf Hot} text which
describes the subproblem, \marg{state} is {\tt 1} or {\tt 0} to
indicate enabled or disabled, \marg{ARCSlotID} is a positive integer
which uniquely references the slot for the ARCs allocated to the
subproblem, and \marg{value} is the initial ARC value.


%% . . . . . . . . . . . . . . . . . . . . . . . . . . . . . . . . . .
\subsubsection{The Object View}

The \FE\ provides a procedure named {\tt objectView} for use as a
\marg{viewBuilder} in a \marg{viewStructure} argument to the {\tt
create\_} command (\S\ref{checking_in}). The \marg{viewParameters} it
expects has the form:

{\tt\{}\marg{viewID} \marg{heritage}{\tt\}}

The \marg{viewID} is a positive integer which is the definitive
reference shared by the \FE\ and \SM\ to the object view.

The \marg{heritage} is {\sf Hot} text which describes what the object
is, and where it came from. Evaluation of \marg{heritage} is delayed
until the view is (re)displayed, to allow for possible name changes.
Thus the \marg{heritage} of a new word could be

{\tt\{A word in the generators of [\_name\_ 29]\}}




%%....................................................................
\subsection{Geometry Management}

In almost all of the examples in Ousterhout's book, widget paths are
cluttered by the demands of the packer. For example, in order to
ensure that a button is drawn correctly in a window, you might make an
enclosing frame which serves no purpose other than to manage the
geometry of the button. Thus:

\begin{verbatim}
  frame .d9.frame6
  button .d9.frame6.ok -text OK; button .d9.frame6.cancel -text Cancel
  pack .d9.frame6.ok .d9.frame6.cancel -side left
  pack .d9.frame6 -side top
\end{verbatim}

It is certainly tedious to build all of these frames and pack the
widgets into them `by hand'. But much worse is the fact that sometimes
you can not avoid hard-wiring a widget path in a procedure other than
the one which made the widget; the procedure might not even be in the
same file.  This means that whenever you change the geometry of a
window, you must track down all references to its widgets and change
them too.

But there is a simple solution. Every widget except {\tt .} has a {\em
parent}, and, if it is packed, a {\em master} which manages its
geometry. These need not be the same. Above, {\tt .d9.frame6} is both
the parent and master of {\tt ok}, but we can make {\tt .d9} the
parent of {\tt ok} instead, and pack {\tt .d9.ok} in {\tt .d9.frame6}.
That way, all (or at least the `exported') widgets have a canonical
path, while allowing transparent changes to their masters. The above
example becomes:

\begin{verbatim}
  frame .d9.frame6
  button .d9.ok -text OK; button .d9.cancel -text Cancel
  pack .d9.ok .d9.cancel -side left -in .d9.frame6
  pack .d9.frame6 -side top
\end{verbatim}


%% . . . . . . . . . . . . . . . . . . . . . . . . . . . . . . . . . .
\subsubsection{{\tt proc framer}}

The {\tt proc framer} in {\tt general.tcl} relieves some of the tedium
of creating frames and packing widgets into them, though you still
have to think about how to arrange frames to get the effect you want.
The above example becomes

\begin{verbatim}
  button .d9.ok -text OK; button .d9.cancel -text Cancel
  framer .d9 .d9 top {
    left {} {} {
      ok {} cancel {}
    }
  }
\end{verbatim}

Look in, e.g., {\tt options.tcl} for more elaborate examples.

The first argument to {\tt framer}, {\tt parent}, is the parent of all
widgets which are to be packed. It saves having to re-type it for
every widget. The second argument is the highest level master of all
widgets packed by {\tt framer}, and of all frames it creates and
packs. It may as well be {\tt parent}. The third argument gives the
initial {\tt -side} argument to {\tt pack}.  What follows is a list of
2- and 4-tuples.

The 2-tuples are widget names followed by a list of arguments to {\tt
pack}, possibly {\tt\{\}}. These widgets must have parent {\tt
parent}.

The 4-tuples specify frames which may contain slave frames and
widgets. They consist of a {\tt -side} option for the slaves, a list
of {\tt frame} options, a list of {\tt pack} options for the frame,
and a further list of widgets and frame specifications.

As {\tt framer} creates the specified frames, it gives them the names
{\tt f1}, {\tt f2}, etc., and makes {\tt parent} their parent.
